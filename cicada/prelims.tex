\section{Additional Preliminaries}

\paragraph{Special Notation} We will use $n$ as the number of users, $m$ as the number of candidates, and $w$ as the maximum weight to be allocated to any one candidate in a ballot/bid ($n,m,w \in \mathbb{N}$). For simplicity and without loss of generality, we assume the user identities are unique integers $i \in [n]$.
We generally use $i \in [n]$ to index users and $j \in [m]$ for candidates.

\subsection{Voting and auction schemes}

We recall the specifics of FPTP, approval, range, and cumulative voting, along with single-item sealed bid auctions. The cryptographically relevant details of these schemes (i.e., the valid ballots' structure: their domain, Hamming weight, and norm) are summarized in~\Cref{tab:voting_schemes}. We will show how to realize them in the Cicada framework in \Cref{sec:cicada_framework}.

\def\sp{\hspace{1em}}
%%------------------- BEGIN Ballot domain TABLE ----------------------
\begin{table*}[tb]
    \centering
\makebox[\linewidth]{
    \setlength{\tabcolsep}{6pt}
    \setlength{\belowbottomsep}{6pt}
    \begin{tabular}{l@{\hspace{12pt}}ccc}
        \toprule
                        & \textbf{Submission domain}    & \textbf{Hamming wt}   & \textbf{Norm} \\\midrule
        FPTP voting     & $[0,1]^m$ & $\leq 1$& $\leq 1$ \\
        Approval voting     & $[0,1]^m$ & $\leq m$& $\leq m$ \\
        Range voting  & $[0,w]^m$ & $\leq m$      & $\leq wm$ \\
        Cumulative voting  & $[0,w]^m$ & $\leq m$      & $\leq w$ \\
        Ranked-choice voting (Borda) 
                        & $\pi([0,m-1])$& $m-1$     & $m(m-1)/2$ \\
        Quadratic voting
        & $[0,\sqrt{w}]^m$& $\leq m$& $\lVert \vec{b} \rVert_2^2 = \langle \vec{b}, \vec{b} \rangle = w$ \\
        \midrule
        Single-item sealed-bid auction & $[0, w]$  & 1 & $\leq w$ \\
        % \midrule
        % Bayesian truth serum %(\Cref{sec:voting_bayesian_truth})
        %                 & $[0,1]^m \times \mathbb{N}^m$ & $1, 1$ & $1, \leq m$ \\
        \bottomrule
    \end{tabular}
}
    \caption{Requirements for the domain, Hamming weight, and norm of a vector $\vec{b}$ for it to be a valid submission in various voting/auction schemes.
    $\pi(S)$ denotes the set of permutations of $S$. The norm is an $\ell_1$ norm unless otherwise specified. $m$ is the number of candidates, and $w$ is the maximum weight that can be assigned to any candidate.
    }
    \label{tab:voting_schemes}
\end{table*}

\paragraph{Majority, approval, range, and cumulative voting.} 
In the classic first-past-the-post (FPTP) voting scheme, voters can cast a vote of $1$ (support) for one candidate and $0$ for all others. A slight generalization of FPTP is approval voting, where users can assign a $1$ vote to multiple candidates, i.e., the cast ballot $s$ can be seen as $s\in\{0,1\}^{m}$, where $m$ is the number of causes. A further generalization is range voting, where users can give each candidate up to some weight $w$ (thus, approval is the special case where $w=1$). A related scheme is cumulative voting, where users can distribute a total of $w$ votes (points) among the candidates (now FPTP is a special case where $w=1$).
In each case, each candidate's points are tallied and the candidate with the highest number is declared the winner.

\paragraph{Ranked-choice voting.} 
% In a ranked-choice voting scheme, voters can signal more fine-grained preferences among $m$ candidates. In the Borda count version~\cite{emerson2013borda}, each voter can cast $m-1$ points to their first-choice candidate, $m-2$ points to their second-choice candidate, etc. In general, they can cast $m-k$ points to their $k$\textsuperscript{th} choice. 
% Several other counting functions exist for ranked voting, but in this work, we only focus on Borda counts. Our protocols can easily be adapted to other counting functions, such as the Dowdall system~\cite{fraenkel2014borda} via minor modifications.
In a ranked-choice voting scheme, voters can signal more fine-grained preferences among $m$ candidates by listing them in order of preference. There are multiple approaches to determining the winner, including single transferrable vote (STV) and instant runoff voting (IRV). In this work, we focus on the simpler Borda count version~\cite{Emerson13}, where each voter can cast $m-1$ points to their first-choice candidate, $m-2$ points to their second-choice candidate, etc., and the candidate with the most points is the winner. Our protocols can be adapted to similar counting functions, such as the Dowdall system~\cite{FraGro14}, via minor modifications.

\paragraph{Quadratic voting.} 
In quadratic voting~\cite{LalWey18}, each user's ballot is a vector $\vec{b} = (b_1, \dots, b_m)$ such that $\langle \vec{b}, \vec{b} \rangle = \lVert \vec{b} \rVert^2_2 \leq w$. Once again, the winner is determined by summing all the ballots and determining the candidate with the most points. Thus, this is also an additive voting scheme. 
However, proving ballot well-formedness efficiently in this particular case benefits greatly from the novel application of the residue numeral system (RNS) to private voting (see~\Cref{sec:packing}).

\paragraph{Single-item sealed-bid auction.} 
In a sealed-bid auction for a single item (e.g., an NFT or domain name), users submit secret bids to the auction contract. The domain of the bids might be constrained, e.g., $b\in\{0,1\}^{k}$ (in our implementations $k\approx 8-16$; see \Cref{sec:cicada_eval}). Therefore, bidders must prove that their bid is well-formed, i.e., falls into that domain. Once all secret bids are revealed, the contract selects the highest bidder and awards them the auctioned item. The price the winner must pay depends on the auction scheme: e.g., highest bid in a first price auction, second-highest in a Vickrey auction. %\todo{should we include this too?}\istvan{Iterated on this. Opinions?}

\subsection{Time-Lock Puzzles}\label{sec:tlp}

A time-lock puzzle (TLP)~\cite{RivShaWag96} consists of three efficient algorithms $\mathsf{TLP} = (\Setup, \mathsf{Gen}, \mathsf{Solve})$ allowing a party to ``encrypt'' a message to the future. To recover the solution, one needs to perform a computation that is believed to be inherently sequential, with a parameterizable number of steps.

\begin{definition}[Time-lock puzzle~\cite{RivShaWag96}] A time-lock puzzle scheme $\sf TLP = (\Setup, \mathsf{Gen}, \mathsf{Solve})$ for solution space $\mathcal{X}$ consists of the following three efficient algorithms:
    \begin{itemize}
        \item \underline{$\tlp.\Setup(\secparam, \timeT) \randout \pparam$:} The (potentially trusted) setup algorithm takes as input a security parameter $\secparam$ and a difficulty (time) parameter $\timeT$, and outputs public parameters $\pparam$. % (usually a group $\mathbb{G}$ with $\lambda$ bits of security). 
        % Typically $\mathbb{G}$ is a group of unknown order, e.g., the group $\mathbb{Z}^{*}_N$. 
        \item \underline{$\mathsf{TLP.Gen}(\pparam, s) \randout Z$:} Given a solution $s \in \X$, the puzzle generation algorithm efficiently computes a time-lock puzzle $Z$.
        \item \underline{$\mathsf{TLP.Solve}(\pparam, Z) \rightarrow s$:} Given a TLP $Z$, the puzzle-solving algorithm requires at least $\timeT$ sequential steps to output the solution $s$.
    \end{itemize}
\end{definition}

Informally, we say that a TLP scheme is \emph{correct} if $\mathsf{TLP.Gen}$ is efficiently computable, and $\tlp.{\sf Solve}$ always recovers the original solution $s$ to a validly constructed puzzle. A TLP scheme is \emph{secure} if $Z$ hides the solution $s$ and no adversary can compute $\mathsf{TLP.Solve}$ in fewer than $\timeT$ steps with non-negligible probability. For the formal definitions, see~\cite{C:MalThy19}.

\paragraph{Homomorphic TLPs.}
Malavolta and Thyagarajan~\cite{C:MalThy19} introduce \emph{homomorphic} TLPs (HTLPs). An HTLP is defined with respect to a circuit class $\mathcal{C}$ and has an additional algorithm $\sf Eval$ defined as:
\begin{itemize}
    \item \underline{$\htlp.{\sf Eval}(\pparam, C, Z_1, \dots, Z_m) \rightarrow Z_*$:} Given the public parameters, a circuit $C \in \mathcal{C}$ where $C: \X^m \rightarrow \X$, and input puzzles $Z_1, \dots, Z_m$, the homomorphic evaluation algorithm outputs a puzzle $Z_*$.
\end{itemize}

Correctness requires that %$\htlp.{\sf Solve}(Z_*)$
the puzzle obtained by homomorphically applying the circuit $C$ to $m$ puzzles should contain the expected solution, namely $C(s_1, \dots, s_m)$, where $s_i \gets \htlp.{\sf Solve}(Z_i)$. Again, we refer the reader to \cite{C:MalThy19} for the formal definition. Moving forward, we will use $\boxplus$ for homomorphic addition and $\cdot$ for scalar multiplication of HTLPs. For the homomorphic application of a linear function $f$, we write $f(Z_1, \dots, Z_m)$.

% Later, we may apply $\sf Eval$ to \emph{vectors} of puzzles of common size $k$ to indicate element-wise application of the function, e.g., $\{Z_1'', \dots, Z_k''\} \gets \htlp.{\sf Eval}(\pparam, f, \{Z_1, \dots, Z_k\}, \{Z_1', \dots, Z_k'\})$ means $Z_i'' \gets \htlp.{\sf Eval}(\pparam, C, Z_i, Z_i')$ for all $i \in [k]$.

% \subsection{HTLP Constructions}\label{app:htlp_constructions}

Malavolta and Thyagarajan~\cite{C:MalThy19} give two HTLP constructions with linear and multiplicative homomorphisms, respectively. They require $N$ to be a \emph{strong} semiprime, i.e., $N = p \cdot q$ such that $p = 2p' + 1$ and $q = 2q' + 1$ where $p', q'$ are also prime. The linearly-homomorphic HTLP is based on Paillier encryption~\cite{EC:Paillier99}, while the multiplicative homomorphism is achieved by working over the subgroup $\JJ_N \subseteq \ZZ_N^*$ of elements with Jacobi symbol $+1$. 
We recall their constructions below.
% We recall their constructions in \Cref{fig:mt19-HTLP}.

% \begin{figure}[tb]
%     \centering
%     \begin{mdframed}
    \begin{construction}[Linear HTLP~\cite{C:MalThy19}]\label{con:paillierHTLP}
    \hfill
    \begin{itemize}
        \item \underline{$\htlp.\Setup(\secparam, \timeT) \randout \pparam$:} Sample a strong semiprime $N$ and a generator $g \sample \ZZ_N^*$, then compute $h = g^{2^\timeT} \mod{N} \in \ZZ_N^*$. (This can be computed efficiently using the factorization of $N$). Output $\pparam := (N, g, h)$.
        \item \underline{$\htlp.{\sf Gen}(\pparam, s; r) \rightarrow Z$:} Given a value $s \in \ZZ_N$, use randomness $r \in \ZZ_{N^2}$ to compute and output
            $$Z := (g^r \mod{N},\ h^{r \cdot N} \cdot (1+N)^s \mod{N^2}) \in \JJ_N \times \ZZ_{N^2}^*$$
        \item \underline{$\htlp.\Open(\pparam, Z, r) \rightarrow s$:} Parse $Z := (u,v)$ and compute $w := u^{2^T} \mod{N} \allowbreak= h^r$ via repeated squaring. Output $s := \frac{(v/w^N \mod{N^2})- 1}{N}$.
        \item \underline{$\htlp.{\sf Eval}(\pparam, f, Z_1, Z_2) \rightarrow Z$:} To evaluate a linear function $f(x_1, x_2) = b + a_1 x_1 + a_2 x_2$ homomorphically on puzzles $Z_1 := (u_1, v_1)$ and $Z_2 := (u_2, v_2)$, return
        $$Z = (u_1^{a_1} \cdot u_2^{a_2} \mod{N}, v_1^{a_1} \cdot v_2^{a_2} \cdot (1+N)^b \mod{N^2}).$$
    \end{itemize}
    \end{construction}

\noindent Correctness holds because for all $s \in \ZZ_N$ and $Z = (u,v) \gets \htlp.{\sf Gen}(\pparam, s)$,
\begin{equation}\label{eq:correctness1}
\htlp.\Open(\pparam, Z) = \frac{(v/(h^R)^N \mod{N^2})-1}{N} = \frac{((1+N)^s) - 1}{N} = s
\end{equation}
since $(1+N)^x = 1+Nx \mod{N^2}$.
Correctness of the homomorphism follows since for all linear functions $f(x_1, x_2) = b + a_1 x_1 + a_2 x_2$ and all $Z_i = (u_i, v_i) \in {\sf Im}(\htlp.{\sf Gen}(\pparam, s_i; r_i))$ for $i \in \{1,2\}$,\footnote{For space and clarity we drop the moduli and assume that we are working in the appropriate ring in each coordinate (namely $\mathbb{Z}_N$ and $\mathbb{Z}_{N^2}$, respectively).}
% \begin{equation}\label{eq:correctness2}
% \begin{aligned}
\begin{align*}
&\htlp\rlap{$.{\sf Eval}(\pparam, f, Z_1, Z_2) = (u_1^{a_1} \cdot u_2^{a_2}, (1+N)^b \cdot v_1^{a_1} \cdot v_2^{a_2})$} && \\
&= (g^{r_1 a_1} \cdot g^{r_2 a_2}, &&(1+N)^b \cdot h^{r_1 N a_1} \cdot (1+N)^{s_1 a_1} \cdot h^{r_2 N a_2} \cdot (1+N)^{s_2 a_2})\\
&= (g^{r_1 a_1 + r_2 a_2},         &&h^{(r_1 a_1 + r_2 a_2) \cdot N} \cdot (1+N)^{b + s_1 a_1 + s_2 a_2}) \\
&\rlap{$= \htlp.{\sf Gen}(\pparam, f(s_1, s_2); r_1 a_1 + r_2 a_2)$} &&
\end{align*}
% \end{aligned}
% \end{equation}
which opens to $f(s_1, s_2)$ by \Cref{eq:correctness1}.


    \begin{construction}[Multiplicative HTLP~\cite{C:MalThy19}]\label{con:multHTLP}
    \hfill
    \begin{itemize}
        \item \underline{$\htlp.\Setup(\secparam, \timeT) \randout \pparam$:} Same as \Cref{con:paillierHTLP}.
        \item \underline{$\htlp.{\sf Gen}(\pparam, s; r) \rightarrow Z$:} Given a value $s \in \JJ_N$, use randomness $r \in \ZZ_{N^2}$ to compute and output
            $$Z := (g^r \mod{N},\ h^r \cdot s \mod{N}) \in \ZZ_N^* \times \ZZ_N^*$$
        \item \underline{$\htlp.\Open(\pparam, Z, r) \rightarrow s$:} Parse $Z := (u,v)$ and compute $w := u^{2^T} \mod{N} \allowbreak= h^r$ via repeated squaring. Output $s := v/w$.
        \item \underline{$\htlp.{\sf Eval}(\pparam, f, Z_1, Z_2) \rightarrow Z$:} To evaluate a multiplicative function $f(x_1,\allowbreak x_2) \allowbreak= a x_1 x_2$ homomorphically on puzzles $Z_1 := (u_1, v_1)$ and $Z_2 := (u_2,\allowbreak v_2)$, return
        $$Z = (u_1 \cdot u_2 \mod{N}, a \cdot v_1 \cdot v_2 \mod{N})$$
    \end{itemize}
    \end{construction}
%     \end{mdframed}
%     \caption{The HTLP constructions of \cite{C:MalThy19}.}
%     \label{fig:mt19-HTLP}
% \end{figure}

\noindent \Cref{con:multHTLP} operates over the solution space $\JJ_N$ (instead of $\ZZ_N$).
It is easy to see that $\htlp.\Open(\pparam,\allowbreak \htlp.{\sf Gen}(\pparam, s)) = s$ for all $s \in \ZZ_N^*$. Furthermore, for all $f(x_1, x_2) = a x_1 x_2$ and all $Z_i = (u_i, v_i) \in {\sf Im}(\htlp.{\sf Gen}(\pparam,\allowbreak s_i; r_i))$ for $i \in \{1,2\}$,
\begin{align*}
    \htlp&\rlap{$.{\sf Eval}(\pparam, f, Z_1, Z_2) = (u_1 \cdot u_2 \mod{N}, a \cdot v_1 \cdot v_2 \mod{N})$} &&\\
    &= (g^{r_1} g^{r_2} \mod{N}, &&h^{r_1} h^{r_2} \cdot a s_1 s_2 \mod{N})\\
    &= (g^{r_1 + r_2} \mod{N},   &&h^{r_1 + r_2} \cdot a s_1 s_2 \mod{N})\\
    &\rlap{$= \htlp.{\sf Gen}(\pparam, f(s_1, s_2); r_1 + r_2)$}. &&
\end{align*}

As an alternative, we introduce a novel linear HTLP based on the exponential ElGamal cryptosystem~\cite{EC:CraGenSch97} over a group of unknown order (\Cref{con:exp_elgamalHTLP}). This can be seen as a lifting of the multiplicative HTLP to put $s$ in the exponent, with changes shown in \chblue{blue}.
The construction requires a small solution space $\X \subset \ZZ_N$, i.e., $\X = \{ s : s \in \JJ_N \land\ s \ll N \}$. 

\begin{construction}[Efficient linear HTLP.]\label{con:exp_elgamalHTLP}
    \hfill
    \begin{description}
        \item[$\htlp.\Setup(\secparam, \timeT) \randout \pp$.] Output $\pp := (N, g, h, y)$, where $y \sample \ZZ_N^*$ and the remaining parameters are the same as in \cref{con:paillierHTLP,con:multHTLP}.
        \item[$\htlp.{\sf Gen}(\pp, s; r) \rightarrow Z$.] Given a value $s \in \chblue{\X \subset \ZZ_N}$, use randomness $r \in \ZZ_{N}$ to compute and output
            $$Z := (g^r \mod{N},\ h^r \cdot \chblue{y^s} \mod{N}) \in \ZZ_N^* \times \ZZ_N^*$$
        \item[$\htlp.\Open(\pp, Z, r) \rightarrow s$.] Parse $Z := (u,v)$ and compute $w := u^{2^T} \mod{N} \allowbreak= h^r$ via repeated squaring. \chblue{Compute $S := v/w$ and brute force the discrete logarithm of $S$ w.r.t. $y$ to obtain $s$.}
        \item[$\htlp.{\sf Eval}(\pp, f, Z_1, Z_2) \rightarrow Z$.] To evaluate a \chblue{linear function $f(x_1, x_2) = b + a_1 x_1 + a_2 x_2$} homomorphically on puzzles $Z_1 := (u_1, v_1)$ and $Z_2 := (u_2, v_2)$, return
        \chblue{$$Z = (u_1^{a_1} \cdot u_2^{a_2} \mod{N}, v_1^{a_1} \cdot v_2^{a_2} \cdot y^b \mod{N}).$$}
    \end{description}
\end{construction}

This scheme is only practical for small $\X$ since, in addition to recomputing $h^r$, recovering $s$ requires brute-forcing the discrete modulus of $y^s$. We discuss the efficiency trade-off between these two constructions in \Cref{sec:htlp-choice}.

\paragraph{Non-malleability.}
Introducing a homomorphism raises the issue of puzzle malleability, i.e., ``mauling'' one puzzle (whose solution may be unknown) into a puzzle with a related solution. This could lead to issues when HTLPs are deployed in larger systems, prompting the definition of non-malleability for TLPs~\cite{TCC:FKPS21}. We instead define and enforce non-malleability at the system level (see \Cref{sec:syntax}).

\subsection{Vector Packing}\label{sec:packing}

In many voting schemes, a ballot consists of a vector indicating the voter's relative preferences or point allocations for all $m$ candidates. To avoid solving many HTLPs, it is desirable to encode this vector into a single HTLP, which requires representing the vector as a single integer. This motivates the following definition.

\begin{definition}[Packing scheme]\label{def:packing}
A setup algorithm $\PSetup$ and pair of efficiently computable bijective functions $(\pack,\unpack)$ is called a \emph{packing scheme} and has the following syntax:
    \begin{itemize}
        \item \underline{$\PSetup(\ell, w) \to \pparam$:} Given a vector dimension $\ell$ and maximum entry $w$, output public parameters $\pparam$.
        \item \underline{$\pack(\pparam, \Vec{a}) \to s$:} Encode $\Vec{a} \in (\ZZ^+)^\ell$ as a positive integer $s \in \ZZ^+$. 
        \item \underline{$\unpack(\pparam, s) \to \Vec{a}$:} Given $s \in \ZZ^+$, recover a vector $\Vec{a} \in (\ZZ^+)^\ell$. 
    \end{itemize}
For \emph{correctness} we require $\unpack(\pack(\Vec{a}))=\Vec{a}$ for all $\Vec{a}\in (\ZZ^+)^\ell$.
\end{definition}

The classic approach to packing~\cite{ACNS:Groth05,EC:HirSak00} uses a \emph{positional numeral system (PNS)} to encode a vector of entries bounded by $w$ as a single integer in base $M := w$ (see \Cref{con:packingPNS} below).
%More specifically, the vector $\vec{a}=(a_1, \dots, a_m)$ with $\forall j \in [m]:a_j < w$ is encoded as a sum of powers of $M$: the ballot contains a single integer $s := \sum_{j=1}^m a_j M^{j-1}$. Then $a_j$ can be obtained as $s \mod{M^{j-1}}$, 
Instead, we will set $M:= nw+1$ to accommodate the homomorphic addition of all $n$ users' vectors. Each voter submits a length-$m$ vector with entries $\leq w$; summing over $n$ voters, the result is a length-$m$ vector with a maximum entry value $nw$. To prevent overflow, we set $M = nw+1$.

\begin{construction}[Packing from Positional Numeral System]\label{con:packingPNS}
\hfill
\begin{itemize}%[topsep=2pt]
    \item \underline{$\PSetup(\ell, w) \to M$:} Return $M := w + 1$.
    \item \underline{$\pack(M, \Vec{a}) \to s$:} Output $s := \sum_{j=1}^{\sizeof{\vec{a}}} a_j M^{j-1}$.
    \item \underline{$\unpack(M, s) \to \Vec{a}$:} Let $\ell := \ceil{\log_M{s}}$. For $j \in [\ell]$, compute the $j$th entry of $\Vec{a}$ as $a_j := s \mod{M^{j-1}}$.
\end{itemize}
\end{construction}

We also introduce an alternative approach in \Cref{con:packingRNS} which is based on the \emph{residue numeral system (RNS)}. The idea of the RNS packing is to interpret the entries of $\vec{a}$ as prime residues of a single unique integer $s$, which can be found efficiently using the Chinese Remainder Theorem (CRT). In other words, for all $j \in [\ell]$, $s$ captures $a_j$ as $s \bmod p_j$.

\begin{construction}[Packing from Residue Numeral System]\label{con:packingRNS}
\hfill
\begin{itemize}%[topsep=2pt]
    % \textit{Public parameters:}~$\pparam:=(p_1,\dots,p_m)$ primes s.t. $\min\limits_{j\in[m]}p_j\geq M(=nw+1)$.\\
    \item \underline{$\PSetup(\ell, w) \to \vec{p}$:} Let $M := w + 1$ and sample $\ell$ distinct primes $p_1, \dots, p_\ell\allowbreak \suchthat p_j \geq M\ \forall j \in [\ell]$. Return $\vec{p} := (p_1, \dots, p_\ell)$.
    \item \underline{$\pack(\vec{p}, \Vec{a}) \to s$:} Given $\Vec{a} \in (\ZZ^+)^\ell$, use the CRT to find the unique $s \in \ZZ^+$ s.t. $s\equiv a_j \pmod{p_j}~\forall j\in[\ell]$.
    \item \underline{$\unpack(\vec{p}, s) \to \Vec{a}$:} return $(a_1, \dots, a_\ell)$ where $a_j \equiv s \mod{p_j}\ \forall j \in [\ell]$.
\end{itemize}
\end{construction}

A major advantage of this approach is that, in contrast to PNS, which is only additively homomorphic for SIMD (single instruction, multiple data), the RNS encoding is fully SIMD homomorphic: the sum of vector encodings $\sum_{i \in [n]} s_i$ encodes the vector $\vec{a}_{+} = \sum_{i \in [n]} \vec{a}_i$, and the product $\prod_{i \in [n]} s_i$ encodes the vector $\vec{a}_{\times} = \prod_{i \in [n]} \vec{a}_i$. 
Note that as in the PNS approach, we set $M = nw + 1$ to accommodate homomorphic addition of submissions; homomorphic multiplication, however, would require $M=w^n+1$, and the primes in $\vec{p}$ would hence be larger as well.
Although the RNS has found application in error correction~\cite{TayCha14,KPTOC22}, side-channel resistance~\cite{TCHES:PFPB18}, and parallelization of arithmetic computations~\cite{AsiHosKon17,BajDuqMel06,GomTyaNam11,VNLVC20}, to our knowledge it has not been applied to voting schemes. We show in \Cref{sec:sigmas} that RNS is a natural fit for some voting schemes, e.g., quadratic voting, leading to more efficient proofs of ballot correctness. 