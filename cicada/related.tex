\section{Related Work}

\subsection{Voting and Auctions}
The cryptographic literature on voting schemes~\cite{PoPETS:HMMP23} and sealed-bid auctions \cite{ToSE:FraRei96,HarTygKik98,CCS:Cachin99,ACM_EC:NaoPinSum99,FC:BDJNPT06,FC:BCDGJK09,ACM_EC:PRST06} is enormous, dating to the 1990s. Most of these schemes are unsuitable to a fully decentralized and trust-minimized setting due to their inefficiency and/or reliance on trusted parties, e.g., tally authorities, servers running the public bulletin board, or auctioneers. Here, we review auction and voting protocols specifically designed to use a blockchain as the public bulletin board. 


\paragraph{Voting.} 
%Multi-authority schemes' trust assumptions are non-adequate in a decentralized setting~\cite{cramer1998practical}. 
The study of voting schemes for blockchain applications dates to at least 2017, when McCorry et al.~\cite{FC:McCShaHao17} proposed a ``boardroom'' voting protocol for DAO governance.
Their protocol's main disadvantage is that it can be aborted by a single party. Groth~\cite{ACNS:Groth05} and Boneh et al.~\cite{C:BBCGI23} develop techniques to create ballot correctness proofs for various voting schemes. These protocols all have proofs with size linear in the number of candidates. We break this barrier by applying polynomial commitments and assuming a transparent, lightweight pre-processing phase. Applying HTLPs to voting was suggested when they were proposed by Malavolta and Thyagarajan~\cite{C:MalThy19}. However, they left the details of making such a protocol practical, secure, and efficient to future work. We aim to fill this gap with our techniques for various election types and our EVM implementation.

\paragraph{Auctions.} 
Auctions are a natural fit for blockchains and were suggested as early as 2018~\cite{EPRINT:GalYou18}, albeit with a \emph{trusted auctioneer}. Bag et al.~\cite{TIFS:BHSR19} introduced SEAL, a privacy-preserving sealed-bid auction scheme without auctioneers. However, their protocol employs two rounds of communication since they apply the Hao-Zielinski Anonymous Veto network protocol~\cite{HaoZie06}. Tyagi et al.\ proposed Riggs~\cite{CCS:TAFWBM23}, a fair non-interactive auction scheme using timed commitments~\cite[\S6]{TCC:FKPS21}. This is perhaps the closest work to ours in implementing auctions (but not voting) in a fully decentralized setting using time-based cryptography. However, their design does not utilize homomorphism to combine puzzles, and as a result, Riggs is not scalable to a real-world setting with thousands of users. To achieve practicality, Riggs relies on an optimistic second round in which users voluntarily open their puzzles. %It is a challenging design in the pessimistic case when users do not open their commitments, i.e., one needs to force-open the commitments.%
Our approach is more practical as one only needs to solve a single (or handful of) TLPs, even in the worst case.
Chvojka et al. suggest a TLP-based protocol for both e-voting and auctions~\cite{ESORICS:CJSS21}, but their protocol has a per-election/auction trusted setup. We observe that HTLPs can help reduce (but not fully eliminate) the trust assumption by enabling a distributed per-protocol setup. We describe this idea, which may be of independent interest, in \Cref{sec:seq_mpc_tlp}.

\subsection{Time-based cryptography} 
Time-based cryptography, which uses inherently sequential functions to delay the revelation of information, also has a lengthy history dating to the introduction of time-lock encryption in 1996~\cite{RivShaWag96}.
Numerous variants have emerged since then, including timed commitments~\cite{C:BonNao00}, proofs-of-sequential-work~\cite{ITCS:MahMorVad13}, VDFs~\cite{C:BBBF18}, and HTLPs~\cite{C:MalThy19}, which we employ here.
For a recent survey, we refer the reader to Medley et al.~\cite{CSF:MedLoeQua23}. The only practical work we know of taking advantage of HTLPs is Bicorn~\cite{FC:CATB23}, which builds a distributed randomness beacon with a single aggregate HTLP for an arbitrary number of entropy contributors.

Delay encryption~\cite{EC:BurDeF21} constitutes the most recent development in time-based cryptography. It allows time-lock encryption of a message to an identity by combining identity-based encryption with inherently sequential computation. However, the only construction is based on isogenies and has enormous memory requirements ($\approx 12$\ TiB), making it impractical.

One can emulate time-based cryptography by applying stronger assumptions than the sequentiality of repeated (modular) squaring in groups of unknown order. McFly~\cite{FC:DHMW23} and tlock~\cite{EPRINT:GaiMelRom23} build time-lock encryption from threshold trust, i.e., by assuming that a subset of signers reliably releases a threshold BLS signature~\cite{AC:BonLynSha01} on the current timestamp. Even though this strong assumption is at odds with our setting, we view this as a promising alternate research direction for building efficient and fair on-chain protocols. In this approach, submissions would be encrypted to a timestamp and could only be decrypted with the knowledge of a BLS signature released at the particular timestamp. This still requires linear work and storage in the number of submissions, so enabling aggregation of multiple time-lock ciphertexts is necessary to make this approach suitable for voting and auction applications. Aggregation is not currently possible in the aforementioned schemes as they lack any homomorphism, and we leave this question to future work. 