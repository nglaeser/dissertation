\section{Related Work}

The cryptographic literature on both voting schemes and sealed-bid auctions is enormous, dating to the 1990s. However, most of these schemes are unsuitable for a fully decentralized and trust-minimized setting due to their inefficiency or reliance on trusted parties, i.e., tally authorities, servers running the public bulletin board, auctioneers, etc. Below, we review auction and voting protocols that use a blockchain as the public bulletin board. 

\paragraph{Voting.} 
%Multi-authority schemes' trust assumptions are non-adequate in a decentralized setting~\cite{cramer1998practical}. 
The study of voting schemes for blockchain applications dates to at least 2017, when McCorry et al.~\cite{FC:McCShaHao17} proposed a ``boardroom'' voting protocol for DAO governance.
The main disadvantage of their protocol is that the entire protocol can be aborted due to a single party. Groth~\cite{ACNS:Groth05} and Boneh et al.~\cite{C:BBCGI23} develop techniques to create ballot correctness proofs for various voting schemes. These protocols all have proofs with size linear in the number of candidates. We break this barrier by applying polynomial commitments and assuming a transparent, lightweight pre-processing phase. Applying HTLPs to voting was suggested when they were proposed by Malavolta and Thyagarajan~\cite{C:MalThy19}. However, they left the details of making such a protocol practical, secure, and efficient to future work. We aim to fill this gap with our techniques for various election types and our EVM implementation.

\paragraph{Auctions.} 
Auctions are a natural fit for blockchains and were suggested as early as 2018~\cite{EPRINT:GalYou18}, albeit with a \emph{trusted auctioneer}. Bag et al.\ introduced SEAL, a privacy-preserving sealed-bid auction scheme without auctioneers~\cite{TIFS:BHSR19}. However, their protocol employs two rounds of communication since they apply the Hao-Zielinski Anonymous Veto network protocol~\cite{HaoZie06}. Tyagi et al.\ proposed Riggs~\cite{CCS:TAFWBM23}, a fair non-interactive auction scheme using timed commitments~\cite[\S6]{TCC:FKPS21}. This is perhaps the closest work to ours in implementing auctions (but not voting) in a fully decentralized setting using time-based cryptography. However, their design does not utilize homomorphism to combine puzzles, and as a result the gas costs are high. To achieve practicality Riggs relies on an optimistic second round in which users voluntarily open their puzzles. %It is a challenging design in the pessimistic case when users do not open their commitments, i.e., one needs to force-open the commitments. We believe our approach is more practical as one only needs to solve a single (or handful of) TLPs, even in the worst case.
Chvojka et al. suggest a TLP-based protocol for both e-voting and auctions~\cite{ESORICS:CJSS21}. Their protocol has a per-auction trusted setup. In~\Cref{sec:seq_mpc_tlp}, we propose using HTLPs for distributed setup to reduce the trust assumption, which may be of independent interest.

\paragraph{Time-based cryptography.} 
Time-based cryptography, which uses inherently sequential functions to delay the revelation of information, also has a lengthy history dating to Rivest, Shamir, and Wagner's 1996 proposal of time-lock encryption~\cite{RivShaWag96}.
Numerous variants have emerged since then, including timed commitments~\cite{C:BonNao00}, proofs-of-sequential-work~\cite{ITCS:MahMorVad13}, VDFs~\cite{C:BBBF18}, and homomorphic time-lock puzzles~\cite{C:MalThy19}, which we employ here.
For a recent survey, we refer the reader to Medley et al.~\cite{CSF:MedLoeQua23}. The only practical work we know of taking advantage of HTLPs is Bicorn~\cite{FC:CATB23}, which builds a distributed randomness beacon with a single aggregate HTLP for an arbitrary number of entropy contributors.

Delay encryption constitutes the most recent development in time-based cryptography. Delay encryption allows time-lock encryption of a message to an identity by combining identity-based encryption with inherently sequential computation. This cryptographic primitive was introduced by Burdges and De Feo~\cite{EC:BurDeF21}. However, their isogeny-based construction has enormous memory requirements ($\approx 12$\ TiB), making their scheme impractical.

One can emulate time-based cryptography by applying stronger assumptions instead of assuming the sequentiality of repeated modular squaring. McFly~\cite{FC:DHMW23} and tlock~\cite{EPRINT:GaiMelRom23} build time-lock encryption from threshold trust, i.e., by assuming that a subset of signers intermittently and reliably releases a threshold signature on the current timestamp. Even though this is a stronger assumption, we view this as a promising alternative research direction. For instance, enabling the aggregation of multiple time-lock ciphertexts could make timelock encryption especially suitable for voting and auction applications; aggregation is not currently possible in the aforementioned schemes as they lack any homomorphism. 
Using any resulting homomorphic time-lock encryption schemes to build efficient voting and auction protocols is a further open problem.
We leave these questions to future work. 


% \paragraph{Number packing schemes.} The residue numeral system found application in error correction~\cite{rns_ecc,rns_ecc2}, side-channel resistance~\cite{papachristodoulou2019practical}, and parallelization of arithmetic computations~\cite{rns_neuralnet,bajard2006combining,asif2017high,rns_homomorphicenc}. Using HTLPs in voting gives rise to a natural new application since it simultaneously benefits from packing tuples of integers into a single integer and parallel arithmetic computation over these tuples. \joe{could kill this or move earlier} \noemi{already mention the last part in the packing section, could add the first sentence there too for context (since we kind of call it a ``new'' packing)} \noemi{done}