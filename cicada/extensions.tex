\section{Extensions}\label{sec:cicada_extensions}
We introduce extensions to the Cicada framework that may be useful in future applications.

\subsection{Everlasting ballot privacy for HTLP-based protocols}\label{sec:everlasting_ballot_privacy}

\begin{figure}
    \centering
    \begin{mdframed}
    \begin{center}
        \textbf{Off-chain batching}
    \end{center}
    \hfill\\
    \textbf{Public parameters:} A semiprime $N$ and $h, y \in \ZZ_N^*$, a voting scheme $\Sigma : \X^n \rightarrow \Y$. \hfill\\
    \newline
    Let $P_1, \dots, P_{k}$ be a group of $k < n$ parties with addresses ${\sf addr}_1, \dots, {\sf addr}_{k}$ wishing to batch their ballots $(u_i, v_i) := (g^{r_i}, h^{r_i} X_i)$.
    \begin{enumerate}
        \item Each party broadcasts $v_i$. Now, every party can compute $v := \prod_i v_i = h^R X$, which encodes the sum of their submissions.
        \item The parties use an $k-1$ malicious-secure MPC protocol~\cite{ESORICS:DKLPSS13,CCS:Keller20} on inputs $u_i$ to compute $u := \prod_i u_i = g^R$.
        \item They also compute two distributed-prover zero-knowledge proofs~\cite{PoPETS:DPPSV22} in the MPC: (i) a discrete logarithm equality proof $\pi_R$ that $\mathsf{dlog}_g(u) = \mathsf{dlog}_h(v)$ with distributed witness $R$, and (ii) a submission correctness proof $\pi_s$ that the aggregated solution $s$ encoded in $X$ is consistent with the sum of $k$ valid submissions, i.e., $s\in k \cdot \mathcal{X}$. Let $\pi_{\sf batch} = (\pi_R, \pi_s)$.
        \item Finally, each party signs the final aggregated submission $Z_{\sf batch} = (u,v)$.
    \end{enumerate}
    \textbf{Output}:~$(Z_{\sf batch},\pi_{\sf batch}, \{{\sf addr}_1, \dots, {\sf addr}_{k}\}, \{\sigma_1, \dots, \sigma_{k}\})$.
    
    \end{mdframed}
    \begin{mdframed}
    \begin{center}
        \textbf{On-chain batched ballot submission}
    \end{center}
    \hfill\\
    % batching only the first component
    % \textbf{Public parameters:} Cicada public parameters $\pp$. \hfill\\
    % \begin{enumerate}
    %     \item Parties $P_1, \dots, P_n$ wishing to batch their ballots submit $(g^0, h^{r_i} X_i)$ and a proof $\pi_i$ of well-formedness of the second component to the tallying contract (except leader $P_1$, who does nothing).
    %     \item $P_1, \dots, P_n$ execute the off-chain batching protocol (below), which outputs $g^R = \prod_{i \in [n]} g^{r_i}$ and $\pi_{batch}$ to $P_1$.
    %     \item $P_1$ submits $(g^R, h^{r_1} X_1)$ and proofs $\pi_1, \pi_{batch}$ by time $T-\tau$.
    %     \item If $P_1$ fails to submit by time $\tau$ or submits an inconsistent/invalid HTLP, the remaining parties re-run the off-chain batching protocol to compute $g^{R'} = \prod_{i \in [n], i \neq 1} g^{r_i}$ and a new proof $\pi_{batch}'$. Any (single) party $i^* \in [n]\setminus \{1\}$ can now override its submission with a new HTLP $(g^{R'}, h^{r_{i^*}} X_{i^*})$ and $\pi_{batch}'$. The second component should be the same as in the previous submission, and cheating can be punished by slashing.
    % \end{enumerate}
    \textbf{Public parameters:} Cicada public parameters $\pp$. \hfill\\
    \begin{enumerate}
        \item The designated party $P_1$ submits $Z_{\sf batch}, \pi_{\sf batch}, \{{\sf addr}_1, \dots, {\sf addr}_{k}\},\allowbreak \{\sigma_1, \dots, \sigma_{k}\}$ to the tallying contract, which verifies the proofs and signatures, and adds $(u,v)$ to the tally HTLP $Z$ as in the basic protocol.
        \item If $P_1$ doesn't submit by time $T-\tau$, any other party in the batch group can submit instead.
    \end{enumerate}
    \end{mdframed}
    
    \caption{The on- and off-chain ballot batching protocols that $k < n$ parties can use to achieve everlasting ballot privacy.}
    \label{fig:batching_ballots}
\end{figure}

\noemi{Note this introduces interactivity but off-chain}

The basic Cicada framework does not guarantee long-term ballot privacy. Submissions are public after the $\Open$ stage. This is because users publish their HTLPs on-chain: once public, the votes contained in the HTLPs are only guaranteed to be hidden for the time it takes to compute $\Ttime$ sequential steps, after which point it is plausible that someone has computed the solution. In many applications, it is desirable that individual ballots remain hidden \emph{even after voting has ended} since the lack of everlasting privacy may facilitate coercion and vote-buying. As mentioned in \Cref{sec:syntax}, this can be achieved modularly by first decoupling the ballots from their voters via a privacy-enhancing overlay. Alternatively, we describe how the $\Seal$ procedure can be modified to prevent the opening of individual ballots, achieving everlasting privacy.

Observe that all known \emph{efficient HTLP constructions} are of the form $(u,v) = (g^r, {h'}^r X)$,\footnote{In the exponential ElGamal case, $h' = h$, while in the Paillier construction, $h' = h^N$ (see \Cref{app:htlp_constructions}). We will drop the tickmark on $h'$ in the remainder of this section to avoid notational clutter.} where the solution is encoded in $X$ and recovering it requires recomputing $h^r=(g^{r})^{2^{\Ttime}}$ via repeated squaring of the first component. Our insight is that the puzzle information-theoretically hides the solution $X$ without the first component. Importantly, publishing $g^r$ is not necessary in any of our HTLP-based voting protocols \emph{except as a means to verifiably compute the first component of the final HTLP}, i.e., $g^{R}=g^{\sum_{i\in[n]} r_i}$. The observation that $g^R$ can be computed \emph{without} revealing the individual values $g^{r_i}$ enables us to construct the first practical and private voting protocols that guarantee \emph{everlasting} ballot privacy with a single on-chain round.

% \todo{change text starting around here...} \noemi{Note you can use aggregatable sigs for efficiency}
For simplicity, consider a protocol in which both the ballot of user $i$ and the tally consists of a single HTLP, respectively $Z_i=(g^{r_i},h^{r_i}X_i)$ and $Z = (g^R, h^R X)$. Observe that for everlasting ballot privacy, updates to $Z$ must inherently be batched: a singleton update $\Aggr(\pparam, Z, Z_i, \pi) \to (g^{R+r_i}, h^{R+r_i} Y)$ (for some $Y$) would reveal $g^{r_i} = g^{R+r_i}/g^R$, which is the opening information to $Z_i$, as the quotient of the first component of $Z$ after and before the update. Hence, the ballot $X_i$ of user $i$ would be recoverable in $\Ttime$ sequential steps, i.e., after computing $h^{r_i}=(g^{r_i})^{2^{\Ttime}}$.

Batching ballot submissions off-chain in groups of $k$ allows parties to achieve everlasting privacy as long as at least one party is honest. 
The parties aggregate their submissions off-chain as $(g^R, h^R X) = (\prod_i g^{r_i}, \prod_i h^{r_i} X_i)$ and compute a proof $\pi_{\sf batch}$ of well-formedness in a distributed-prover zero-knowledge proof protocol~\cite{PoPETS:DPPSV22}. We use the observation that the individual second components $v_i$ are hiding to optimize the batching by computing $h^R X$ in the clear
% Intuitively, the parties compute their aggregate randomness $g^R = \prod_i g^{r_i}$ and a proof $\pi_{\sf batch}$ of its well-formedness in a distributed-prover zero-knowledge proof protocol~\cite{dayama2021prove}. Each party submits only the second component of its ballot to the contract, except a designated party (e.g., $P_1$) who also submits $g^R$. The updated tally HTLP $Z^*$ is computed as 
% $\Aggr(\pparam, Z, \{(g^R, h^{r_1} X_1),\allowbreak (g^0, h^{r_2} X_2), \dots,\allowbreak (g^0, h^{r_k} X_k) \}, \pi_{\sf batch}),$
% where $\pi_{\sf batch}$ additionally proves to the contract that ${\sf dlog}_g(g^R) = {\sf dlog}_h(\prod_i h^{r_i} X_i)$, 
(see~\Cref{fig:batching_ballots} for details).

\begin{figure}[htb]
    \centering
    \begin{mdframed}
    \begin{center}
        \textbf{Borda ballot correctness}
    \end{center}
    \todo{incomplete}
    \begin{enumerate}
       \item $\mathcal{P}$ sends $(g^a\bmod{N},(1+N)^{a}\bmod{N^2})$ for $a\sample\mathbb{Z}_N$. 
       \item $\mathcal{V}$ sends a challenge $e\sample[1,n]$.  
       \item $\mathcal{P}$ sends $(g^{r\cdot e+a},h^{r\cdot N\cdot e+a}(1+N)^{s\cdot e+a}\bmod{N^2})$.
    $\mathcal{V}$ accepts iff the following hold:
    \end{enumerate}
    \end{mdframed}
    \label{fig:borda-pok}
\end{figure}


This idea opens up a new design space for the MPC protocol used for batching, such as doing the randomness generation in a preprocessing phase instead, allowing dynamic additions to the anonymity set, optimizing the batch proof generation, and dealing with parties who fail to submit. We leave the full exploration of this large design space and related questions to future work.

% As an alternative, the MPC could be run in a pre-processing stage which outputs a random $t$-of-$n$ sharing of a uniform $R$ and the proof $\pi_{\sf batch}$. Once party $i$ is ready to submit, it computes $Z_i = h^{r_i} X_i$ and sends it to the contract as above.

% The protocol is described in detail in \Cref{fig:batching_for_everlasting_privacy}.
% For $k$ parties holding ballots $Z_i = (g^{r_i}, h^{r_i} X_i), i \in [k]$, each party hides the first component of its ballot by computing $Z_i' = (g^{r_i}h^{u_i}, h^{r_i} X_i)^{k}_{i=1}$ for some $u_i\sample\mathbb{Z}_N$. Next, the parties run a secure multi-party computation (MPC) protocol (in the 2-party case, they just run this in the clear) to compute $g^{R}h^{U} = g^{\sum_i r_i}h^{\sum_i u_i}$. Then they open this commitment to $g^R$ by computing the opening information $h^U=\prod\limits^k_{i=1} h^{u_i}$ in MPC. $P_i$ will submit $(h^{r_i} X_i, \pi_i)$ to the contract, where $\pi_i$ is the original proof of puzzle well-formedness. Without loss of generality, let $P_1$ additionally send $g^R$. The updated tally $\htlp$ $Z^*$ is computed as $\Aggr(\pparam, Z^*, \{(g^R, h^{r_1} X_1), (g^0, h^{r_2} X_2), \dots, (g^0, h^{r_k} X_k) \}, \pi_{\sf batch})$, where $\pi_{\sf batch}$ additionally proves to the contract that ${\sf dlog}_g(g^R) = {\sf dlog}_h(\prod_i h^{r_i} X_i)$. % have the same discrete logarithms with respect to the bases $g$ and $h$, respectively.
    
% \todo{(before bids/ballots are fixed) and run an MPC to produce a random $t$-of-$n$ sharing (output $r_i$ to $P_i$ along with a proof $\pi_{r_i}$). As long as there are $n_0 > t$ parties to start, additional parties could join dynamically. Once a party is ready to submit, it computes $Z_i = (g^{r_i}, h^{r_i} X_i)$ and sends it to the contract as usual along with $\pi_{r_i}$, see Figure~\ref{fig:batching_for_everlasting_privacy}. Ballots must be accumulated by homomorphically reconstructing $r$, so this may rule out some class of functions over the $X_i$'s. This version is more resilient to dropouts since it can tolerate up to $n-t$ users not submitting their puzzles. The drawback is that the submissions are larger.}

% \begin{figure}[htb]
% \begin{mdframed}
% \begin{center}
%     \textbf{Privacy-preserving batching of ballots} 
% \end{center}
% \textbf{Public parameters:} Group of unknown order $\GG$ with  $g,h\sample\GG$, batch size $k$.\hfill\\
% \textbf{Public input:} The second component $h^{r_i} X_i$ of each voter's ballot, $i \in [k]$.\hfill\\
% \textbf{Private input:} The first component $g^{r_i}$ of each voter's ballot, $i \in [k]$.
% \begin{enumerate}
%     \item Each voter $i$ publishes $g^{r_i}h^{u_i}$ for $u_i \sample \ZZ$ along with a proof $\pi_i' \gets \Pi_{\sf PoEqDLog}.{\sf Prove}(g,g^{r_i}h^{u_i},h,h^{r_i} X_i; r_i)$.
%     \item Every party checks every other party's proof $\pi_i'$ and reconstructs $g^{R}h^{U}:=\prod^{k}_{i=1} g^{r_i}h^{u_i}$. (If any proof fails to verify, that party is excluded from the batch. \todo{check this doesn't intro predictable bias like Pedersen DKG})
%     \item The parties participate in an MPC to compute $h^U=\prod^{k}_{i=1}h^{u_i}$ and obtain $g^R=g^{R}h^{U}/h^{U}$.
%     \item Each party uses $h^U$ to compute $g^R = (g^R h^U)/h^U$. The parties collaboratively compute $\pi_{batch} \gets \Pi_{\sf PoEqDlog}.{\sf Prove}(g,g^R,h,\prod h^{r_i}X_i; R)$.
%     \item Each voter computes its ballot correctness proof $\pi_i$ for $h^{r_i} X_i$.
% \end{enumerate}
% \textbf{Output:} $(h^{r_i} X_i,\pi_i)^{k}_{i=1},g^R,\pi_{batch}$.

% \end{mdframed}
% \caption{Off-chain batching protocol for everlasting ballot privacy.}
% \label{fig:batching_for_everlasting_privacy}
% \end{figure}

% For simplicity, consider a protocol in which each user submits a single HTLP $Z_{\sf userID} := (Z_{\sf userID}^{(0)}, Z_{\sf userID}^{(1)})$ to the contract. With this modification, the user would instead submit only the second component $Z_{\sf userID}^{(1)}$ along with some masked version of the first component. \noemi{unclear how best to avoid leaking $\Delta r$ when the before/after of the update is known in the clear (e.g., if the accumulated vote HTLP is currently $g^{r_t}$, and after your update it's $g^{r_{t+1}}$, your update is easily deducible as $g^{r_{t+1}}/g^{r_{t}}$)... solutions could be batching updates (parties incentivized to batch anyway to reduce fees) or distributing shares of 1 at some point and weighting contribution by share.}

% \section{Adding Coercion Resistance}\label{sec:coercion_resistance}
% In certain applications, it is desirable to provide coercion-resistance~\cite{juels2005coercion} for the voters.
% \istvan{todo: for each user the contract stores their latest ballot. They can add the ``negative'' of their previous ballot to the tally. And now the user just needs to submit their new ballot which is subsequently added to the tally. The cost of this approach is that now all the votes need to be stored in the storage which might add $\approx 200,000$ gas more per ballot for the first submission.}
% Another potential solution might be to add the ballot randomizers to an accumulator/zero-knowledge sets.

% \todo{there is an impossibility result (\url{https://hackingdistributed.com/2018/07/02/on-chain-vote-buying/}) that coercion-resistance in the permissionless setting requires trusted hardware (trusted setup)}

\subsection{Succinct ballot-correctness proofs}\label{sec:succinct_ballot_correctness}

Real-world elections often have hundreds of candidates, e.g., Optimism's retroactive public good funding~\cite{retropgf_voterguide}. However, the state-of-the-art ballot correctness proofs~\cite{C:BBCGI23,ACNS:Groth05} for all voting schemes (e.g., FPTP, approval voting, etc.) are linear in the number of candidates, rendering these schemes impractical in the blockchain setting. To counter these issues, we design constant-size ballot correctness proofs with constant verification time at the expense of an added preprocessing phase. The high-level idea is as follows. All correct ballots (e.g., $\{ \pack(s) : s\in\{0,1\}^{m}\}$ in the case of approval voting) are inserted into an accumulator or polynomial commitment (PC)~\cite{AC:KatZavGol10} during a transparent preprocessing phase. When users submit their votes $Z \sample \htlp.\mathsf{Gen}(s)$, they prove in zero-knowledge that $Z$ encodes a correct ballot, i.e., the users show that the solution $s$ of $Z$ had been previously inserted into the accumulator or PC with a succinct (blinded) membership proof~\cite{CCS:ZBKMNS22}.
% We detail our succinct ballot-correctness proof using the KZG commitment in \Cref{app:succinct}.

In this section, we assume that a common reference string for the KZG polynomial commitment (PC) scheme~\cite{AC:KatZavGol10} is already available to users, namely $\mathsf{crs}:=\{g_1^{\tau^{j}}\}^{d}_{j=1}$, where $g_1\in\mathbb{G}_1$ is a generator in a bilinear pairing-friendly cyclic group $\mathbb{G}_1$ over $\mathbb{F}_p$ for some prime $p$, $\tau\sample\mathbb{F}_p$ hidden to everyone. The $\mathsf{crs}$ is typically established during a sequential, secure multi-party computation (MPC), e.g.,~\cite{FCW:BowGabGre18}.

%-----------------------------------------------------------BEGIN Preprocessing Ballots --------------------------------------------------------------
\begin{figure}
    \centering
    \begin{mdframed}
    \begin{center}
        \textbf{Preprocessing ballots for succinct ballot-correctness proofs}
    \end{center}
    \hfill\\
    \textbf{Public parameters:} The common reference string $\mathsf{crs}:=\{g_1^{\tau^{j}}\}^{d}_{j=1}\in\mathbb{G}_1^{d}$. A semiprime $N$ and $h, y \in \ZZ_N^*$. A voting scheme $\Sigma : \X^n \rightarrow \Y$, e.g., approval voting. \hfill\\
    \begin{enumerate}
        \item Let $\X$ be the set of correct ballots and $\vert\X\vert=d$
        \item Let $f(x)\in\mathbb{F}^{\leq d}_p[x]$ be a univariate polynomial s.t. $\forall s_i\in\X:f(i):=\pack(s_i)$. The polynomial $f(x)$ could be computed using Lagrangian interpolation.
        \item Let $\mathsf{com}$ be the KZG commitment to the polynomial $f$.
    \end{enumerate}
    \textbf{Output}:~$\mathsf{com}$.
    
    \end{mdframed}
    \caption{Preprocessing ballots to enable succinct ballot-correctness proofs.}
    \label{fig:preprocessing_ballots}
\end{figure}
%-----------------------------------------------------------END Preprocessing Ballots --------------------------------------------------------------

Let us assume that users have established during a preprocessing phase (\Cref{fig:preprocessing_ballots}) a short commitment $\mathsf{com}$ that encodes all the possible ballots in a particular voting scheme, e.g., $\X= [0,1]^{m}$ for approval voting. The size of classical proofs of well-formedness, e.g., OR-composition of sigma-protocols, scale linearly in the number of candidates $m$. The following proof strategy yields a constant-size proof of correctness for moderately-sized $\X$, i.e., $\sizeof{\X} \leq d$\footnote{The largest KZG CRS we know of~\cite{largekzg} is for $d = 2^{28}$, so in the case of $\X = [0,1]^m$ this strategy requires $m \leq 28$.}.
% $m$, say $m\leq 30$, depending on the size of the available  $\mathsf{crs}$ of the polynomial commitment scheme.

First, given a ballot $Z=(g^r,h^ry^s) \in \tilde{\GG}_1 \times \tilde{\GG}_2$, the user creates an elliptic curve point $Z_1=h^r_1y^s_1 \in \GG_1$ for random generators $h_1,y_1\sample\mathbb{G}_1$ in a pairing-friendly group. Using the discrete logarithm across different groups techniques developed in~\cite{EPRINT:COPZ22}, the user can show that $Z$ and $Z_1$ have the same discrete logarithms $r$ and $s$ with for their bases $h,y\in\mathbb{G},h_1,y_1\in\mathbb{G}_1$, respectively. Now that $Z_1$ and the polynomial commitment are in the same pairing-friendly group $\GG_1$, the user can create a blinded KZG opening proof~\cite{CCS:ZBKMNS22} to prove ballot correctness. Specifically, the proof $\pi$ shows that the value $s$ in $Z_1$ matches an evaluation of the polynomial $f$ committed by $\mathsf{com}$ at some (hidden) point $j$, i.e., $f(j)=s$. Note that the verifier only sees constant-size commitments of $f,j$, and $s$. Since the blinded KZG proof $\pi$ is also constant-size, this strategy yields the first succinct ballot-correctness proofs for many common voting schemes, e.g., approval and range voting.

\subsection{Coercion-Resistance}\label{sec:extension_coercion_resistance}
Lastly, we briefly outline how one could add coercion-resistance~\cite{WPES:JueCatJak05} to our framework. In the e-voting literature, there are two main pathways to obtaining coercion-resistance: receipt-freeness or allowing unlikable revotes. Receipt-freeness seems challenging to achieve in the blockchain context, and we leave it to future work.  Therefore, we follow the revoting paradigm akin to Lueks et al. ~\cite{USENIX:LueQueTro20}. One can allow indistinguishable revotes as follows. We could store our ballots in a zero-knowledge set (e.g., Semaphore is a readily available implementation of this concept for Ethereum~\cite{semaphore}). Additionally, we would need to store a Merkle tree of nullifiers on-chain containing the ballots that are revoked due to revoting. Whenever users want to revote, they could prove in zero-knowledge that they revoke a previous ballot that they inserted in the zero-knowledge set while they reveal the accompanying nullifier and insert it into the nullifier tree. We leave it to future work to flesh out the technical details and implementation of such an extension.

\subsection{Bayesian truth serum}\label{sec:bayesian_truth}

Bayesian truth serum~\cite{Prelec04} is a method for eliciting truthful subjective answers where objective truth does not exist or is not knowable. The core of the idea is to reward answers that are ``surprisingly common'' by leveraging respondents' own predictions of what will be common. Thus, for a question with many (mutually exclusive) potential answers, the score of user $i$ responding $\vec{x}_i := (x_{i1}, \dots, x_{im})$ and $\vec{y}_i := (y_{i1}, \dots, y_{im})$ is calculated as
%
\begin{equation}\label{eqn:bts}
    {\sf score}_i := \sum_{j \in [m]} x_{ij} \log{\frac{\overline{x}_j}{\overline{y}_j}} + \alpha \sum_{j \in [m]} \overline{x}_j \log{\frac{y_{ij}}{\overline{x}_j}}
\end{equation}
%
where $\alpha > 0$ is a constant. The variable $x_{ij} \in \{0,1\}$ denotes user $i$'s decision (choose or don't choose) for option $j \in [m]$, $\overline{x}_j$ is the empirical frequency of choice $j$ over all the users' answers, $y_{ij}$ is user $i$'s estimate of $\overline{x}_j$ (i.e., their estimate of the probability of answer $j$ among all users), and $\overline{y}_j$ is the empirical (geometric) average of $y_{ij}$ over all the users' answers. Since each user can only choose a single answer, $x_{ij}$ will be 0 for all but one value of $j$, which we denote $j^*$. Thus, we can think of the equation above as equivalent to
%
\[
    x_{ij^*} \log{\frac{\overline{x}_{j^*}}{\overline{y}_{j^*}}} + \alpha \sum_{j \in [m]} \overline{x}_j \log{\frac{y_{ij}}{\overline{x}_j}}.
\]

The first term is referred to as the \emph{information score} and the second as the \emph{prediction score}. The information score is highest when the user's choice $k^*$ is ``surprisingly common'', i.e., when the empirical frequency of answer $j^*$ ($\overline{x}_{j^*}$) is higher than the crowd's estimate of the empirical frequency of $j^*$ ($\overline{y}_{j^*}$). Therefore participants are incentivized to submit their truthful responses, even (and especially) if they believe them to be uncommon. 

The prediction score is the Kullback-Leibler divergence~\cite{KulLei51} between the user's estimate of the average answer and the true average answer, weighted by $\alpha$. This is maximized when the two values are equal (i.e., the divergence is 0), and so incentivizes truthful reporting of $y_{ij}$, the user's estimate of $\overline{x}_j$. 

We show how Bayesian truth serum can be implemented in the Cicada framework. First, rewrite \Cref{eqn:bts} as
%
\begin{equation}\label{eqn:bts-cicada}
    \begin{aligned}
        {\sf score}_i :=& \sum_{j \in [m]} x_{ij} (\log{\overline{x}_j}-\overline{y}_j') + \alpha \sum_{j \in [m]} \overline{x}_j (y_{ij}' - \log{\overline{x}_j})\\
        % =& \sum_{j \in [m]} x_{ij} (\log{\overline{x}_j}-\overline{y}_j') + \alpha \sum_{j \in [m]} (\overline{x}_j y_{ij}' - \overline{x}_j \log{\overline{x}_j})
    \end{aligned}
\end{equation}
%
where $y_{ij}' = \log{y_{ij}}$ and $\overline{y}_{j}' = \log{\overline{y}_j}$.
\def\tallyx{\ensuremath{\overline{\vec{x}}}}
\def\tallyy{\ensuremath{\overline{\vec{y}}'}}
The smart contract will use two (lists of) HTLPs $\Z_{\tallyx}^{\sf tally}, \Z_{\tallyy}^{\sf tally}$ to keep track of two running ``tallies'':
\begin{align*}
\tallyx &= (\overline{x}_1, \dots, \overline{x}_m) = \sum_i \vec{x}_i\\
\tallyy &= (\overline{y}_1', \dots, \overline{y}_m') = \sum_i \frac{1}{n} \vec{y}_i'
\end{align*}

Each user's ballot consists of the vectors $\vec{x}_i, \vec{y}_i'$, where $\vec{x}_i \in [0,1]^m$ has $\ell_1$ norm 1 and $\vec{y}_i' = \log{\vec{y}} \in \mathbb{N}^m$ with $\sum_{j \in [m]} y_{ij} = n$. Assuming no packing for simplicity, the ballot is encoded as three lists of HTLPs: a list of linear HTLPs $\Z_i^{\sf ans} := \{Z_{ij}^{\sf ans}\}_{j \in [m]}$ for the entries of $\vec{x}_i$, and two lists of (respectively) linear and multiplicative HTLPs $\Z_i^{+} := \{Z_{ij}^{+}\}_{j \in [m]}$ and $\Z_i^{\times} := \{Z_{ij}^{\times}\}_{j \in [m]}$, both encoding the entries of $\vec{y}_i'$. The smart contract coordinator must ensure that the following hold:
\begin{description}
    \item[Check \#1a.] All $Z_{ij}^{\sf ans}$ encode $x_{ij} \in [0,1]$.
    \item[Check \#1b.] $\sum_{j \in [m]} x_{ij} = 1$.
    \item[Check \#2a.] All $Z_{ij}^{+}$ encode $y_{ij}' > 0$.
    \item[Check \#2b.] $\sum_{j \in [m]} 2^{y_{ij}'} = n$ (assuming $\log$ base 2).
    \item[Check \#3.] $Z_{ij}^{\times}$ contains the same value as $Z_{ij}^{+}$ for all $j \in [m]$.
\end{description}

Most of these checks can be achieved using the protocols in \Cref{sec:sigmas}: Check \#1a with the binary solution protocol, \#1b and \#2b by providing randomness which opens the homomorphic sum to the correct value, and \#2a with \zkpopos. Check \#2b additionally requires a zero-knowledge proof of exponentiation, e.g., \cite{C:BonBunFis19}. Because the puzzles to check in \#3 use different constructions, we can't apply \zkposeq directly; instead, one can combine two \zkpoks proofs with a standard PoK for discrete logarithm.

The aggregation algorithm $\Aggr((\Z_{\tallyx}^{\sf tally}, \Z_{\tallyy}^{\sf tally}), i, \Z_i, \pi_i)$ updates the tally to $(\Z_{\tallyx}^{\sf tally} \boxplus \Z_i^{\sf ans}, \Z_{\tallyy}^{\sf tally} \boxplus \frac{1}{n} \cdot \Z_i^{+})$. During the opening phase, anyone can solve for the final tallies $\tallyx_{\sf final}, \tallyy_{\sf final}$ and the individual user submissions $\{(\vec{x}_i, \vec{y}_i')\}_{i \in [n]}$. If correct, they are used in $\Finalize$ to compute the final set of scores as follows:

\begin{enumerate}
    \item Let $\tallyx' := \log{\tallyx}$. Compute $\vec{I}' := \tallyx' - \tallyy$ and $\vec{P}' := \tallyx \cdot \tallyx'$.
    % \item Compute $P := \sum_j \overline{x}_j \log{\overline{x}_j}$.
    \item For each user $i \in [n]$:
    \begin{enumerate}
        \item Compute $i$'s information score $I_i := \sum_{j \in [m]} I_{ij}$, where $\vec{I}_i = (I_{i1},\allowbreak \dots, I_{im}) := \vec{x}_i \cdot \vec{I}'$.
        \item Compute $i$'s prediction score $P_i := \sum_{j \in [m]} P_{ij}$, where $\vec{P}_i = (P_{i1},\allowbreak \dots, P_{im}) := \tallyx \cdot \vec{y}_i' - \vec{P}'$.
        \item User $i$'s score is $I_i - P_i$.
        % \item Compute ${\Z_i^{\times}}' := \Z_i^{\times} \cdot \tallyx$.
        % \item $Z_i^{Y} := \bigotimes_{j \in [m]} {Z_{i,j}^{\times}}'$.
        % \item $Z_i^{I} := \bigoplus_{j \in [m]} (I_j \cdot Z_{i,j}^{{\sf ans}})$.
        % \item Set $Z_i^{\sf score} := Z_i^I \oplus \alpha (\log{Z_i^{Y}} - P)$. \noemi{this logarithm can't be computed over a puzzle, so actually $Z_Y^{(i)}$ has to be solved first before the score puzzle can be computed (contract could keep a list).}
    \end{enumerate}
\end{enumerate}

\subsection{A trusted setup protocol for the CJSS scheme}\label{sec:seq_mpc_tlp}

Chvojka, Jager, Slamanig, and Striecks~\cite{ESORICS:CJSS21} describe how to combine a public-key encryption scheme with a TLP to obtain a private voting or auction protocols which, unlike the HTLP-based approach suggested by~\cite{C:MalThy19}, is ``solve one, get many for free''. The high-level idea of the protocol is to encrypt each user's bid with a common public key whose corresponding secret key is inserted into a TLP (see~\Cref{fig:solve_one_protocol}). Therefore, none of the bids can be decrypted until the corresponding encryption secret key is obtained by solving the TLP. One drawback of this scheme, however, is that it requires an additional trusted setup procedure to create a TLP containing the secret key corresponding to the encryption public key used. Furthermore, unlike the HTLP approach, the setup cannot be reused and must be re-run for every protocol invocation.

\begin{figure}[ht]
\begin{mdframed}
\begin{center}
    \textbf{The CJSS Framework}
\end{center}
Let $\Pi_{\sf E}$ be a CCA-secure public-key encryption scheme, $\tlp$ a time-lock puzzle scheme, and $\Sigma: \X^n \rightarrow \Y$ a base voting/auction protocol.
\begin{description}
    \item[$\mathsf{Setup}(\secparam, \Ttime) \randout (\pparam, \Z)$.] Sample a key-pair $(\pk, \sk) \gets \Pi_{\sf E}.{\sf Gen}(\secparam)$ and TLP parameters $\pparam_{\sf tlp} \gets \tlp.\Setup(\secparam, \Ttime)$. Compute $Z_{\sk} \sample \tlp.\mathsf{Gen}(\pparam_{\sf tlp},\sk)$ and return $\pparam := (\pparam_{\sf tlp}, \pk)$ and $\Z := (Z_{\sk}, \perp)$.
    \item[$\Seal(\pparam, i, s) \randout (\ct_i, \pi_{i})$.] Parse $\pk$ from $\pparam$ and compute an encrypted bid/ballot as $\ct_i \gets \Pi_{\sf E}.\enc(\pk,s_i)$ along with a proof $\pi_{i}$ that $\ct_i$ is a valid encryption under $\pk$.
    \item[$\Aggr(\pparam, \Z, \ct_{i}, \pi_{i}) \rightarrow \Z'$.] Verify $\pi_{i}$. If the check passes, parse $\Z := (Z_{\sk}, \mathcal{C})$ and update to $\Z' := (Z_{\sk}, \mathcal{C} \cup \{ \ct_{i} \})$.
    \item[$\Open(\pparam,\Z,\perp) \rightarrow \sk$.] Let $\Z := (Z_{\sk}, \mathcal{C})$ and publish $\sk \leftarrow \htlp.\mathsf{Solve}(\pparam_{\sf tlp},Z_{\sk})$.
    \item[$\Finalize(\pparam,\sk) \rightarrow y$.] Use the secret key $\sk$ to decrypt each ciphertext $\ct_i \in \mathcal{C}$ to $s_i \gets \Pi_{\sf E}.\dec(\sk, \ct_i)$. Compute the final result in the clear as $\Sigma(s_1, \dots, s_n)$.
\end{description}
\end{mdframed}
\caption{The ``solve one, get many for free'' paradigm (CJSS)~\cite{ESORICS:CJSS21}.}
\label{fig:solve_one_protocol}
\end{figure}


% \noemi{Make a note about how the $\sf Open$ step can be optimized via a homomorphic encryption scheme, but that this introduces questions of non-malleability which we will discuss in XX.}

We observe that, for encryption schemes with discrete-log key-pairs such as Cramer-Shoup~\cite{C:CraSho98}, there is a natural decentralized setup protocol secure against all-but-one corruptions. Using the blockchain as a broadcast channel (similar to \cite{ACNS:NRBB24}), a simple sequential MPC protocol to set up the parameters works as follows. Suppose there is some smart contract that stores the public key $\pk=g^{\sk} \mod{N}$ and a $\tlp$ $Z_{\sk}$ containing $\sk$ (initially, one can set $\sk = 0$). Each contributor $i$ can update $\pk$ by adding $s_i$ homomorphically in the exponent and contributing an HTLP $Z_i=(g^{r_i}\mod{N},h^{r_i\cdot N}\cdot(1+N)^{s_i})$. The contribution must be accompanied by a proof of well-formedness. For the previous state $\pk, Z_{\sk}$, contributor $i$ proves that its contribution $\pk_i, Z_i$ passes the following checks:

\begin{description}
    \item[Check $\#1$.] It knows the discrete logarithm of $\pk_i$ with respect to the base $g$. This can be achieved with a proof of knowledge of the exponent~\cite{C:Schnorr89}.
    \item[Check $\#2$.]\label{item:check2} It knows the representation of the HTLP contribution $Z_i$ with respect to the bases $g, h^N, (1+N)$ (i.e., the discrete logarithms $r_i, r_i, s_i$). This can be proven by a ``knowledge of representation'' proof system in groups of unknown order (e.g., \cite{C:BonBunFis19}).
    \item[Check $\#3$.] Finally, the discrete logarithms $a,b,c$ from check \#2 are such that $a=b$ and $c={\sf dlog}_g(\pk_i)$.
\end{description}

The state is updated with the $i$th contribution iff all the checks pass. After the update, $Z_{\sk}:=Z_{\sk}\cdot Z_i$ and $\pk:=\pk\cdot \pk_i =g^{s+s_i}$. A single honest contributor suffices to guarantee a uniformly distributed keypair. 

% We prove the security of this protocol in \Cref{app:secproofs}.

% \noemi{We could compare/contrast this with key-updatable PKE~\cite{jaeger2018optimal,jost2019efficient}?}