\section{Voting and auction schemes}

We recall the specifics of FPTP, approval, range, and cumulative voting, along with single-item sealed bid auctions. The cryptographically relevant details of these schemes (i.e., the valid ballots' structure: their domain, Hamming weight, and norm) are summarized in~\Cref{tab:voting_schemes}. In~\Cref{sec:cicada_framework}, we create private voting protocols for these schemes of interest.

\def\sp{\hspace{1em}}
%%------------------- BEGIN Ballot domain TABLE ----------------------
\begin{table*}[tb]
    \centering
\makebox[\linewidth]{
    \setlength{\tabcolsep}{6pt}
    \setlength{\belowbottomsep}{6pt}
    \begin{tabular}{l@{\hspace{12pt}}ccc}
        \toprule
                        & \textbf{Submission domain}    & \textbf{Hamming wt}   & \textbf{Norm} \\\midrule
        FPTP voting     & $[0,1]^m$ & $\leq 1$& $\leq 1$ \\
        Approval voting     & $[0,1]^m$ & $\leq m$& $\leq m$ \\
        Range voting  & $[0,w]^m$ & $\leq m$      & $\leq wm$ \\
        Cumulative voting  & $[0,w]^m$ & $\leq m$      & $\leq w$ \\
        Ranked-choice voting (Borda) 
                        & $\pi([0,m-1])$& $m-1$     & $m(m-1)/2$ \\
        Quadratic voting
        & $[0,\sqrt{w}]^m$& $\leq m$& $\lVert \vec{b} \rVert_2^2 = \langle \vec{b}, \vec{b} \rangle = w$ \\
        \midrule
        Single-item sealed-bid auction & $[0, w]$  & 1 & $\leq w$ \\
        % \midrule
        % Bayesian truth serum %(\Cref{sec:voting_bayesian_truth})
        %                 & $[0,1]^m \times \mathbb{N}^m$ & $1, 1$ & $1, \leq m$ \\
        \bottomrule
    \end{tabular}
}
    \caption{Requirements for the domain, Hamming weight, and norm of a vector $\vec{b}$ for it to be a valid submission in various voting/auction schemes.
    $\pi(S)$ denotes the set of permutations of $S$. The norm is an $\ell_1$ norm unless otherwise specified. $m$ is the number of candidates, and $w$ is the maximum weight that can be assigned to any candidate.
    }
    \label{tab:voting_schemes}
\end{table*}

\paragraph{Majority, approval, range, and cumulative voting.} 
In the classic first-past-the-post (FPTP) voting scheme, voters can cast a vote of $1$ (support) for one candidate and $0$ for all others. A slight generalization of FPTP is approval voting, where users can assign a $1$ vote to multiple candidates, i.e., the cast ballot $s$ can be seen as $s\in\{0,1\}^{m}$, where $m$ is the number of causes. A further generalization is range voting, where users can give each candidate up to some weight $w$ (thus, approval is the special case where $w=1$). A related scheme is cumulative voting, where users can distribute a total of $w$ votes (points) among the candidates (now FPTP is a special case where $w=1$).
In each case, each candidate's points are tallied and the candidate with the highest number is declared the winner.

\paragraph{Ranked-choice voting.} 
% In a ranked-choice voting scheme, voters can signal more fine-grained preferences among $m$ candidates. In the Borda count version~\cite{emerson2013borda}, each voter can cast $m-1$ points to their first-choice candidate, $m-2$ points to their second-choice candidate, etc. In general, they can cast $m-k$ points to their $k$\textsuperscript{th} choice. 
% Several other counting functions exist for ranked voting, but in this work, we only focus on Borda counts. Our protocols can easily be adapted to other counting functions, such as the Dowdall system~\cite{fraenkel2014borda} via minor modifications.
In a ranked-choice voting scheme, voters can signal more fine-grained preferences among $m$ candidates by listing them in order of preference. There are multiple approaches to determining the winner, including single transferrable vote (STV) and instant runoff voting (IRV). In this work, we focus on the simpler Borda count version~\cite{Emerson13}, where each voter can cast $m-1$ points to their first-choice candidate, $m-2$ points to their second-choice candidate, etc., and the candidate with the most points is the winner. Our protocols can be adapted to similar counting functions, such as the Dowdall system~\cite{FraGro14}, via minor modifications.

\paragraph{Quadratic voting.} 
In quadratic voting~\cite{LalWey18}, each user's ballot is a vector $\vec{b} = (b_1, \dots, b_m)$ such that $\langle \vec{b}, \vec{b} \rangle = \lVert \vec{b} \rVert^2_2 \leq w$. Once again, the winner is determined by summing all the ballots and determining the candidate with the most points. Thus, this is also an additive voting scheme. 
However, proving ballot well-formedness efficiently in this particular case benefits greatly from the novel application of the residue numeral system (RNS) to private voting (see~\Cref{sec:packing}).

\paragraph{Single-item sealed-bid auction.} 
In a sealed-bid auction for a single item (e.g., an NFT or domain name), users submit secret bids to the auction contract. The domain of the bids might be constrained, e.g., $b\in\{0,1\}^{k}$ (in our implementations $k\approx 8-16$; see \Cref{sec:implementation}). Therefore, bidders must prove that their bid is well-formed, i.e., falls into that domain. Once all secret bids are revealed, the contract selects the highest bidder and awards them the auctioned item. The price the winner must pay depends on the auction scheme: e.g., highest bid in a first price auction, second-highest in a Vickrey auction. %\todo{should we include this too?}\istvan{Iterated on this. Opinions?}