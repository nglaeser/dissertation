\subsection{The Cicada Framework}

We present Cicada, our framework for non-interactive private auctions/elections, in \Cref{fig:cicada}. Cicada can be applied to voting and auction schemes where the scoring function $\Score = f \circ t$ has a linear tally function $t$. The framework uses a linear HTLP (\Cref{sec:tlp}), vector packing scheme (\Cref{sec:packing}), and matching NIZK (\Cref{sec:sigmas}) to ensure correctness of submissions by proving both the well-formedness of the puzzle and the solution's membership in $\X$.

At a high level, Cicada enables auction or voting schemes with the following five steps. 
\emph{First}, system participants agree on a delay parameter $\Ttime$ and packing parameter $\ell$. The $\Setup$ algorithm outputs HTLP public parameters $\pparam$ (note this might require private randomness) and initializes a set of the tally HTLPs $\mathcal{Z}$ containing zeros. 
\emph{Second}, a user $i$ uses the $\Seal$ algorithm to encode their submission (a bid or ballot) $\vec{v}_i$ into (a set of) HTLP(s) $\Z_i$. $\Seal$ also outputs a NIZK $\pi_i$ proving that the submission $\Z_i$ is well-formed, i.e., is in the domain $\mathcal{X}$ of the scoring function $\Score$. Users will send these submissions and corresponding proofs to the on-chain coordinator. 
\emph{Third}, the coordinator (Cicada smart contract) runs $\Aggr$ to verify the user-submitted proof $\pi_i$, and if it is valid, aggregate the user submission $\Z_i$ into the tally HTLPs $\Z$, resulting in updated tally HTLP(s) $\Z'$. 
\emph{Fourth}, after the voting/bidding period has ended, any party can open the tally HTLPs $\Z$ off-chain by running $\Open$, which outputs the opening(s) $\mathcal{S}$ of the tally HTLP(s) $\Z$ along with a proof of correct opening $\pi_{\textsf{open}}$ (using a proof of solution $\pos$, see \Cref{sec:sigmas}). The off-chain solver will send $\mathcal{S}, \pi_{\mathsf{open}}$ to the contract. 
\emph{Fifth}, the on-chain contract runs $\Finalize$ to verify the correctness of $\mathcal{S}$ by checking $\pi_{\mathsf{open}}$. If the check passes, it computes the auction/election winner(s) as $y=f(\vec{v})$.

% \noemi{In Cicada we need to talk about the function that is applied to the final tally, not the ballots -- maybe we could introduce a definition of ``linear'' protocols $\Sigma : \X^n \to \Y$ which are those functions that have an associated function $\Sigma_{\sf winner}: \X' \to \Y$ such that $\Sigma_{\sf winner}(t) = \Sigma(s_1, \dots, s_n)$ where $t = f(s_1, \dots, s_n)$ for some linear function $f$.}
%\noemi{difference between $\Sigma$ and $\Sigma_{\sf winner}$ is the domain, $\X^n$ and $\X$, resp.}

Intuitively, submission privacy follows from the security of the HTLP and zero-knowledge of the NIZK used: the submission can't be opened before time $\Ttime$ and none of the proofs leak any information about it. Non-malleability is enforced by requiring the NIZK to be a proof of knowledge and including the user's identity $i$ in the instance to prove, e.g., including it in the hash input of the Fiat-Shamir transform. This prevents a malicious actor from replaying a different user's ballot correctness proof.
% \istvan{Correctness is due to the correctness of the underlying NIZKs. Submission privacy is reduced to the HTLP sequentiality, while non-malleability is ensured by the proof of knowledge property of our applied NIZKs.} 

\begin{figure*}[t!]
\begin{mdframed}
\begin{center}
    \textbf{The Cicada Framework}
\end{center}
Let $\Sigma: \X^n \rightarrow \Y$ be the scoring function of a voting/auction scheme
where $\Score = f \circ t$ for a linear function $t$ and
% $\Sigma = f \circ g$ for some linear function $g$ and 
$\X = [0,w]^m$. 
Let $\htlp$ a linear HTLP, $\Ttime \in \NN$ a time parameter representing the election/auction length, and $(\PSetup, \pack, \unpack)$ a packing scheme.
Let $\NIZK$ be a NIZKPoK for 
submission correctness (language depends on $\Sigma, \htlp$; see \Cref{sec:sigmas})
% the language $\{(i, Z) : \exists x \text{ s.t. } Z \in {\sf Im}(\htlp.{\sf Gen}(\pack(x))) \land\ x \in \X\}$ 
and \pos\ be a proof of correct HTLP solution (see \Cref{sec:sigmas}). 

\hrulefill %%%%%%%%%%%%%%%%%%%%%%%%%%%%%%
\begin{description}
    \item[$\Setup(\secparam, \Ttime, \ell) \randout (\pp, \Z)$.] 
    Set up the public parameters $\pp_{\NIZK} \sample \NIZK.\Setup(\secparam)$, $\pp_{\sf tlp} \sample \htlp.\Setup(\secparam, \Ttime)$, and $\pp_{\sf pack} \gets \PSetup(\ell, w)$. 
    Let $\Z = \{Z_j\}_{j \in [m/\ell]}$ where $Z_j \sample \htlp.{\sf Gen}(0)$. Output $\pp := (\pp_{\sf tlp}, \pp_{\sf pack}, \pp_\NIZK)$ and $\Z$.
    \item[$\Seal(\pp,i, \vec{v}_i) \randout (\Z_i, \pi_i)$.] Parse $\vec{v}_i := \vec{v}_{i,1} || \dots || \vec{v}_{i,m/\ell}$. Compute $Z_{i,j} \gets \htlp.{\sf Gen}(\pack(\vec{v}_{i,j}))~\forall j \in [m/\ell]$ and $\pi_i \gets \NIZK.\prove((i, \Z_i), \vec{v}_i)$.
    % $s_{i,j} \gets \pack(\vec{v}_{i,j})$ for all $j \in [m/\ell]$. 
    Output $(\Z_i := \{Z_{i,j}\}_{j \in [m/\ell]}, \pi_i)$ 
    \item[$\Aggr(\pp,\Z,i,\Z_i,\pi_i) \rightarrow \Z'$.] If $\NIZK.\vrfy((i, \Z_i), \pi_i) = 1$, update $\Z$ to $\Z \boxplus \Z_i$. %, where $\boxplus$ is applied pairwise to elements of $\Z,\Z_i$.
    \item[$\Open(\pp,\Z) \rightarrow (\mathcal{S}, \pi_{\sf open})$.] Parse $\Z := \{Z_j\}_{j \in [m/\ell]}$ and solve for the encoded tally $\mathcal{S} = \{s_j\}_{j \in [m/\ell]}$ where $s_j \gets \htlp.{\sf Solve}(Z_j)$. Prove the correctness of the solution(s) as $\pi_{\sf open} \gets \pos.{\sf Prove}(\mathcal{S}, \Z, 2^\Ttime)$ and output $(\mathcal{S}, \pi_{\sf open})$.
    \item[$\Finalize(\pp, \Z, \mathcal{S}, \pi_{\sf open}) \rightarrow \{y,\perp\}$.] If $\pos.{\sf Verify}(\mathcal{S}, \Z, 2^\Ttime, \pi_{\sf open}) \neq 1$, return $\perp$. Otherwise, parse $S := \{s_j\}_{j \in [m/\ell]}$ and let $\Vec{v} := \vec{v}_1 || \dots || \vec{v}_{m/\ell}$, where $\vec{v}_j \gets \unpack(s_j)~\forall j \in [m/\ell]$. Output 
    % $y = f(\vec{v})$.
    $y$ such that $y = \Sigma(\vec{v})$.
\end{description}
\end{mdframed}
\caption{The Cicada framework for non-interactive private auctions and elections.}
\label{fig:cicada}
\end{figure*}

As we will see next, this captures many common schemes such as cumulative voting and sealed-bid auctions. 
Cicada introduces a crucial design choice via the packing parameter $\ell\in[m]$, which defines a storage-computation trade-off that we detail in~\Cref{sec:implementation}. %The on-chain footprint of a protocol is minimized by using a ballot correctness proof system $\sf NIZK$ with low verification cost. % this seems kind of obvious? And if anything it belongs in the implementation section

% \subsection{Application to voting schemes}\label{sec:voting}

% \todo{revisit this} The main reason we are interested in building protocols for cardinal voting schemes is that they allow voters to express more fine-grained preferences. Put differently, they bypass Arrow's impossibility theorem~\cite{arrow1950difficulty}, i.e., cardinal voting schemes satisfy non-dictatorship, unrestricted domain, independence of irrelevant alternatives and Pareto efficient. 

\paragraph{Additive voting.}
Many common voting schemes are ``additive'', meaning each ballot (a length-$m$ vector) is simply added to the tally, and a finalization function $f$ is applied to the tally after the voting phase has ended to determine the winner. 
Additive voting schemes include first-past-the-post (FPTP), approval, range, and cumulative voting. Simple ranked-choice voting schemes, e.g., Borda count~\cite{Emerson13}, are also additive, differing only in what qualifies as a ``proper'' ballot (restrictions on vector entry domain, vector norm, etc.; see \Cref{tab:voting_schemes}). 
Thus, we can use Cicada to instantiate private voting protocols for all these schemes.

% \paragraph{Majority vote and approval voting.}
% We start off by describing a protocol for approval voting. Approval voting is a generalization of majority vote, whereby, a voter can cast a vote for multiple candidates. In case of a majority vote, a ballot can be thought of as a binary vector of Hamming weight $1$, while approval voting lifts this restriction, i.e., the ballot can be any length-$m$ binary vector $\vec{b} \in\{0,1\}^{m}$. The challenge is to design a protocol that applies a single HTLP for tallying the approval votes.

% Following~\cite{hirt2000efficient}, we encode the $i$th voter's ballot as the integer
% \begin{equation}\label{eq:approval_voting_ballot_encoding}
%     % b_j\in \Big\{
%     % \sum\limits_{i=1}^{m} b_j^{(i)} (n+1)^{i-1}: b_j^{(i)} = 1 \iff\textit{j votes for candidate i}\in[m]\Big\}. 
%     b_i\in \Big\{
%     \sum\limits_{j=1}^{m} b_i^{(j)} (n+1)^{j-1}: b_i^{(j)} = 1 \iff\textit{i votes for candidate j}\in[m]\Big\}. 
% \end{equation}
% Note the number of votes for candidate $j$ can be obtained as $\sum\limits^{n}_{i=1}b_i\bmod{(n+1)^{j-1}}$.

% \begin{figure}[h!]
% \begin{mdframed}
% \begin{description}
%     \item[$\mathsf{Setup}(\secparam, \Ttime) \randout (\pparam, Z^*)$.]  Set up HTLP parameters $\pparam_{\sf tlp} \sample \htlp.\Setup(\secparam, \Ttime)$. Let $Z^* \sample \htlp.{\sf Gen}(\pparam_{\sf tlp}, 0)$. Output $\pparam := (\pparam_{\sf tlp},Z^*)$.
%     \item[$\Seal(\pparam,i,b_i) \randout (Z_i, \pi_{\sf i})$.] Let $b_i$ encode the $i$th user's approval vote as in Equation~\ref{eq:approval_voting_ballot_encoding}. Let $Z_i \gets \htlp.{\sf Gen}(\pparam_{\sf tlp}, b_i)$. The  correctness proof $\pi_i$ is obtained as $\pi_i:=\bigwedge\limits_{i=1}^{m}\zkprove(\mathcal{R}_{\pokemon}(Z_i/Z'_j,0,n^{j-1};b_i) \lor\mathcal{R}_{\pokemon}(Z_i/Z'_j,1,n^{j-1};b_i))$, where $Z'_j$ is the residual approval vote, i.e., $Z'_j\gets\htlp.{\sf Gen}(\pparam_{\sf tlp}, b_i\mod{n^{j-1}})$. 
    
%     Output $(Z_i,\pi_i)$.
%     \item[$\Aggr(\pparam,Z^*,Z_j,\pi_{\sf j}) \rightarrow Z^*$.] If $\zkvfy((\pparam,j,Z_j),\pi_{\sf j}) \neq 1$, leave $Z^*$ unchanged and abort. Otherwise, update $Z^* = Z^* \oplus Z_j$.
%     \item[$\Open(\pparam,Z^*,\mathcal{R}) \rightarrow s^*$.] Solve the tally puzzle for $s^* \gets \htlp.{\sf Solve}(\pparam_{\sf tlp}, Z^*)$.
%     \item[$\Finalize(\pparam, s^*) \rightarrow {\sf winner}$.] Compute the sum of the votes for each candidate $j \in [m]$ as $s^*_j := s^* \mod{n^{j-1}}$. The winner is $\argmax_j \{s^*_1, \dots, s^*_m\}$.
% \end{description}


% \end{mdframed}
% \caption{A protocol for approval voting among $m$ candidates using a single HTLP.}
% \label{fig:voting_approval}
% \end{figure}

% \paragraph{Ranked choice voting, range voting, and cumulative voting.}

% In a na\"ive Borda count protocol, the tally consists of $m$ HTLPs $\{Z^{*}_i\}^{m}_{i=1}$ (one per candidate) that encode the accumulated points of each candidate, initially set to $0$.  During the voting phase, each voter creates $m$ HTLPs $\{Z_i\}^{m}_{i=1}$ encoding integers from $0$ to $m-1$. Afterwards, voters homomorphically add each ballot HTLP $Z_i$ to the corresponding tally HTLP $Z^{*}_i$. This solution is suboptimal in terms of the number of HTLPs needed. 
% 
% We show a protocol that requires only a single HTLP. Let the $i$th voter's preference vector be defined as $\mathbf{q^{(i)}}=(q^{(i)}_1,q^{(i)}_2,\dots,q^{(i)}_m)$, which is required to be a permutation of $[0,m-1]$ .
% % $\forall j:q^{(i)}_j\in[0,m-1]\land\forall j,k: q^{(i)}_{j}\neq q^{(i)}_k$ i.e., it is a permutation of $[0,m-1]$. 
% % Let $\mathbf{p}=(p_1,p_2,\dots,p_m)$ be a vector of prime numbers such that $\min_{i}p_i\geq (m-1)n$ and $\prod\limits_{i=1}^{m} p_i\leq N$.
% Now encode $\vec{q}^{(i)}$ as an integer $s_i$ using the Chinese Remainder Theorem (see \cref{sec:packing}) and encode it in an HTLP $Z_i$.
% % Specifically, $s_i\equiv q^{(i)}_{j}\mod{p_j}$, which can be efficiently found by the Chinese Remainder's Theorem (CRT).
% The user submits their puzzle $Z_i$ with the accompanying zero-knowledge proofs for ballot correctness, and the on-chain contract homomorphically adds $Z_i$ and $Z^*$.

% At the end of the voting phase, a single party solves the tally HTLP $Z^*$ to obtain the solution $s^{*}$. We claim that the total preference points (Borda counts) for candidate $j$ can be computed as $s^{*}\mod{p_j}$. The correctness of this protocol for ranked voting follows from the observation that for any preference $j$, we have 
% \begin{equation}\label{eq:borda_correctness_argument}
%     s^{*}\bmod{p_j}=\sum\limits^n_{i=1}s_i\bmod{p_j}\equiv\sum\limits^n_{i=1}q^{(i)}_j\bmod{p_j}=\sum\limits^n_{i=1}q^{(i)}_j\leq(m-1)n. 
% \end{equation}

\begin{theorem}\label{thm:cicada}
    Given a linear scoring function $\Sigma$, a secure NIZKPoK $\NIZK$ and proof of solution $\pos$, a secure $\htlp$, and a packing scheme $({\sf PSetup},\allowbreak \pack,\allowbreak \unpack)$, the Cicada protocol $\Pi_\Sigma$ (\Cref{fig:cicada}) is a secure time-locked voting/auction protocol. % i.e., it satisfies correctness, bid/ballot privacy, and non-malleability.
\end{theorem}

% We delegate the full proof to \Cref{sec:secproofs}.
\begin{proof}
\def\Exp{\ensuremath{\mathsf{ExpSPriv}_{\Pi_\Sigma}^\adv(\secpar,\Ttime,i)}}
    For simplicity, we give a proof for the simple case of $\X=[0,1]$, i.e., submissions consist of a single bit, but our argument generalizes to larger domains $\X$. Let $n \in \mathbb{N}$ be the number of users.

    The correctness of the Cicada framework (cf.~\Cref{def:correctness_cicada}) follows by construction and from the correctness of the underlying building blocks (i.e., soundness in the case of the proof systems).
    
    Next, we prove submission privacy.
    Let $\Exp$ be the original submission privacy game for the Cicada scheme $\Pi_\Sigma$ with $\Ttime$-bounded adversary $\adv$, cf.~\Cref{def:submission_privacy}. We define a series of hybrids to show that 
    \[ 
        \Pr[\Exp = 1] \leq \negl 
    \]
    for all $\secpar,\Ttime \in \mathbb{N}$ and $i\in[n]$.
    
    \underline{$\hybrid_0$:} This is the original game $\Exp$, where $Z_i \gets \htlp.{\sf Gen}(b)$ and $\pi_i \gets \NIZK.\prove(i, Z_i, b)$.
    
    \underline{$\hybrid_1$:} Replace $\pi$ with $\tilde{\pi} \gets {\sf NIZK}.\Sim(i, Z_i)$. $\hybrid_1$ is indistinguishable from $\hybrid_0$ by the zero-knowledge property of $\sf NIZK$.

    \underline{$\hybrid_2$:} Replace $Z_i$ with $Y_i \gets \htlp.{\sf Gen}(1-b)$ and $\tilde{\pi}$ with $\tilde{\sigma} \gets {\sf NIZK}.\Sim(i, Y_i)$. $\hybrid_1$ and $\hybrid_2$ are indistinguishable because the distributions $\{Z_i, \Sim(i, Z_i)\}$ and $\{Y_i, \Sim(i, Y_i)\}$ are indistinguishable since $\{Z_i\}, \{Y_i\}$ are indistinguishable by the security of $\htlp$.

    % \underline{$\hybrid_3$:} Replace $\tilde{\pi}$ with $\tilde{\sigma} \gets {\sf NIZK}.\Sim(i, Y_i)$. $\hybrid_3$ is indistinguishable from $\hybrid_2$ again by security of $\htlp$: since the distributions of $Z_i, Y_i$ are indistinguishable, so must the distributions ...
    
    \underline{$\hybrid_3$:} Replace $\tilde{\sigma}$ with $\sigma \gets \NIZK.\prove(i, Y_i, 1-b)$. $\hybrid_3$ is indistinguishable from $\hybrid_2$ by the zero-knowledge property of $\sf NIZK$.

    This series of hybrids implies $\Pr[b'=b] \approx_\secpar \Pr[b'=1-b]$, where $b'$ is the output of $\adv$ in $\hybrid_0$ or $\hybrid_3$, respectively. Therefore $\Pr[\Exp = 1] \leq \frac{1}{2} + \negl[\secpar]$.
    
    % --- reduction attempt ---
    % Suppose towards a contradiction that there exists a $\Ttime$-bounded adversary $\adv$ who wins the submission privacy game with non-negligible probability. We will use give a $\Ttime$-bounded adversary $\bdv$ capable of violating either the zero-knowledge of $\sf NIZK$ or the privacy of $\htlp$.

    % Given a puzzle $Z$ containing an unknown bit $b_1$ and a proof $\pi$ that some independent unknown bit $b_2 \in [0,1]$, $\bdv$ constructs two different queries $(Z, \tilde{\pi})$ and $(\tilde{Z}, \pi)$ as follows: sample $i \sample [n]$ and compute $\tilde{\pi} \gets {\sf NIZK}.\Sim(i, Z)$ and $\tilde{Z} \gets \htlp.{\sf Gen}(0)$. Query $\adv$ on both 

    Finally, we show that if $\NIZK$ is a PoK and $\htlp$ is secure, then Cicada is non-malleable, cf.~\Cref{def:non_malleability}. Suppose towards a contradiction that Cicada is malleable. We will use this and the fact that $\NIZK$ is a PoK to construct an adversary $\bdv$ which has non-negligible advantage in the $\htlp$ security game. Again, we work in the simple case $\X = [0,1]$, i.e., $m,\ell,w=1$, but the argument generalizes to other parameter settings.
    
    Since by our assumption Cicada is malleable, there exists $\adv$ which outputs $(i, \cdot, \Z_i, \pi_i) \notin \mathcal{Q}$ such that $\NIZK.\vrfy((i,\allowbreak \Z_i), \pi)=1$ with non-negligible probability. 
    % Also, $\sf NIZK$ is a PoK, so it has an efficient knowledge extractor $\Ext$. 
    Given a puzzle $Z_b$ containing some unknown bit $b$, $\bdv$ works as follows. First, it computes $(\pparam, Z) \sample \Setup(\secparam, \Ttime, 1)$ and sends them to the non-malleability adversary $\adv$. $\bdv$ responds to $\adv$'s oracle queries $(j, b_j)$ with honestly computed $(Z_j, \pi_j)$, keeping track of queries and responses in the set $\mathcal{Q}$. When $\adv$ outputs $(i, Z_i, \pi_i)$, $\bdv$ looks for $(i, b_i, Z_i, \pi_i) \in \mathcal{Q}$ and outputs $b_i$. Since $\adv$ has non-negligible advantage, it follows that $\NIZK.\vrfy((i, Z_i), \pi_i) = 1$. This implies that either $\Pr[b_i=b] = \frac{1}{2} + \negl[\secpar]$ or $\NIZK$ is not knowledge sound. Both possibilities contradict our assumptions, namely that the $\htlp$ is secure and the $\NIZK$ is knowledge sound. Thus, Cicada must be non-malleable.
\end{proof}