\section{Our Contributions}\label{sec:cicada_contrib}

This work introduces Cicada, a framework for fair and non-interactive auctions and voting.\footnote{Adult North American cicadas emerge from the ground at predictable intervals of 13 or 17 years. Similarly, our ballots/bids remain hidden only for a fixed interval.}
Cicada uses time-lock puzzles (TLPs)~\cite{RivShaWag96} to achieve tally-privacy in a non-interactive, trustless, and efficient way. Intuitively, the TLPs play the role of commitments to bids/ballots that any party can open after a predefined time, avoiding the reliance on a second ``reveal'' round. 
Since solving a TLP is computationally intensive, 
we use \emph{homomorphic} TLPs (HTLPs) to combine bids/ballots on-the-fly into a smaller number of TLPs which is independent of the number of users. Therefore, besides removing interactivity, Cicada offers improved storage compared to commit-reveal schemes, which incur linear storage costs. 
Since fully homomorphic TLPs are not practically efficient, we only require additive HTLPs, which have efficient constructions~\cite{C:MalThy19}. 
We use vector packing to encode bids/votes into HTLPs and use custom zero-knowledge proofs (ZKPs) suited to the HTLP setting to enforce submission correctness.
Besides simple protocols like first-past-the-post (FPTP) voting, we show how to realize more complicated auctions and elections, e.g., cumulative voting.
We define a syntax and security properties for time-locked voting and auction protocols and prove the security of Cicada with respect to these definitions.

\paragraph{Implementation}
Cicada is the first protocol for efficient, trustless, non\hyp interactive, and fair auction/voting protocols. 
Despite the previous existence of our building blocks (HTLPs, ZKPs, and vector packing), combining these primitives in a way that maintains practical efficiency is non-trivial and has been a stumbling block for deployment. Thus, we also view our implementation of Cicada and the analysis of optimal deployment choices 
as a central contribution of this work. 

We provide open-source implementations tailored to the popular Ethereum Virtual Machine (EVM) with a word size of $256$ bits. 
Our protocols are particularly well-suited to the EVM since, unlike prior works, Cicada is non-interactive and we do not need to keep bids, ballots, and proofs in persistent storage as they are not required for any subsequent round. 
Our most efficient realizations work in $\mathbb{Z}^{*}_{N}$ for $N\approx 2^{1024}$, groups which are not natively supported by the EVM. We therefore also implement several gas-efficient libraries to support modular arithmetic in such groups of unknown order which may be of independent interest. 
We demonstrate in~\Cref{sec:cicada_eval} that our protocols can be run today directly on Ethereum and cost only several USD on popular layer-2 solutions.

\paragraph{Non-goals}
As discussed, our protocols target fairness via tally-privacy. Thus we view everlasting privacy as a non-goal. Nonetheless, we observe that Cicada is not inherently at odds with it and outline how it could be added to the framework in \Cref{sec:everlasting_ballot_privacy}.
Relatedly, we consider anonymity an orthogonal problem. Where desired, users can add anonymity via privacy-enhancing overlays~\cite{PerSemSto19,semaphore,CCS:Neff01}. 

The voting literature also often views coercion-resistance, i.e., an adversary's inability to coerce voters to cast specific ballots, as a crucial property. Coercion-resistance is implied by receipt-freeness, since it makes it impossible for a coerced voter to prove their compliance to the adversary~\cite{STOC:BenTui94}. Similar to previous work~\cite{USENIX:Adida08}, we see receipt-freeness as a desideratum in high-stakes democratic elections but view it as a non-goal in our protocol design and target low coercion-risk settings. Still, we sketch an extension in~\Cref{sec:extension_coercion_resistance} wherein we achieve coercion-resistance via a different pathway than receipt-freeness. %We leave it to future work to achieve receipt freeness for on-chain voting schemes.