\begin{table*}[tb]
  \centering
  \makebox[\linewidth]{
    \setlength{\tabcolsep}{3pt}
    \setlength{\belowbottomsep}{6pt}
    \begin{tabular}{lccccc} 
    \toprule
    \multirow{2}*{\textbf{Approach}} & \multirow{2}*{No TTPs} & Non-interactive & Everlasting & \multirow{2}*{Practical} \\
      & & voting & privacy & \\
  \midrule 
  MPC~\cite{FC:BDJNPT06,C:AOZZ15}
  & \no & \yes & \yes & \yes \\ 
  ZK proofs~\cite{USENIX:Adida08,maci,plume,rln}
  & \no & \yes & \yes & \yes \\ 
  FHE~\cite{PQCRYPTO:CGGI16,CCS:DLNS17} 
  & \no & \yes & \yes & \no \\
  HE+ZK proofs~\cite{CCS:DLNS17}
  & \no & \yes & \yes & \halfcirc \\ 
  Commit-reveal~\cite{AUSC:FujOkaOht93,WTSC:GalYou18}
  & \yes & \no & \no & \yes \\
  TLPs+HE~\cite{ESORICS:CJSS21} (\Cref{sec:seq_mpc_tlp})
  & \halfcirc & \yes & \no & \yes \\\midrule
  Cicada (this work) 
  & \yes$^*$ & \yes & \halfcirc & \yes \\ 
    \bottomrule
  \end{tabular}
  }
  \caption{Major approaches for tally-private auction/voting schemes. MPC stands for secure multi-party computation. \cite{CCS:DLNS17} aims to be practical but uses lattice-based cryptography, which is not feasible on-chain today. \cite{C:AOZZ15,ESORICS:CJSS21} require a trusted setup but no TTP. 
  The asterisk indicates that our scheme does not inherently require any TTPs (in particular, if the class group HTLP construction (\Cref{sec:htlp-choice}) is used, Cicada has a transparent setup). Everlasting privacy can be added via an extension (\Cref{sec:everlasting_ballot_privacy}).
  }
  \label{tab:approaches}
  \end{table*}
  %------------------------------END Approaches Table---------------------------