%-----------------------------BEGIN Approaches Table---------------------------
\begin{table*}[tb]
    \footnotesize
    \centering
    \makebox[\linewidth]{
     \setlength{\tabcolsep}{3pt}
     \setlength{\belowbottomsep}{6pt}
    %  \newcolumntype{R}{>{\begin{turn}{90}\begin{minipage}}r%
    % <{\end{minipage}\end{turn}}%
    % }
    %  \newcolumntype{v}{>{\rotatebox{270}}r<{}}}
     \begin{tabular}{lccccc} 
     \toprule
      \textbf{Approach} & NI        & No TTPs   &Efficient & Tally privacy & Ev. ballot privacy \\
    \midrule 
    Commit-reveal~\cite{AUSC:FujOkaOht93,WTSC:GalYou18}              & \no                 & \yes   &\yes    &\yes & \no     \\ %\color{green}{One-way functions} \\
    Zero-knowledge proofs~\cite{maci}              & \no                 & \no   &\yes    &\yes & \yes     \\ %\color{green}{One-way functions} \\
    Fully homomorphic TLPs~\cite{C:MalThy19}  & \yes                 & \yes   &\no    &\yes & \no     \\ %\color{red}{Indistinguishability Obfuscation} \\
    FHE~\cite{STOC:Gentry09,PQCRYPTO:CGGI16,CCS:DLNS17}                              & \yes                 & \no   &\no    &\yes & \yes     \\ %\color{green}{Bounded Distance Decoding Problem} \\
    %Functional encryption~\cite{boneh2011functional}&\color{green}{$1$}&\cmark&\cmark&\xmark& \\
    Multi-party computation~\cite{FC:BDJNPT06,C:AOZZ15}                        & \no                 & \halfcirc &\yes    &\yes & \yes     \\ %\color{red}{Trusted third parties}& \\
    TLPs + homomorphic encryption~\cite{ESORICS:CJSS21}                    & \yes                 & \halfcirc   &\yes    &\yes & \no     \\\midrule
    HTLPs (our approach)                                    & \yes                 & \yes$^{*}$   &\yes     & \yes & \halfcirc \\ %\color{yellow}{Sequentiality of repeated squaring}~\cite{rotem2020generically}\\[1ex]
     \bottomrule
     $^{*}$ when using class groups &&&&&
    \end{tabular}
    }
    \caption{Qualitative comparison of major cryptographic approaches for designing private auction/voting schemes. 
    % (H)TLP stands for (homomorphic) time-lock puzzle. 
    %Our approach can support everlasting privacy while remaining non-interactive and efficient. 
    % No TTPs refer to the absence of trusted third parties. 
    % An asterisk indicates that those schemes can be instantiated with a transparent setup using class groups, cf.~\Cref{sec:feasability}.}
    NI = non-interactive, Ev. = everlasting. \cite{C:AOZZ15,ESORICS:CJSS21} require a trusted setup but no TTP. Everlasting ballot privacy can be added to our approach via an extension (\Cref{sec:cicada_extensions}).
    }
    \label{tab:approaches}
    \end{table*}
    %------------------------------END Approaches Table---------------------------