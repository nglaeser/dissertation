\paragraph{Special Notation} We will use $n$ as the number of users, $m$ as the number of candidates, and $w$ as the maximum weight to be allocated to any one candidate in a ballot/bid ($n,m,w \in \mathbb{N}$). For simplicity and without loss of generality, we assume the user identities are unique integers $i \in [n]$.
We generally use $i \in [n]$ to index users and $j \in [m]$ for candidates.

Auctions and voting are essential applications of Web3. For example, decentralized marketplaces run auctions to sell digital goods like non-fungible tokens (NFTs)~\cite{opensea_auction} or domain names~\cite{ARXIV:XWYLLX21}, while decentralized autonomous organizations (DAOs) deploy voting schemes to enact decentralized governance~\cite{optimismgov}. Most auction or voting schemes currently deployed on blockchains, e.g., NFT auctions on OpenSea or Uniswap governance~\cite{ARXIV:FriMulWat22}, lack bid/ballot privacy. This can negatively influence user behavior, for example, by vote herding or discouraging participation~\cite{FC:ElkLip04,WTSC:GalYou18,FC:SuzYok03}. The lack of privacy can cause surges in congestion and transaction fees as users try to outbid each other to participate, a negative externality for the entire network.

\begin{table*}[tb]
  \centering
  \makebox[\linewidth]{
    \setlength{\tabcolsep}{3pt}
    \setlength{\belowbottomsep}{6pt}
    \begin{tabular}{lccccc} 
    \toprule
    \multirow{2}*{\textbf{Approach}} & \multirow{2}*{No TTPs} & Non-interactive & Everlasting & \multirow{2}*{Practical} \\
      & & voting & privacy & \\
  \midrule 
  MPC~\cite{FC:BDJNPT06,C:AOZZ15}
  & \no & \yes & \yes & \yes \\ 
  ZK proofs~\cite{USENIX:Adida08,maci,plume,rln}
  & \no & \yes & \yes & \yes \\ 
  FHE~\cite{PQCRYPTO:CGGI16,CCS:DLNS17} 
  & \no & \yes & \yes & \no \\
  HE+ZK proofs~\cite{CCS:DLNS17}
  & \no & \yes & \yes & \halfcirc \\ 
  Commit-reveal~\cite{AUSC:FujOkaOht93,WTSC:GalYou18}
  & \yes & \no & \no & \yes \\
  TLPs+HE~\cite{ESORICS:CJSS21} (\Cref{sec:seq_mpc_tlp})
  & \halfcirc & \yes & \no & \yes \\\midrule
  Cicada (this work) 
  & \yes$^*$ & \yes & \halfcirc & \yes \\ 
    \bottomrule
  \end{tabular}
  }
  \caption{Major approaches for tally-private auction/voting schemes. MPC stands for secure multi-party computation. \cite{CCS:DLNS17} aims to be practical but uses lattice-based cryptography, which is not feasible on-chain today. \cite{C:AOZZ15,ESORICS:CJSS21} require a trusted setup but no TTP. 
  The asterisk indicates that our scheme does not inherently require any TTPs (in particular, if the class group HTLP construction (\Cref{sec:htlp-choice}) is used, Cicada has a transparent setup). Everlasting privacy can be added via an extension (\Cref{sec:everlasting_ballot_privacy}).
  }
  \label{tab:approaches}
  \end{table*}
  %------------------------------END Approaches Table---------------------------

Existing private voting protocols~\cite{maci,plume,rln} achieve privacy at the cost of introducing a trusted authority who is still able to view all submissions.
Alternatively, the only private \emph{and} trustless auction in deployment we are aware of~\cite{ARXIV:XWYLLX21} uses a two-round commit-reveal protocol: in the first round, every party commits to their bid, and in the second round they open the commitments and the winner can be determined.
Other protocols relying on more heavyweight cryptographic building blocks have been proposed in the literature.
We summarize the various approaches for private voting and auctions in \Cref{tab:approaches}.
Unfortunately, all of them suffer from at least one of the following limitations, hindering widespread adoption:

\begin{description}
    \item[Interactivity.] Interactivity is a usability hurdle that often causes friction in the protocols' execution. Mandatory bid/ballot reveals are also a target for censorship.     
    %\item[Lack of censorship-resistance]  Voting and auction schemes can be easily censored due to their public nature.
    A malicious party can bribe the block proposers to exclude certain bids or ballots until the auction/voting ends~\cite{ARXIV:PaiResFox23}. %With bid/ballot privacy the cost of censorship can be increased substantially.
    %
    \item[Trusted third party (TTP).] Many protocols~\cite{maci} use a trusted coordinator to tally submissions during the voting/bidding phase. This introduces a strong assumption which is at odds with the trustless ethos of the blockchain ecosystem.
    %
    \item[Inefficiency.] 
    Compute and storage costs are substantial bottlenecks in decentralized applications running on a public blockchain. Some approaches~\cite{PQCRYPTO:CGGI16,CCS:DLNS17} avoid the previous pitfalls by relying on complex cryptographic primitives such as fully-homomorphic encryption (FHE), whose overheads are impractical in the blockchain setting.
    % \item[Inefficiency.] Introducing more complicated cryptographic primitives such as fully-homomorphic encryption (FHE) introduces additional overheads which can incur extra gas fees or at least extra time for all participants.
\end{description}