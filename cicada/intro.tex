Auctions and voting are essential applications of Web3. For example, decentralized marketplaces run auctions to sell digital goods like non-fungible tokens (NFTs)~\cite{opensea_auction} or domain names~\cite{ARXIV:XWYLLX21}, while decentralized autonomous organizations (DAOs) deploy voting schemes to enact decentralized governance~\cite{optimismgov,ARXIV:FFHVW24}. Most auction or voting schemes currently deployed on blockchains, e.g., OpenSea NFT auctions or Uniswap governance \cite{ARXIV:FriMulWat22}, do not hide user bids/votes during the submission phase. This can negatively influence user behavior, for example, by vote herding or discouraging participation~\cite{FC:ElkLip04,WTSC:GalYou18,FC:SuzYok03}. It can also cause network congestion as users engage in bidding wars, leading to surges in transaction fees and causing a negative externality for the entire network.

\paragraph{Privacy as a means for fairness}
These negative outcomes can be avoided if a protocol offers fairness, i.e., leaks no information about votes or bids until the end of the submission phase. A natural way to achieve this is to hide everything but the outcome. Existing works keep submissions hidden by introducing one or more trusted authorities who are still able to view all submissions and are trusted to correctly compute the result~\cite{FC:BCDGJK09,C:AOZZ15} or prove its correctness in zero-knowledge (ZK)~\cite{USENIX:Adida08,maci,plume,rln}. 
Others rely on advanced cryptographic primitives~\cite{PQCRYPTO:CGGI16,CCS:DLNS17}, e.g., (fully) homomorphic encryption ((F)HE~\cite{STOC:Gentry09}). 
Neither of these approaches is suitable for an on-chain setting. Relying on a trusted third party (or a threshold of them) is at odds with the ethos of decentralization, whereas advanced cryptographic primitives are impractical on-chain today, incurring high computation or storage fees.

In any case, the privacy guarantees offered by these protocols are stronger than is necessary for achieving fairness: bids/ballots need only remain hidden \emph{until the end of the submission phase}. This is acceptable in practice, with some deployed systems (e.g., ENS domain name bids~\cite{ARXIV:XWYLLX21}) achieving fairness via this weaker notion of privacy. We refer to this weaker notion as \emph{tally-privacy} (as opposed to everlasting privacy).
In fact, limited privacy can actually be a desideratum in certain settings, e.g., in representative democracies or in DAOs where delegates' votes are published to encourage accountability.

A common paradigm for tally-private, trustless protocols is a two-round commit-reveal protocol~\cite{AUSC:FujOkaOht93,WTSC:GalYou18,CCS:TAFWBM23}. In the first round, every party commits to their bid or ballot. In the second round, they open their commitments, and the winner is determined.
Due to the lightweight cryptography used, these schemes are efficient enough to be deployed on-chain.
However, their interactivity is a usability hurdle that causes friction in the protocols' execution since users must remember to come online and reveal their submissions after the voting or bidding period has elapsed. The reveal phase can also be an avenue for censorship: an attacker can bribe block proposers to exclude the openings of certain bids or ballots until after the result has been determined~\cite{ARXIV:PaiResFox23}.

We summarize these approaches for private voting and auctions in \Cref{tab:approaches}.

\begin{table*}[tb]
  \centering
  \makebox[\linewidth]{
    \setlength{\tabcolsep}{3pt}
    \setlength{\belowbottomsep}{6pt}
    \begin{tabular}{lccccc} 
    \toprule
    \multirow{2}*{\textbf{Approach}} & \multirow{2}*{No TTPs} & Non-interactive & Everlasting & \multirow{2}*{Practical} \\
      & & voting & privacy & \\
  \midrule 
  MPC~\cite{FC:BDJNPT06,C:AOZZ15}
  & \no & \yes & \yes & \yes \\ 
  ZK proofs~\cite{USENIX:Adida08,maci,plume,rln}
  & \no & \yes & \yes & \yes \\ 
  FHE~\cite{PQCRYPTO:CGGI16,CCS:DLNS17} 
  & \no & \yes & \yes & \no \\
  HE+ZK proofs~\cite{CCS:DLNS17}
  & \no & \yes & \yes & \halfcirc \\ 
  Commit-reveal~\cite{AUSC:FujOkaOht93,WTSC:GalYou18}
  & \yes & \no & \no & \yes \\
  TLPs+HE~\cite{ESORICS:CJSS21} (\Cref{sec:seq_mpc_tlp})
  & \halfcirc & \yes & \no & \yes \\\midrule
  Cicada (this work) 
  & \yes$^*$ & \yes & \halfcirc & \yes \\ 
    \bottomrule
  \end{tabular}
  }
  \caption{Major approaches for tally-private auction/voting schemes. MPC stands for secure multi-party computation. \cite{CCS:DLNS17} aims to be practical but uses lattice-based cryptography, which is not feasible on-chain today. \cite{C:AOZZ15,ESORICS:CJSS21} require a trusted setup but no TTP. 
  The asterisk indicates that our scheme does not inherently require any TTPs (in particular, if the class group HTLP construction (\Cref{sec:htlp-choice}) is used, Cicada has a transparent setup). Everlasting privacy can be added via an extension (\Cref{sec:everlasting_ballot_privacy}).
  }
  \label{tab:approaches}
  \end{table*}
  %------------------------------END Approaches Table---------------------------