\chapter{Introduction}

Bitcoin~\cite{Nakamoto08} was the first digital currency to successfully implement a fully trustless and decentralized payment system. Underpinning Bitcoin is a \emph{blockchain}, a distributed append-only ledger to record transactions. In Bitcoin, the blockchain's consistency is enforced via a ``proof-of-work'' (PoW) consensus mechanism in which participants solve difficult computational puzzles (hash preimages) to append the newest bundle of transactions (a block) to the chain. 

Ethereum~\cite{ethereum} introduced programmability via \emph{smart contracts}, special applications which sit on top of the consensus layer and can maintain state and modify it programmatically. This has led to the emergence of a number of decentralized applications enabling more diverse functionalities. % in a trustless manner.

% This proposal describes various applications of cryptography to blockchains. In \Cref{sec:building-blocks}, I discuss work on privacy-enhancing building blocks of blockchains, namely zero-knowledge proofs. In particular, I summarize my work adding universally composable security to zero-knowledge proofs with updatable common reference strings~\cite{CSF:AGRS24}. \Cref{sec:applications} moves on to privacy-enhancing blockchain applications, specifically my work analyzing the security of a class of coin mixing services~\cite{CCS:GMMMTT22} and designing a framework for non-interactive and private on-chain elections and auctions~\cite{EPRINT:GSZB23}. Finally, in \Cref{sec:proposed}, I discuss my proposed work: a new construction for a threshold cryptocurrency wallet and a systematic analysis of approaches to on-chain key distribution.

Unfortunately, as the number of applications and their users has increased, developers have been forced to sacrifice trustlessness and sometimes even security or privacy in favor of performance and scalability. This dissertation uses cryptographic tools to enable blockchain applications which are both \emph{practical} and \emph{secure} while staying true to the original blockchain ethos and minimizing trust. All the works discussed were created in collaboration with industry practitioners and have seen interest or deployment in the blockchain ecosystem.

In \Cref{sec:prelims}, I introduce the necessary background and building blocks used throughout this dissertation. 
\Cref{sec:naysayer} describes \emph{naysayer proofs}, a new paradigm for verifying zero-knowledge proofs in the so-called optimistic setting. One notable application of naysayer proofs is for scalability, where it sits between the existing solutions---optimistic rollups and zero-knowledge rollups---and can provide better performance and accessibility for certain parties in rollup ecosystems. Since their publication, naysayer proofs have seen interest in production deployment from at least two startups (that I am aware of) in the blockchain space. % Hungry Cats Studio (a16z spring 2024 CSX in London), Ethos Labs

In \Cref{sec:cicada}, I move to on-chain applications and describe Cicada, a smart contract protocol for realizing the first \emph{fair} and \emph{non-interactive} on-chain elections and auctions. This resolves important security and usability hurdles present in previous systems, which are widely used for on-chain governance votes or sales of digital goods such as non-fungible tokens (NFTs). Cicada is accompanied by a Solidity smart contract implementation, making it easily deployable on Ethereum and a large number of ``layer 2'' (L2) chains. Furthermore, we have been reached out to Optimism, one of the largest L2s in the Ethereum ecosystem, to discuss the use of Cicada for their retroactive public goods funding (retroPGF) vote.\footnote{\url{https://community.optimism.io/docs/governance/}}

Finally, \Cref{sec:hcwl} moves off-chain and looks at securing the edges of the system: cryptocurrency wallets. I envision a system that allows users to conveniently back up their rarely-used, high-value keys (such as the signing keys of high-balance wallets). This new setting necessitates a novel set of design requirements. I develop new security definitions that capture them and construct a UC-secure protocol that implements threshold BLS signatures in our new model. The protocol is practically efficient for the envisioned large numbers of custodians: for a $67$-out-of-$100$ threshold configuration, setup takes $170$ms and signing less than $1$ms.
This design was created with collaborators from Lit Protocol\footnote{\url{https://www.litprotocol.com/}} and the Linux Foundation and is in the process of being adopted in production by the former. An Apache 2.0-licensed implementation is also publicly available in Hyperledger Labs\footnote{\url{https://github.com/hyperledger-labs}}, a popular open source organization.

I conclude in \Cref{sec:conclusion} by discussing potential future directions for these three works and a broader outlook on the role of cryptography in blockchains. 

% \subsection{Universal Composability (UC) Framework}

% \copied{This section is copied from the KSP overleaf}

% The universal composability (UC) framework~\cite{FOCS:Canetti01} defines the security requirements of a protocol via an \emph{ideal functionality} which is executed by a trusted party. To prove that a protocol \emph{UC-realizes} a given ideal functionality, we show that the execution of this protocol (in the real or hybrid world) can be \emph{emulated} in the ideal world, where in both worlds there is an additional adversary $\env$ (the environment) which models arbitrary concurrent protocol executions. Specifically, we show that for any adversary $\adv$ attacking the protocol execution in the real world (by controlling communication channels and corrupting parties involved in the protocol execution), there exists an adversary in the ideal world (the simulator) $\Sim$ who can produce a protocol execution which no environment $\env$ can distinguish from the real-world execution.

% All parties are represented as probabilistic interactive Turing machines (ITMs) with input, output, and communication tapes. For simplicity, we assume that all communication is authenticated, so an adversary can only delay but not forge or modify messages from parties involved in the protocol. Therefore, the order of message delivery is also not guaranteed (asynchronous communication). We consider a PPT malicious, adaptive adversary who can corrupt or tamper with parties at any point during the protocol execution (modeled in the ideal world via the interfaces in \cref{fig:FSign3}).

% The execution in both worlds consists of a series of sequential party activations. Only one party can be activated at a time (by writing a message on its input tape). In the real world, the execution of a protocol $\Pi$ occurs among parties $P_1, \dots, P_n$ with adversary $\adv$ and environment $\env$. In the ideal world, interaction takes place between dummy parties $\tilde{P}_1, \dots, \tilde{P}_n$ communicating with the ideal functionality $\F$, with the adversary (simulator) $\Sim$ and environment $\env$. In both cases, the environment is activated first and activates either the adversary ($\adv$ resp. $\Sim$) or an uncorrupted (dummy) party by writing on its input tape. If $\adv$ (resp. $\Sim$) is activated, it can take an action or return control to $\env$. After a (dummy) party (or $\F$) is activated, control returns to $\env$. The protocol execution ends when $\env$ completes an activation without writing on the input tape of another party.

% We denote with $\real_{\Pi,\adv,\env}(\secpar,x)$ the random variable describing the output of the real-world execution of $\Pi$ with security parameter $\secpar$ and input $x$ in the presence of adversary $\adv$ and environment $\env$. We write the corresponding distribution ensemble as $\{ \real_{\Pi,\adv,\env}(\secpar, x) \}_{\secpar \in \mathbb{N}, x \in \{0,1\}^*}$. The output of the ideal-world interaction with ideal functionality $\F$, adversary (simulator) $\Sim$, and environment $\env$ is represented by the random variable $\ideal_{\F,\Sim,\env}(\secpar, x)$ and corresponding distribution ensemble $\{ \ideal_{\F,\Sim,\env}(\secpar, x) \}_{\secpar \in \mathbb{N}, x \in \{0,1\}^*}$.

% The actions each party can take are summarized below:
% \begin{itemize}
%     \item Environment $\env$: \textbf{read} output tapes of the adversary ($\adv$ or $\Sim$) and any uncorrupted (dummy) parties; \textbf{write} on the input tape of one party (the adversary $\adv$ or $\Sim$ or any uncorrupted (dummy) parties).
%     \item Adversary $\adv$: \textbf{read} its own tapes and the outgoing communication tapes of all parties; \textbf{deliver} a pending message to party by writing on its input tape \emph{or} \textbf{corrupt} a party (which becomes inactive: its tapes are given to $\adv$ and $\adv$ controls its actions from this point on, and $\env$ is notified of the corruption).
%     \item Real-world party $P_i$: only follows its code (potentially writing to its output tape or sending messages via its outgoing communication tape).
%     \item Dummy party $\tilde{P}_i$: acts only as a simple relay with the ideal functionality $\F$, copying inputs from its input tape to its outgoing communication tape (to $\F$) and any messages received on its ingoing communication tape (from $\F$) to its output tape.
%     \item Adversary $\Sim$: \textbf{read} its own input tape and the public headers of the messages on $\F$'s and dummy parties' outgoing communication tapes; \textbf{send} a message to $\F$ by writing on its incoming incoming communication tape \emph{or} \textbf{deliver} a message from $\F$ to a dummy party by copying it from $\env$'s outgoing communication tape to the dummy party's incoming communication tape \emph{or} \textbf{corrupt} a dummy party (which becomes inactive: its tapes are given to $\Sim$ and $\Sim$ controls its actions from this point on, and $\env$ and $\F$ are notified of the corruption).
%     \item Ideal functionality $\F$: \textbf{read} incoming communication tape; \textbf{send} messages to the dummy parties and/or adversary $\Sim$ by writing to its outgoing communication tape.
% \end{itemize}

% \begin{definition}
%     We say a protocol $\Pi$ \emph{UC-realizes} an ideal functionality $\F$ if for any PPT adversary $\adv$, there exists a simulator $\Sim$ such that for any environment $\env$, the distribution ensembles $\{ \real_{\Pi,\adv,\env}(\secpar, x) \}_{\secpar \in \mathbb{N}, x \in \{0,1\}^*}$ and $\{ \ideal_{\F,\Sim,\env}(\secpar,\allowbreak x) \}_{\secpar \in \mathbb{N},x \in \{0,1\}^*}$ are computationally indistinguishable.
% \end{definition}

% \begin{definition}[NP Relation]
%     An NP relation $\Rel$ is \todo{...}

%     The language corresponding to $\Rel$ is defined as $\Lang_\Rel := \{ Y : \exists y \text{ such that } (Y, y) \in \Rel \}$.
% \end{definition}

% \begin{definition}[CPA-security]
%     An encryption scheme $\Pi_\ENC$ is said to be CPA-secure if \todo{...}
% \end{definition}

% \begin{definition}[Rerandomizable encryption]
%     An rerandomizable encryption scheme $\Pi_\ENC$ has an additional algorithm $\sf Rand$ \todo{...}
% \end{definition}

% \begin{definition}[existential unforgeability]
%     A digital signature scheme $\Pi_\DS$ is said to be existentially unforgeable (or EUF-CMA-secure) if \todo{...}
% \end{definition}