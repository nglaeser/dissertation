In 2008, Satoshi Nakamoto introduced Bitcoin, the first digital currency without a trusted authority whose security is maintained by a decentralized blockchain. Since then, a plethora of decentralized applications have been proposed utilizing blockchains as a public bulletin board. This growing popularity has put pressure on the ecosystem to prioritize scalability at the expense of trustlessness and decentralization.

This work explores the role cryptography has to play in the blockchain ecosystem to improve performance while maintaining minimal trust and strong security guarantees. 
% I discuss how to generically add universally composable security to any non-interactive zero-knowledge proof (NIZK), a crucial building block in many blockchain systems, in a way that is compatible with an updatable common reference string. This strengthens security for any system relying on NIZKs, including many blockchains and blockchain applications, while maintaining minimal trust assumptions.
I discuss a new paradigm for scalability, called \emph{naysayer proofs}, which sits between optimistic and zero-knowledge approaches.
Next, I cover two on-chain applications:
First, I show how to improve the security of a class of coin mixing protocols by giving a formal security treatment of the construction paradigm and patching the security of an existing, insecure protocol. 
Second, I show how to construct practical on-chain protocols for a large class of elections and auctions which simultaneously offer fairness and non-interactivity without relying on a trusted third party. 
Finally, I look to the edges of the blockchain and formalize new design requirements for the problem of backing up high-value but rarely-used secret keys, such as those used to secure the reserves of a cryptocurrency exchange, and develop a protocol which efficiently meets these new challenges.
% introduce a new type of threshold wallet which adds offline components at each custodian to improve security while maintaining distributed trust. Furthermore, the construction enables transparency by allowing the wallet owner to request custodians to prove ``remembrance'' of the secret key material.

All of these works will be deployed in practice or have seen interest from practitioners. These examples show that advanced cryptography has the potential to meaningfully nudge the blockchain ecosystem towards increased security and reduced trust.

% Blockchains are inherently public, [but sometimes we want privacy. We need to use crypto to do this. And so on and so forth\dots] \noemi{actually probably need to expand to also include ``security'', since the threshold sigs project isn't really about privacy}
% A key challenge in [my area of research] is [problem I halfway figured out how to solve]. (And so on and so forth\dots)