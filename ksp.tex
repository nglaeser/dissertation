\subsection{Threshold cryptocurrency wallets in the hot-cold paradigm}\label{sec:ksp}

One ongoing work to be finalized and submitted in the coming weeks is a threshold wallet construction. To achieve very strong security guarantees we assume that the wallet client $C$ places trust in $n$ institutional custodians who consist of both a \emph{hot}, i.e., online, component, and a \emph{cold}, i.e., offline, component. Our construction is secure even when all $n$ hot storages and up to a threshold $t$ of cold storages are compromised. We assume the cold storages are resource-constrained and minimize their computation and communication with the hot storages. In particular, for each message signed, the cold storages send a single message-specific ``cold partial signature'' to the corresponding hot storage. The hot storage also computes a partial signature, and the hot and cold partial signatures are combined into a partial threshold signature. Additionally, the signing key shares can be refreshed in a distributed manner to provide proactive security, i.e., prevent an adversary from incrementally compromising parties' signing key shares. The client can furthermore request ``proofs of remembrance'' from any of the hot and cold storages to ensure they are continuing to persistently store the secret key material to avoid loss of funds.

In more detail, we give a construction for (threshold) BLS signatures~\cite{AC:BonLynSha01}. Let $\GG$ be a group of prime order $p$ with generator $g$, $H: \{0,1\}^* \to \GG$ be a cryptographic hash function, and $h: \GG \to \ZZ_p$ be a universal hash function\footnote{In actuality, we use a function $h: \GG^3 \to \ZZ_p$, but since the details of $h$ do not affect the intuition of the construction,  for ease of exposition we use the simpler function $h: \GG \to \ZZ_p$ in this document.}.
Given $t$-out-of-$n$ Shamir shares $\sk_1, \dots, \sk_n$ of a BLS signing keypair $(\sk, \vk)$, each cold storage $P_i^\cold$ will store an encryption secret key $\dk_i$ and the corresponding hot storage $P_i^\hot$ stores an encrypted signing key share $\hx_i := \sk_i + h(\ek_i^{\sk})$. To sign a message $m \in \{0,1\}^*$, $P_i^\cold$ uses its decryption key to compute a cold partial signature $\sigma_i^\cold := H(m)^{h(\vk^{\dk_i})}$, which it sends (out of band) to $P_i^\hot$. $P_i^\hot$ uses its encrypted key share to compute a hot partial signature $\sigma_i^\hot := H(m)^{\hx_i}$ and combines it with $\sigma_i^\cold$ to obtain a partial BLS signature $\sigma_i := \sigma_i^\hot / \sigma_i^\cold = H(m)^{\sk_i}$. Given $t$ valid partial signatures, a full BLS signature on $m$ can be computed in the standard manner (i.e., reconstructing the secret key in the exponent).

The homomorphic nature of the encryption scheme allows key share refreshes via addition of Shamir secret shares of zero. We show how to use KZG polynomial commitments~\cite{AC:KatZavGol10} and their evaluation proofs to show well-formedness of the share updates. The hot storage proofs of remembrance are also achieved via KZG evaluation proofs, but blinded via a folklore technique~\cite[\S2.5]{EC:CHMMVW20} so as not to publicly reveal the value of the hot storage key shares.

We are still working on distributed protocols for the initial key share generation and distribution as well as a hot storage recovery protocol in which other parties can help recover a lost encrypted key share $\hx_i$.
The protocol will have a UC security proof and a proof of concept implementation, both of which are still in progress.