\subsection{Our Contribution}
Building on the design requirements stated above, we introduce a new wallet system called \sysname (``THReshold Online/offline BACKups''). \sysname is designed to serve as a high-security backup system for high-value cryptographic keys, simultaneously meeting all the outlined design requirements while offering high efficiency.
Below, we describe in order how our construction addresses each of the design requirements.

\begin{enumerate}
    \item \textbf{Non-interactive Protocol.}
    \sysname requires almost no interaction between parties both for recovery and to verify the correct storage of key material. This allows it to efficiently scale to large, highly distributed networks of custodians. Furthermore, cold parties in \sysname can even \emph{generate} their key material independently, without interacting with the hot parties or with each other. This reduces the cold parties' attack surface even more.
    \item \textbf{Hot-Cold Model.} 
    Our construction distributes secret shares among \emph{pairs} of hot and cold parties. In this way, an attacker must corrupt \emph{both} the cold and hot components of a custodian to obtain its key share. This requirement to corrupt a threshold $t$ of hot-cold \emph{pairs} is different from a generic $2t$-out-of-$2n$ threshold wallet, since the attacker cannot forge a signature by corrupting any arbitrary set of $2t$ parties: corrupting, e.g., $2t$ hot parties should be of no use in forging a signature (in fact, even corrupting all $n$ hot parties should be useless). Indeed, our threat model is instead comparable to a Boolean signing policy which requires at least $t$ pairs of hot \emph{and} cold parties to contribute, and is therefore much more difficult to attack since the cold components are almost always offline.
    
    The cold parties in our construction only need to store a constant number of elements, which importantly is independent of the number of custodied secrets. This is particularly desirable since cold parties are normally resource-constrained devices such as hardware wallets.
    
    \item \textbf{Proactive Refresh.}
    Our protocol allows periodic updates of the hot key shares. % in a way that still lets hot parties prove the shares are being stored correctly. % (unchanging) verification key (see below).
    These share refreshes force an attacker to compromise at least $t$ hot-cold pairs within a single \emph{epoch} to exfiltrate the key: otherwise, any key material obtained is made obsolete by the refresh operation. 
    
    \item \textbf{Proofs of Remembrance.}
    We show how the hot and cold parties in our protocol can produce ``proofs of remembrance'', i.e., zero-knowledge proofs that they still possess their key material. Our proof systems must guarantee that the underlying secret key is still available and unchanged while also accounting for changing values of the \emph{individual} shares due to proactive refreshes. We show how to meet both needs simultaneously, allowing a client to intermittently and independently audit its (hot or cold) custodians. This allows the client to ensure that its key material has not been overwritten or forgotten, and its funds are still accessible.

    \item \textbf{Threshold BLS Construction.} \sysname produces standard BLS signatures so that it is not possible to identify users of the system (i.e., holders of high-value cryptographic keys). As we argued above, this is crucial for a system designed to protect high-value secrets.
\end{enumerate}

We implemented \sysname and show that it is practically efficient for essentially all reasonable settings of $t, n$. Producing a signature takes less than 1ms and computing proofs of remembrance is on the order of milliseconds. Our implementation is publicly available in Hyperledger Labs\footnote{\url{https://github.com/hyperledger-labs/agora-key-share-proofs}} and is Apache 2.0-licensed. 

\paragraph{Open problems} BLS signatures have the advantage of being simple to understand and deploy. This has led to their widespread use in production systems, including Ethereum's consensus protocol~\cite[\S2.9.1]{eth2book}, Filecoin~\cite{filecoin-spec}, transactions on the Chia Network~\cite{chia-bls}, and a BLS smart contract wallet~\cite{bls-wallet}. With the event of account abstraction on Ethereum~\cite{account-abstraction}, users can specify alternative signature schemes to verify their transactions, further easing the adoption of BLS-based wallets.  However,  ECDSA~\cite{EPRINT:GenGolNar16,SP:DKLs18,CCS:LinNof18,CCS:GenGol18,EPRINT:AumHamShl20,EPRINT:GKSS20,EPRINT:DJNPO20,CCS:CGGMP20,PKC:CCLST20,EPRINT:CCLST21,CANS:Pettit21,SP:ANOSS22} and Schnorr~\cite{SAC:KomGol20,C:BCKMTZ22,EPRINT:BatLonMen22} are also popular threshold signature schemes in the literature. We leave the exciting problem of extending our results to these signature schemes open. %\todo{Remove/edit if I include the FROST protocol}