\subsection{Security Analysis}\label{sec:security}

We prove our scheme secure in the universal composability framework\footnote{Specifically, we use the version presented in \cite{EPRINT:CLOS02}.}, which we summarize in \Cref{sec:uc}.

Messages to and from the ideal functionalities consist of two pieces: a \emph{header} and the \emph{contents}. The header normally consists of a session identifier $\sid$, a description of the action the functionality should take/has taken, and the sender/recipient. In this paper, we will use the convention of putting the message contents in parentheses so it is clear where the (public) header ends and the (private) contents begin, e.g., $(\texttt{sid}, \mathsf{Action}, P_{\sf sender}, (\mathsf{contents}))$ or $(\texttt{sid}, \mathsf{ActionDone}, P_{\sf recipient}, \mathsf{publicvars}, (\mathsf{contents}))$.

\paragraph{Ideal Functionality}

\begin{figure*}
    \centering
    \fbox{\parbox{\textwidth}{%
    \renewcommand{\labelitemi}{\textendash}
    \setlength{\itemsep}{0pt}
    % \textbf{Public parameters:} Groups $\gp_1, \gp_2$ of prime order $p$ with generators $g_1, g_2$, respectively; a degree-$d$ KZG common reference string $\crs$; a hash function $\blshash: \{0,1\}^* \to \gp_1$; an extractor $\lhlhash: \gp_2^2 \to \ZZ_p$.
    % \textbf{Internal variables:}\hfill\\
    % $\coldlist = \{(P^\cold, \ek, \dk, \texttt{leaked}, \texttt{tampered}, \texttt{corrupted}, \texttt{allowc})\}$, a table with the state of each cold storage's key material.\\
    % $\clientlist = \{(C, \mathcal{I}, t, y, \com)\}$, a table of registered clients and their metadata.\\
    % $\hotlist = \{(P^\hot, C_j, \hx, \pi, \texttt{time}, \texttt{leaked}, \texttt{tampered})\}$, which keeps track of the hot key material and whether it has been leaked or tampered with.\\
    % $\hotstates = \{P^\hot, \texttt{corrupted}, \texttt{allowc}\}$, which keeps track of hot corruptions.
    % \medskip\\
    \textbf{Registration}
    \begin{itemize}
    \item On input $(\texttt{sid},\mathsf{ColdRegister},P_i^\cold)$, it proceeds as follows:
        \begin{enumerate}
        \item Sample $\dk_i \sample \ZZ_p$ and set $\ek_i := g_2^{\dk_i}$. 
        \item Delete any existing entries for $P_i^\cold$ in $\coldlist$ and add $(P_i^\cold, \ek_i, \dk_i,\allowbreak \texttt{leaked}:=0,\texttt{tampered}:=0,\texttt{corrupted}:=0,\texttt{allowc}:=1)$ to $\coldlist$.
        \item Output $(\texttt{sid},\mathsf{ColdRegistered}, P_i^\cold, \ek_i)$.
        \end{enumerate}
    
    \item On input $(\texttt{sid},\mathsf{ClientRegister}, C, (t, \mathcal{I}))$, where $\mathcal{I} = \{(P_i^\hot, P_i^\cold)\}_{i \in [n]}$ is a set of institutional entities and $t \leq n$ a signing threshold, it proceeds as follows:
        \begin{enumerate}
        \item For each $i \in [n]$, retrieve $(P_i^\cold, \ek_i, *, *, *, *, *)$ from $\coldlist$. If a public key for some party is unavailable then output $y := \bot$.
        \item Otherwise, sample $x \sample \ZZ_p \setminus \{0\}$. Let $y := g_2^x$.
        \item Generate $t$-of-$n$ Shamir shares of $x$ as $x_1, \ldots x_n$. Let $y_i := g_2^{x_i} ~\forall i \in [n]$.
        \item Interpolate the degree-$n$ polynomial $\tilde{f}$ such that $\tilde{f}(i) = \lhlhash(\ek_i^x) + x_i~\forall i \in [n]$. Compute $\com \gets \mathsf{KZG}.\Commit(\crs, \tilde{f})$. % and ${\sf d}\com := g_1^{\tau^{d-n} \cdot \tilde{f}(\tau)}$.
        \item Delete any existing entries $(C, *, *, *, *) \in \clientlist$ and add $(C, \mathcal{I}, t, y, \com)$ to $\clientlist$. Send $(\texttt{sid},\allowbreak \mathsf{ClientRegistered}, C, y, \{y_i\}_{i \in [n]})$ to $C$.
        \item For each $i \in [n]$:
        \begin{enumerate}
            \item Compute $(\hx_i, \pi_i) \gets \mathsf{KZG}.\Open(\crs, \tilde{f}, i)$. %and let $\aux_i := (\com, \pi_i)$, 
            \item Output $(\texttt{sid},\mathsf{ClientRegistered}, P_i^\hot, (C, b_i=1))$.
            \item Once $\adv$ has allowed delivery of the message, delete any existing entries $(P_i^\hot, C, *, *, *, *, *) \in \hotlist$ and add $(P_i^\hot, C,\allowbreak \hx_i, \pi_i, \texttt{time} := 0, \texttt{leaked} := 0,\allowbreak \texttt{tampered} := 0)$ to $\hotlist$. Delete any existing entries $(P_i^\hot, \texttt{corrupted}, \texttt{allowc}) \in \hotstates$ and add $(P_i^\hot, \texttt{corrupted} := 0, \texttt{allowc} := 1)$.
        \end{enumerate}
        \end{enumerate}
    \end{itemize}
    }}
    \caption{The BLS \hcwl functionality $\FSign$ (registration).}
        \label{fig:FSign1}
    \end{figure*}
\begin{savenotes}
    \begin{figure*}
    % \begin{figure}[tbhp]
    \centering
    \fbox{\parbox{.9\textwidth}{%
    \renewcommand{\labelitemi}{\textendash}
    \setlength{\itemsep}{0pt}
    \textbf{Signing}
    \begin{itemize}
        \item On input $(\texttt{sid},\mathsf{TSign}, P_i^\hot, (C, m))$, retrieve $(C, \mathcal{I}, *, \vk, *) \in \clientlist$ and $(P_i^\hot, P_i^\cold) \in \mathcal{I}$. Then:
        % if $\exists t,y$ for the largest $t$ such that $(y, \hx_i, \aux_i, t, *) \in \hotlist_i$ and $\pred(y,\hx_i, \aux_i) = 1$ \todo{Define \pred.} 
        \begin{enumerate}
            \item Get $(P_i^\cold, *, \dk_i, *, \texttt{tampered}, \texttt{corrupt}_\cold, \texttt{allowc}) \in \coldlist$. If $\texttt{allowc}=1$, set it to 0.
            \begin{itemize}%[\labelitemi=*]
                \item If $\texttt{corrupt}_\cold = 1$, send $(\texttt{sid}, \mathsf{CSignRequest}, \adv,\allowbreak (P_i^\cold, C, m))$ to $\adv$, wait for response $b$, and set $b_\cold := (b \land \neg \texttt{tampered})$. Otherwise ($\texttt{corrupt}_\cold = 0$), set $b_\cold := \neg \texttt{tampered}$.
                % \item Otherwise, compute $\sigma_i^\cold := \blshash(m)^{\lhlhash(\vk^{\dk_i})}$.
            \end{itemize}
            \item Retrieve $(P_i^\hot, C, \hx_i, *, \texttt{time}, *, \texttt{tampered}, \texttt{corrupt}_\hot, *) \in \hotlist$ for the maximum $\texttt{time}$. 
            \begin{itemize}
                \item If $\texttt{corrupt}_\hot = 1$, send $(\texttt{sid}, \mathsf{HSignRequest}, \adv, (P_i^\hot, C, m))$ to $\adv$, wait for response $b$, and set $b_\hot := (b \land \neg \texttt{tampered})$. Otherwise, set $b_\hot := \neg \texttt{tampered}$.
                % \item Otherwise, compute $\sigma_i^\hot := \blshash(m)^{\hx_i}$. 
            \end{itemize}
            \item If $b_\cold \land b_\hot$, compute $\sigma_i^\cold := \blshash(m)^{\lhlhash(\vk^{\dk_i})}$ and $\sigma_i^\hot := \blshash(m)^{\hx_i}$. Let $\sigma_i := \sigma_i^\hot/\sigma_i^\cold$. Output $(\texttt{sid}, \mathsf{TSignResult}, P_i^\hot, (C, m, \sigma_i))$.
            \begin{itemize}
                \item If $(\texttt{corrupt}_\cold \land \neg \texttt{corrupt}_\hot)$, also send $\sigma_i^\hot$ to $\adv$. If $(\neg \texttt{corrupt}_\cold \land \texttt{corrupt}_\hot)$, send $\sigma_i^\cold$ to $\adv$.
                \item Additionally, for every party $P_j^\hot$ such that $(P_j^\hot, *) \in \mathcal{I}$, retrieve $(P_j^\hot, *, \texttt{allowc}) \in \hotstates$ and set $\texttt{allowc}$ to 0.
            \end{itemize}
            \item If instead $\neg (b_\cold \land b_\hot)$, output $(\texttt{sid}, \mathsf{TSignResult}, P_i^\hot, (C, m, \perp))$.
            \begin{itemize}
                \item If $(\texttt{corrupt}_\cold \land \neg \texttt{corrupt}_\hot)$ or $(\neg \texttt{corrupt}_\cold \land \texttt{corrupt}_\hot)$, also sample $r \sample \ZZ_p$ and send $\blshash(m)^r$ to $\adv$.
            \end{itemize}
        \end{enumerate}
    
    \end{itemize}
    
    \textbf{Proofs of Remembrance}
    \begin{itemize}
        % \noemi{anyone ($Q$) can request a CProof} 
        \item On input $(\texttt{sid},\mathsf{CProof}, P_i^\cold, (C))$, retrieve $(P_i^\cold, \ek_i, \dk_i, *, *, \texttt{corrupted}, *) \in \coldlist$.
        % check if $\exists t, \ek_i, \dk_i $ such that $(P_i^\cold, \ek_i, \dk_i, t,  *) \in \coldlist$ 
        If $\texttt{corrupted} = 1$, send $(\sid, \mathsf{CProofRequest}, \adv, (P_i^\cold))$ to the adversary, who will send back a bit $b^*$ to represent whether the query should be responded to honestly. Set $b := b^* \land (\ek_i = g_2^{\dk_i})$ and
        % $\ek_i = g_2^{\dk_i}$, set $\pi \gets \Pi_{\sf DL}.\prove((\ek_i, g_2); \dk_i)$, else $\pi := \perp$; then 
        output $(\sid, \mathsf{CProofResult}, C, (P_i^\cold, b))$.
    
        \item On input $(\sid,\mathsf{HProof}, C, (P_i^\hot))$, retrieve $(P_i^\hot, C, \hx_i, \pi_i, *, *, *, \texttt{corrupted}, *) \in \hotlist$ and $(C, *, *, *, \com) \in \clientlist$. %Parse $\aux_i := (\com, *)$ and check if 
        % $\exists t$ such that $(y, \hx_i, \aux_i, t, *) \in \hotlist_i$ and 
        % $\pred(y,\hx_i, \aux_i, \pi_i) = 1$ \todo{Define \pred}. 
        If $\texttt{corrupted} = 1$, send $(\sid, \mathsf{HProofRequest}, \adv, (P_i^\hot))$ to the adversary, who will send back a bit $b^*$. Set $b := b^* \land \mathsf{KZG}.\vrfy(\crs, \com, i, \hx_i, \pi_i)$ and
        % compute $\pi \gets \Pi_{\sf EKS}.\prove(\crs, \com, (\hx_i, \pi_i))$, else set $\pi := \perp$; then 
        output $(\sid, \mathsf{HProofResult}, C, (P_i^\hot, b))$.
    \end{itemize}
    
    % \textbf{Client Updates}
    \textbf{Proactive Refresh}
    \begin{itemize}
    \item On input $(\sid,\shref,C)$, it proceeds as follows:
        \begin{enumerate}
        %\item Sample $x \sample \ZZ_p$. 
        \item Output $\bot$ if a prior share refresh from $C$ is still pending.\footnotemark
        \item Retrieve $(C, \mathcal{I}, t, y, \com, \texttt{time}) \in \clientlist$ for the maximum value of $\texttt{time}$. 
        \item Generate $t$-out-of-$n$ Shamir shares of $0$ as $x_1, \ldots, x_n \in \ZZ_p$ using polynomial $f$. 
        \item Compute $\ucom \gets \mathsf{KZG}.\Commit(\crs, f)$ and $\com' := \com \cdot \ucom$.
        %, $\dcom' := g_1^{\tau^{d-t+1} \cdot f(\tau)}$, and $(0, \pi_0') \gets \mathsf{KZG}.\Open(\crs, f, 0)$.
        \item Add $(C, \mathcal{I}, t, y, \com', \texttt{time++})$ to $\clientlist$ and send $(\sid, \shref\mathsf{Result}, C, (1))$ to $C$.
        \item For each $i \in [n]$: 
        \begin{enumerate}
            \item Compute $(\delta_i, \pi_i') \gets \mathsf{KZG}.\Open(\crs, f, i)$. % and let $\aux_i' := (\com', \pi_i')$.
            \item Send $(\sid, \shref\mathsf{Result}, P_i^\hot, (C, b_i=1))$ to $P_i^\hot$. \noemi{(If $C$ is corrupt, $\adv$ decides the value of $b_i$.)}
            \item Once $\adv$ has delivered/allowed delivery of the message, retrieve $(P_i^\hot, C, \hx_i, \pi_i,\allowbreak \timeT,  *, \texttt{tampered}) \in \hotlist$ and add $(P_i^\hot, C, \hx_i+\delta_i, \pi_i \cdot \pi_i', \timeT\texttt{++}, \texttt{leaked} = 0, \texttt{tampered})$ to $\hotlist$. Also retrieve $(P_i^\hot, *, \texttt{allowc}) \in \hotstates$ and update $\texttt{allowc}$ to 1.
            % \todo{What if $\hx_i, \pi_i$ are tampered? need to do the $\hx, \pi$ update to the last \emph{untampered} values? or check if KZG verifies?} \noemi{Our core protocol has no concept of ``recovering'' from a tampering or checking for tampering during updates, so no we don't check anything like that. You could imagine in an implementation/extension that you could have a protocol for checking whether a party's signing material has been corrupted (e.g., by requesting a signature on zero) and then run some recovery protocol if so (\cref{sec:ss-rec}).}
        \end{enumerate}
        % \item Output $(\texttt{sid}, \mathsf{ClientUpdateResult}, C, (b=1))$.
        \end{enumerate}
        % \item On input $(\texttt{sid},\mathsf{ClientUpdateDeliver}, C, I_i)$ from $\adv$, where $I_i$ is an institutional entity, update $(C, \hx_i, \aux_i, \texttt{time},  *)$ in $\hotlist_i$ to $(C,\hx_i+\delta_i, \aux_i',\texttt{time++},  \texttt{unleaked}:=1)$. Send $(\texttt{sid}, \mathsf{ClientUpdateResult}, C, \delta_i, \aux_i')$ to $P_i^\hot$.
    
    
        %\sanjam{how are multiple updates handled? What is one update arrives late -- how does that affect signing?}
      %  \item On input $(\texttt{sid},\mathsf{ClientUpdateDeliver}, P_i)$ from $\adv$ send $(\texttt{sid},\mathsf{ClientSecretShareUpdate}, C, \delta_i)$ to $P_i^{\hot}$ and set $c := c+\delta_i$ and $\texttt{flag}_i^{g^x} = 1$.
    \end{itemize}
    }}
        \caption{The BLS \hcwl functionality $\FSign$ (signing, proofs of remembrance, and refresh).}
        \label{fig:FSign2}
    \end{figure*}
    \end{savenotes}
    \footnotetext{We assume share refreshes are sequential and delivery to all $P_i$ precedes any new refreshes.} % \noemi{I think we don't need to assume this. If it doesn't happen, correctness doesn't apply.}}

We now present our hot-cold threshold BLS functionality $\FSign$, whose interfaces can only be called by \emph{either} an institutional hot or cold storage (specified by the superscripts $\hot$ and $\cold$) or a client. For readability, we split $\FSign$ into three figures. The first, \Cref{fig:FSign1}, describes how parties register in the system.
The signing, share refresh, and proof of remembrance interfaces of $\FSign$ are described in \Cref{fig:FSign2}.

The most complicated part of the functionality is the signing interface, which provides partial BLS signatures $\sigma_i$ on requested messages. For uncorrupted institutions $I_i$ (that is, neither $P_i^\cold$ and $P_i^\hot$ are corrupt), $\sigma_i$ is computed honestly.
If the hot party has been corrupted but the cold remains honest, the functionality asks the adversary whether to use the correct value for the hot signature; if so, it computes $\sigma_i$ correctly and also sends the cold signature to the adversary. Otherwise, it outputs $\sigma_i = \bot$ and sends $\blshash(m)^r$ for a uniform $r$ to $\adv$ (as the ``corresponding'' cold signature). (The idea is that in this case, the hot signature is incorrect so $\sigma_i$ will not verify, and the cold signature should be ``useless'' without it: it should look random to the adversary.) 
If the hot party is honest and the cold is corrupt, the functionality behaves in the same way but with the hot and cold roles reversed. 
Finally, if both parties in a pair are corrupt, the adversary will get no information about the hot or cold partial signatures from the functionality, which only outputs either the correct $\sigma_i$ or $\bot$, depending on whether the adversary says to compute the hot and cold partial signatures correctly.

We will argue that this implements threshold BLS signatures, so the security of a protocol implementing $\FSign$ follows from security of threshold BLS.

\paragraph{Adversarial interference} \Cref{fig:FSign3} gives the leak and tamper interfaces of $\FSign$, which an adversary can use to interfere with the information stored by the hot and cold storages in the system.
%\emph{only by the hot storages} in the system.
We also give an explicit corruption interface, which only allows non-adaptive corruptions of the cold and hot parties (in the latter case, on a per-epoch basis). The interfaces also capture the fact that the client $C$ is out of scope for corruption. 
% \noemi{maybe we can get away without that} 

\begin{figure*}
    % \begin{figure}[tbhp]
    \centering
    \fbox{\parbox{.9\textwidth}{%
    \renewcommand{\labelitemi}{\textendash}
    \setlength{\itemsep}{0pt}
    \textbf{Leak}
    \begin{itemize}
    \item On input $(\texttt{sid},\mathsf{Leak}, \adv, (P_i^\hot))$, for every entry $(P_i^\hot, C_j, \hx_i, \pi_i,\allowbreak \texttt{time}, \texttt{leaked}, *) \in \hotlist$, set $\texttt{leaked}$ to $1$ and send $(\texttt{sid}, \mathsf{LeakResult}, \adv, ((P_i^\hot, C_j, \hx_i, \pi_i, \texttt{time}))$ to $\adv$.
    
    \item On input $(\texttt{sid},\mathsf{Leak}, \adv, (P_i^\cold))$, retrieve the entry $(P_i^\cold, \ek_i, \dk_i, \texttt{leaked}, *, *, *) \in \coldlist$. Set $\texttt{leaked}$ to $1$ and send $(\texttt{sid}, \mathsf{LeakResult}, \adv, ((P_i^\cold, \dk_i))$ to $\adv$.
    \end{itemize}
    
    \textbf{Tamper}
    \begin{itemize}
    \item On input $(\texttt{sid},\mathsf{Tamper}, \adv, (P_i^\hot, C, f, g))$, where $f, g$ are functions: %\protect\footnotemark, 
    \begin{enumerate}
        \item Retrieve $(P_i^\hot, C, \hx_i, \pi_i, \texttt{time}, \texttt{leaked}, \texttt{tampered}) \in \hotlist$ for the maximum value of $\texttt{time}$. Let $\hx_i' := f(\hx_i)$ and $\pi_i' := g(\pi_i)$. 
        \item If $\hx_i', \pi_i' \neq \perp$, let $b:=1$ and update the entry to $(C, P_i^\hot, \hx_i', \pi_i', \texttt{time}, \texttt{leaked}, \texttt{tampered}=1)$. Otherwise let $b:=0$. 
        \item Send $(\texttt{sid}, \mathsf{TamperDone}, \adv, (C, P_i^\hot, b))$ to $\adv$.
    \end{enumerate}
    
    \item On input $(\texttt{sid},\mathsf{Tamper}, \adv, (P_i^\cold, f))$, where $f$ is a function:  
    \begin{enumerate}
        \item Retrieve $(P_i^\cold, \ek_i, \dk_i, \texttt{leaked}, \texttt{tampered}, \texttt{corrupt}, \texttt{allowc}) \in \coldlist$. Let $\dk_i' := f(\dk_i)$.
        \item If $\dk_i' \neq \perp$, let $b:=1$ and update the entry to $(P_i^\cold, \ek_i, \dk_i', \texttt{leaked}, \texttt{tampered}=1, \texttt{corrupt}, \texttt{allowc})$. Otherwise let $b:=0$. 
        \item Send $(\texttt{sid}, \mathsf{TamperDone}, \adv, (P_i^\cold, b))$ to $\adv$.
    \end{enumerate}
    % \noemi{Where are we using the \texttt{unleaked} bit? I guess it should be in an extra interface where we output a valid (full or partial, depending on how we want to define it) signature iff the (threshold of) party/ies has been met (adding up leaked + requested) --- or something like that}
    
    % \textbf{Partial Signatures}
    % \noemi{$\adv$ can request only $\cSign$ or only $\hSign$ output for the corresponding party of a corrupt hot/cold}
    
    % \begin{itemize}
    % \item On input $(\texttt{sid},\mathsf{CSign}, \adv, (y, m))$,
    % % proceed as follows: 
    % % if $\exists t, \ek_i, \dk_i $ for the largest $t$ such that $(P_i^\cold, \ek_i, \dk_i, t,  *) \in \coldlist$ and $\ek_i = g_2^{\dk_i}$ 
    % retrieve $(C, \mathcal{I}, *, y, *) \in \clientlist$ and $(*, P_i^\hot, P_i^\cold) \in \mathcal{I}$. Then retrieve $(P_i^\hot, C, *, *, \texttt{time}, \texttt{leaked}, *, \texttt{corrupted}) \in \hotlist$ for the maximum value of $\texttt{time}$. If $(\texttt{leaked} \lor \texttt{corrupted}) \neq 1$, set $\sigma_i^\cold := \perp$.
    % Otherwise, retrieve $(P_i^\cold, *, *, \dk_i, *) \in \coldlist$ and compute $\sigma_i^\cold := \blshash(m)^{\lhlhash({y^{\dk_i}})}$. Output $(\texttt{sid}, \mathsf{CSignResult}, \adv, (y, m, \sigma_i^\cold))$.
    % %to $\adv$, else output $\bot$.
    
    % \item On input $(\texttt{sid},\mathsf{HSign}, \adv, (C, m))$,
    % % if $\exists t,y$ for the largest $t$ such that $(y, \hx_i, \aux_i, t, *) \in \hotlist_i$ and $\pred(y,\hx_i, \aux_i) = 1$ \todo{Define \pred.} 
    % retrieve $(C, \mathcal{I}, *, *, *) \in \clientlist$ and $(*, P_i^\hot, P_i^\cold) \in \mathcal{I}$. Then retrieve $(P_i^\cold, *, *, \texttt{leaked}, *, \texttt{corrupted}) \in \coldlist$. If $(\texttt{leaked} \lor \texttt{corrupted}) \neq 1$, set $\sigma_i^\hot := \perp$.
    % Otherwise, retrieve $(C, P_i^\hot, \hx_i, *, \texttt{time}, *) \in \hotlist$ for the maximum value of $\texttt{time}$
    % and compute $\sigma_i^\hot := \blshash(m)^{\hx_i}$. Output $(\texttt{sid}, \mathsf{HSignResult}, \adv, (C, m, \sigma_i^\hot))$.
    % % send $(\texttt{sid}, \mathsf{HSignDone}, P_i^\hot, C, m, \sigma_i^\hot)$ to $\adv$.
    % % ; if no such entry exists instead output $\bot$.
    % % \item On input $(\texttt{sid}, \mathsf{HSignDeliver}, P_i^\hot, C, m)$ from $\adv$, output the corresponding \noemi{not formally specified} $(\texttt{sid}, \mathsf{HSignResult}, P_i^\hot, C, m, \sigma_i^\hot)$.
    \end{itemize} 
    
    \textbf{Corrupt}
    \begin{itemize}
        \item On input $(\sid,\mathsf{Corrupt}, \adv, (P_i^\hot))$, retrieve $(P_i^\hot, \texttt{corrupted}, \texttt{allowc}) \in \hotstates$ and check that $\texttt{allowc}=1$. If so, set $\texttt{corrupted}$ to 1. ($\adv$ receives $P_i^\hot$'s tapes and $\env$ is notified.)
        \item On input $(\sid,\mathsf{Corrupt}, \adv, (P_i^\cold))$, retrieve $(P_i^\cold, *, *, *, *, \texttt{corrupted}, \texttt{allowc}) \in \coldlist$ and check if $\texttt{allowc}=1$. If so, set $\texttt{corrupted}$ to 1. ($\adv$ receives $P_i^\cold$'s tapes and $\env$ is notified.)
    \end{itemize}
    }}
        \caption{The BLS \hcwl functionality $\FSign$ (adversarial interfaces).}
        \label{fig:FSign3}
    \end{figure*}
    % \footnotetext{If the result of applying $f$ or $f'$ is not in the correct domain, we assume that the ideal functionality rejects the tampering and continues with the previous values. \noemi{can also say we define the output as $\perp$ in that case.}}

% We do not define interfaces for verification since it is an algorithm with public inputs and can therefore be run at any time by any party.

\begin{theorem}[security]\label{thm:sec}
Our \hcwl construction (\Cref{sec:construction}) UC-realizes $\FSign$ in the $\Fs,\Fpk$-hybrid model.
\end{theorem}

\begin{proof}
Let $\adv$ be the adversary interacting with the parties (namely, $P_1^\hot, P_1^\cold,\allowbreak \dots, P_n^\hot, P_n^\cold, C$) running the protocol presented in \Cref{sec:construction}. We will construct a simulator $\Sim$ running in the ideal world against $\FSign$ so that no environment $\env$ can distinguish an execution of the ideal-world interaction from the real protocol. $\Sim$ will interact with $\FSign$, $\env$, and invoke a copy of $\adv$ to run a simulated interaction of the protocol (we call this simulated interaction between $\adv$, $\env$, and the parties the \emph{internal interaction} to distinguish it from the \emph{external interaction} between $\Sim$, $\env$, and $\FSign$).

Each protocol/algorithm begins with a party receiving some input. In the ideal world, this input is received by some dummy party, who immediately copies it to its outgoing communication tape where $\Sim$ can read it (public header only) and potentially deliver it to the ideal functionality $\FSign$. (Recall that for simplicity we assume authenticated communication, so $\Sim$ cannot modify the messages it delivers to/from $\FSign$.) To complete the protocol, $\Sim$ will deliver the response of $\FSign$ to the dummy party, who copies it to its output tape (which is visible to $\env$).

\renewcommand{\labelitemi}{\textendash}
\paragraph{Message delivery} $\Sim$ waits to deliver any messages until $\adv$ delivers the corresponding message in the internal interaction.

\paragraph{Corruptions} Whenever $\adv$ corrupts a party $P_i^\cold$ or $P_i^\hot$ in the internal execution, $\Sim$ corrupts the corresponding dummy party (via the $\mathsf{Corrupt}$ interface).

\paragraph{Leak and Tamper}
Whenever $\adv$ leaks or tampers with the inputs of a party in the internal execution, $\Sim$ uses the corresponding interface of $\FSign$ to learn/change the same information (it can use any functions $f,g$ for tampering). %It keeps track of any tampering/leaks in the lists $\coldlist', \hotlist'$.

\paragraph{Cold Registration} $\Sim$ will deliver the message $(\texttt{sid}, \mathsf{ColdRegister}, P_i^\cold)$ from $\tilde{P}_i^\cold$ to $\FSign$ once $\adv$ delivers $(\texttt{sid}, \mathsf{PKSetup}, P_i^\cold)$ to $\Fpk$ in the internal interaction.
\begin{itemize}
    \item If $\tilde{P_i}^\cold$ is corrupt, $\Sim$ records $(P_i^\cold, \ek_i^*, \texttt{corrupt} = 1)$ in a local database $\coldlist'$, where $\ek_i^*$ is the value sent by $\adv$ on (the corrupted) $\tilde{P_i}^\cold$'s behalf.
    \item Otherwise, if $\tilde{P_i}^\cold$ is honest, it stores $(P_i^\cold, \ek_i, \texttt{corrupt} = 0) \in \coldlist'$, where $\ek_i$ is the value from $\FSign$'s response.
\end{itemize}
It delivers the response to $\tilde{P}_i^\cold$ once $\adv$ delivers the result of $\Fpk$ in the internal interaction. 

\paragraph{Client Registration and Share Refreshes}
Recall that $C$ is out of scope for corruptions. Thus, in any case $\Sim$ will deliver the message $(\texttt{sid}, \mathsf{ClientRegister}, C,\allowbreak (t, \mathcal{I}))$ on $\tilde{C}$'s outgoing communication tape to $\FSign$ once $\adv$ delivers $(\texttt{sid}, \mathsf{SSSetup}, C, (t, \{P_i^\hot\}_{i \in [n]}, *)$ from $C$ to $\Fs$ in the internal interaction. $\FSign$ will send $(\texttt{sid}, \mathsf{ClientRegistered}, C, (\vk))$ to $\tilde{C}$ and $(\texttt{sid},\allowbreak \mathsf{ClientRegistered}, P_i^\hot, (C, b_i))$ to $\tilde{P_i}^\hot$. $\Sim$ delivers these messages to $\tilde{C}$ and each \emph{honest} party $\tilde{P_i}^\hot$, respectively, once the corresponding response by $\Fs$ is delivered to them in the internal interaction.
For any corrupt $\tilde{P_i}^\hot$, $\Sim$ instead outputs on its behalf the same bit $b_i$ as $\adv$ does in the internal interaction. The simulation for share refreshes works the same way.

% in main body for both short and full version
\paragraph{Signing} 
We will use the fact that partial BLS signatures $\sigma_i$ can be verified with respect to the partial verification key $\vk_i$. Let $\mathsf{P}\vrfy(\vk_i, m, \sigma_i)$ be the partial verification algorithm, which in the case of BLS consists of checking that $(g_2, \vk_i, \blshash(m), \sigma_i)$ is a co-Diffie-Hellman tuple (see \Cref{subsec:threshold-bls}).
On input $(\texttt{sid}, \mathsf{TSign}, P_i^\hot, (C, m))$, the simulator first delivers the message to $\FSign$ once $\adv$ delivers the corresponding request from $C$ to $P_i^\hot$ in the internal interaction. Then it retrieves the identity of the corresponding $P_i^\cold$ and behaves as follows:
\begin{itemize}
    \item If $P_i^\hot, P_i^\cold$ are both honest, $\Sim$ immediately delivers $\FSign$'s output $(\sid,\allowbreak \mathsf{TSignResult}, P_i^\hot, (C, m, \sigma_i))$.
    \item If $P_i^\hot$ is honest and $P_i^\cold$ is corrupt, $\Sim$ will receive a $\mathsf{CSignRequest}$ from $\FSign$ to which it must respond with a bit $b$. It looks at the values $\sigma_i^\hot$ and ${\sigma_i^\cold}^*$ in the internal interaction (the latter is output by $\adv$ on behalf of the corrupt $P_i^\cold$) and responds with $b := \mathsf{P}\vrfy(\vk_i, m, \sigma_i^\hot/{\sigma_i^\cold}^*)$, where $\vk_i, m$ are the $i$th partial verification key and message requested in the \emph{internal} execution (all known to $\Sim$). As before, it delivers $\FSign$'s output to $\tilde{C}$ immediately.
    \item If $P_i^\hot$ is corrupt and $P_i^\cold$ is honest, $\Sim$ will receive an $\mathsf{HSignRequest}$ from $\FSign$ to which it must again respond with a bit $b$. Similarly, it now looks at the values ${\sigma_i^\hot}^*$ (output by $\adv$) and $\sigma_i^\cold$ in the internal execution and responds with $b := \mathsf{P}\vrfy(\vk_i, m, {\sigma_i^\hot}^*/\sigma_i^\cold)$. Again it immediately delivers $\FSign$'s output to $\tilde{C}$.
    \item Finally, if both $P_i^\hot$ and $P_i^\cold$ are corrupt, $\Sim$ checks if $\mathsf{P}\vrfy(\vk_i, m, {\sigma_i^\hot}^*/{\sigma_i^\cold}^*) = 1$ in the internal execution. If so, it responds $1$ to both $\mathsf{CSignRequest}$ and $\mathsf{HSignRequest}$; otherwise (w.l.o.g.) it sends $0$ to both. Then it delivers $\FSign$'s output  to $\tilde{C}$.
\end{itemize}

\paragraph{Proofs of Remembrance}
On input $(\sid, \mathsf{CProof}, C, (P_i^\cold))$, the simulator delivers the message to $\FSign$ once $\adv$ delivers the corresponding request from $C$ to $P_i^\cold$ in the internal interaction. 
\begin{itemize}
    \item If $\tilde{P}_i^\cold$ is honest, $\Sim$ delivers $(\texttt{sid}, \mathsf{CProofResult}, C,\allowbreak (P_i^\cold, b))$ to $\tilde{C}$ once $\adv$ delivers the output of $\Fpk$ to $C$ in the internal execution. 
    \item On the other hand, if $\tilde{P}_i^\cold$ is corrupted, $\Sim$ will receive a message from $\FSign$ requesting a bit $b^*$. It retrieves the party's $\ek_i$ from $\coldlist'$ and the proof $\pi_i^\cold$ computed by $\adv$ in the internal execution and sends back $b^* := \Pi_{\sf DL}.\vrfy((\ek_i,\allowbreak g_2), {\pi_i^\cold})$. Finally it delivers $\FSign$'s output $(\texttt{sid}, \mathsf{CProofResult},\allowbreak C, (P_i^\cold, b))$ to $\tilde{C}$ once $\adv$ delivers the message from $\Fpk$ to $C$ in the internal execution.
\end{itemize}
On input $(\texttt{sid}, \mathsf{HProof}, P_i^\hot, (C))$, $\Sim$ acts in the same way as $\mathsf{CProof}$, except in the corrupted case it sets $b^* := \Pi_{\sf EKS}.\vrfy(\com, {\pi_i^\hot})$, where both $\com, \pi_i^\cold$ are the party's values in the internal execution.
% If $\tilde{P}_i^\hot$ is not corrupted, $\Sim$ delivers the output $(\texttt{sid}, \mathsf{HProofResult}, C, (P_i^\hot, b))$ to $\tilde{C}$ once $\adv$ delivers the output of $\Fpk$ to $C$ in the internal execution.
% On the other hand, if $\tilde{P}_i^\hot$ is corrupted, $\Sim$ will receive a message from $\FSign$ requesting a bit $b^*$. It retrieves the value of $\com$ from the internal execution and sends back $b^* := \Pi_{\sf EKS}.\verify(\com, {\pi_i^\hot})$. 
% Finally it delivers $\FSign$'s output $(\texttt{sid}, \mathsf{CProofResult}, C, (P_i^\cold, b))$ to $\tilde{C}$ once $\adv$ delivers the message from $\Fpk$ to $C$ in the internal execution.

It is straightforward to see that the simulated registration and share refreshes are identical to a real-world execution.
The simulation of the signing protocol is computationally indistinguishable from the real-world execution by the correctness and security of threshold BLS signatures, which guarantees that $\Sim$ sends $b=1$ iff ${\sigma_i^\cold}^* = \blshash(m)^{\lhlhash(\vk^{\dk_i})}$, resp. ${\sigma_i^\hot}^* = \blshash(m)^{\hx_i}$. 
Similarly, the indistinguishability of simulating proofs of remembrance follows from the correctness and soundness of $\Pi_{\sf DL}$ and $\Pi_{\sf EKS}$.
\end{proof}

Finally, notice that the signing interface of $\FSign$ is identical to the functionality offered by threshold BLS, except that $\adv$ additionally learns either $\sigma_i^\hot$ or $\sigma_i^\cold$ (in step 3) or a uniform value $\blshash(m)^r$ (in step 4). Clearly the uniform value is independent of any private information can be simulated perfectly as a uniform $\GG_2$ element. As for step 3, the simulator can again send a uniform $\GG_2$ element, now in place of $\sigma_i^\hot$ (alternatively, $\sigma_i^\cold$), and the simulation is statistically indistinguishable by \Cref{lemma:LHL}.
