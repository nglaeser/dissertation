\section{Additional Preliminaries}

\paragraph{A note on our ideal functionalities} We model security in the UC framework (\Cref{sec:uc}). As in some previous work~\cite{CCS:LinNof18,C:Katz24}, our UC functionalities in this chapter contain some cryptographic operations specific to our construction. While this formulation is less general, it renders the analysis more straightforward and suffices for our purposes.

\subsection{Leftover Hash Lemma}\label{sec:lhl}  

We use the presentation of the leftover hash lemma (LHL)~\cite{FOCS:ImpZuc89} from\break \cite{EC:AMPR19}.\footnote{We specifically use the improved version from the Journal of Cryptology version of this paper.}  Let $\left(\mathcal{X},\oplus\right)$ be a finite group of size $\left|\mathcal{X}\right|$, and let $n$ be a positive integer. For any fixed $2n$-vector of group elements $\mathbf{x} = \left\lbrace x_{j,b} \right\rbrace_{j\in \left[ n \right],b\in\left\lbrace 0,1 \right\rbrace} \in\mathcal{X}^{2n}$, denote by $\mathcal{S}_{\mathbf{x}}$ the following distribution:
\begin{equation}
	\mathcal{S}_{\mathbf{x}} = \Big\{\bigoplus_{j\in[n]}x_{j,r_j} : \left(r_1,\cdots,r_n\right)\gets\left\lbrace 0,1 \right\rbrace^n\Big\}.\nonumber
\end{equation}
Also, let $\mathcal{U}_{\mathcal{X}}$ denote the uniform distribution over $\mathcal{X}$, and let $\Delta \left(\mathcal{D}_1,\mathcal{D}_2\right)$ denote the statistical distance between the distributions $\mathcal{D}_1$ and $\mathcal{D}_2$. We will use the following special case of leftover hash lemma~\cite{FOCS:ImpZuc89}. The proof can be found in the JoC version of~\cite{EC:AMPR19}.

\begin{lemma}{\textnormal{(Leftover Hash Lemma.)}}
	\label{lemma:LHL}	
	Let $\left(\mathcal{X},\oplus\right)$ be a finite group, and let $\mathcal{S}_{\mathbf{x}}$ and $\mathcal{U}_{\mathcal{X}}$ be two distributions over $\mathcal{X}$ as defined above. For any (large enough) positive integer $n$, it holds that
	\begin{equation}
		\Pr_{\mathbf{x}\gets\mathcal{X}^{2n}}\left[\Delta\left(\mathcal{S}_{\mathbf{x}},\mathcal{U}_{\mathcal{X}}\right)> \sqrt[4]{\frac{\sizeof{\mathcal{X}}}{2^n}}  \right]\leq \sqrt[4]{\frac{\sizeof{\mathcal{X}}}{2^n}} \nonumber.
	\end{equation}
	In particular, for any $n > \log(\sizeof{\mathcal{X}}) + \omega(\log(\lambda))$, if $\mathbf{x}$ is sampled uniformly then with overwhelming probability the statistical distance between the two distributions is negligible.
\end{lemma}