\subsection{Model}\label{sec:hc_model}

\begin{figure*}[tbh]
    \centering
    \begin{tikzpicture}[par/.style={sloped,fill=white,inner sep=-.4ex}]
    \tikzstyle{box} = [rectangle, draw]
    \tikzstyle{charrow} = [-Stealth,dashed]
    % \tikzstyle{hharrow} = [Stealth-Stealth]
    \tikzstyle{hharrow} = [-Stealth]
        %%% P1 %%%
        \node[box] (p1cold) {$\sk_1^\cold$};
        \node[box,right=10em] (p1hot) at (p1cold) {$\sk_1^\hot$};
        % \draw[->,thick] (p1cold) -- node[par]{||} (p1hot);
        \draw[charrow] (p1cold) -- node[yshift=1em]{$\sigma_1^\cold$} (p1hot);
        % \node[par] at (p1) {||};
        %%% labels
        \node[below of=p1cold,yshift=1em]{$P_1^\cold$};
        \node[below of=p1hot,yshift=1em] {$P_1^\hot$};
        %%% P2 %%%
        \node[box,right=10em] (p2hot) at (p1hot) {$\sk_2^\hot$};
        \node[box,right=10em] (p2cold) at (p2hot) {$\sk_2^\cold$};
        \draw[charrow] (p2cold) -- node[yshift=1em]{$\sigma_2^\cold$} (p2hot);
        %%% labels
        \node[below of=p2cold,yshift=1em] {$P_2^\cold$};
        \node[below of=p2hot,yshift=1em] {$P_2^\hot$};
        %%% P3
        \node[right=5em] (midpoint) at (p1hot) {};
        \node[box,above=7.5em] (p3hot) at (midpoint) {$\sk_3^\hot$};
        \node[box,left=10em] (p3cold) at (p3hot) {$\sk_3^\cold$};
        \draw[charrow] (p3cold) -- node[yshift=1em]{$\sigma_3^\cold$} (p3hot);
        %%% labels
        \node[above of=p3cold,yshift=-1em] {$P_3^\cold$};
        \node[above of=p3hot,yshift=-1em] {$P_3^\hot$};
        %%% hot connections
        \begin{scope}[transform canvas={yshift=.5em}]
        \draw[hharrow] (p1hot) -- node[above]{$\sigma_1$} (p2hot);
        \end{scope}
        \begin{scope}[transform canvas={yshift=-.5em}]
        \draw[hharrow] (p2hot) -- node[below]{$\sigma_2$} (p1hot);
        \end{scope}
        \begin{scope}[transform canvas={xshift=-.5em}]
        \draw[hharrow] (p2hot) -- node[left]{$\sigma_2$}(p3hot);
        \draw[hharrow] (p1hot) -- node[left]{$\sigma_1$}(p3hot);
        \end{scope}
        \begin{scope}[transform canvas={xshift=.5em}]
        \draw[hharrow] (p3hot) -- node[right]{$\sigma_3$}(p2hot);
        \draw[hharrow] (p3hot) -- node[right]{$\sigma_3$}(p1hot);
        \end{scope}
    \end{tikzpicture}
    \caption{A \hcwl with $n=3$. Given a message $m$, each $P_i^\cold$ uses its cold share $\sk_i^\cold$ to compute a cold partial signature $\sigma_i^\cold$ on $m$ and sends it to its hot storage. $P_i^\hot$ uses $m$ and its hot share $\sk_i^\hot$ to compute $\sigma_i^\hot$, which it combines with $\sigma_i^\cold$ to get $\sigma_i$. The hot parties broadcast their partial signatures to each other to reconstruct the full signature $\sigma$ on $m$. The dashed lines between each cold and hot storage represent the authenticated channel between them which is only active at signing time. The hot parties are always online and connected, represented by the solid lines.}
    \label{fig:hc_model}
\end{figure*}

We will use the superscripts $\hot$ and $\cold$, respectively, to denote the hot and cold components of some value, and a subscript $i$ to mean that the value corresponds to the $i$th party. For example, the $i$th party's cold signature is written as $\sigma_i^\cold$. We use the words ``user'' and ``client'' interchangeably to refer to the wallet owner. 

Let $I_1, \ldots, I_n$ be parties (each representing a custodian institution) who will store shares of some user's signing key $\sk$. Each institution $I_i$ controls two parties: a hot wallet ($P_i^\hot$) and a cold wallet ($P_i^\cold$). Thus we represent an institution by the tuple $I_i = (P_i^\hot, P_i^\cold)$. 
As in the standard threshold wallet setting, the hot parties are connected to each other via authenticated (but not private) channels. We also assume the parties can send broadcast messages (e.g. by posting to a blockchain).  %\noemi{we also assume a broadcast channel between the hot parties?}
In contrast, each cold party is connected by an authenticated channel only to its corresponding hot party. This channel is only active during the signing phase, further reducing the cold party's attack surface. (For example, the cold party could be a read-only USB device which is plugged into a PC (the hot party) only briefly to produce a signature.) 
An illustration of this model is given in \Cref{fig:hc_model} for $n=3$.

We assume $P_i^\hot$ has more storage space and computational power, while $P_i^\cold$ has limited storage (in particular, we want the space complexity to be independent of the number of users). 
Therefore, our protocol aims to minimize the storage and computation on the part of the cold party.