\section{Hot and Cold Key Shares}\label{sec:keyshares}

The core idea of our construction is for each hot party to store an encryption of the signing key share $\sk_i$ given to the custodian, with the corresponding decryption key kept in cold storage.
That is, each custodian will generate an encryption keypair $(\ek_i, \dk_i)$ and store the hot key share $\sk_i^\hot := \enc(\ek_i, \sk_i)$ in the hot storage and $\sk_i^\cold := \dk_i$ in the cold storage.
We want to enable threshold signing of a message $m$ by allowing the hot and cold parties to derive signature shares $\sigma_i^\hot, \sigma_i^\cold$ for $m$ from their secret material $\sk_i^\hot, \sk_i^\cold$ (the secret key and ciphertext, respectively). Together, the signature shares can be used to recover a partial BLS signature $\sigma_i$ of $m$ under $\sk_i$.

Our construction uses a modified ElGamal~\cite{C:ElGamal84} encryption scheme to encrypt key shares. The homomorphic properties of the scheme will allow decryption of the key share ``in the exponent'' to obtain a message-specific partial BLS signature.
Recall that the ElGamal ciphertext for a message $m \in \GG$ is computed as $(g^r, \ek^r \cdot m) \in \GG^2$ where $\GG$ is a prime order group and $g \in \GG$ is a generator. In our case, however, we will be encrypting secret key shares, which are field elements (specifically, elements of $\ZZ_p$).

Adapting ElGamal to plaintexts $m \in \ZZ_p$, we will compute a ciphertext as $(g^r, \lhlhash(\ek^r) + m) \in \GG \times \ZZ_p$, where $\lhlhash: \GG \to \ZZ_p$.
Now $m$ is masked with the uniformly random output of $\lhlhash$ and can be unblinded using $\dk$ by recomputing $\lhlhash(\ek^r) = \lhlhash((g^r)^\dk)$.
Like the original ElGamal encryption, this construction is malleable (in this case, additively rather than multiplicatively), which allows ``shifting'' of the message $m$ in a ciphertext by an additive factor (made explicit via the $\sf Shift$ algorithm). We will use this property to enable proactive refresh of the hot (encrypted) key shares.

\begin{construction}[additive ElGamal]\label{con:additive_elgamal}
Let $\GG$ be a DLog-hard group of order $p$ with generator $g$ and $\lhlhash: \GG \rightarrow \ZZ_p$. 
\begin{itemize}
\item \underline{$(\ek, \dk) \gets \kgen(1^\lambda)$:}
Sample $\dk \sample \ZZ_p$. Set $\ek := g^{\dk}$ and output $(\ek, \dk)$.

\item \underline{$ct \gets \enc(\ek, m; r)$:}
Given an encryption key $\ek \in \GG$ and a message $m \in \ZZ_p$, use randomness $r \in \ZZ_p$ to compute the ciphertext $ct := (g^r, m + \lhlhash(\ek^r))$.

\item \underline{$m' \gets \dec(\dk, ct)$:}
Given a secret key $\dk \in \ZZ_p$ and a ciphertext $ct \in \GG \times\ZZ_p$, parse $ct$ as $(ct_0, ct_1)$ and return $ct_1 - \lhlhash(ct_0^{\dk})$.

\item \underline{$ct' \gets \mathsf{Shift}(ct, \delta)$:}
Given a ciphertext $ct \in \GG \times \ZZ_p$ and a shift $\delta \in \ZZ_p$, parse $ct$ as $(ct_0, ct_1)$ and output the shifted ciphertext $ct' := (ct_0, ct_1 + \delta)$.
\end{itemize}
\end{construction}

\subsection{Additive Secret Sharing from Additive ElGamal}\label{sec:enc-lhl}

The $\enc$ algorithm in \Cref{con:additive_elgamal} takes an explicit randomness input $r$. In the context of our wallet construction, instead of sampling fresh randomness $r \in \ZZ_p$ for each hot party's ciphertext (hot key share), we will use a value based on the user secret being stored. In particular, when storing a signing keypair $(\vk := g^{\sk}, \sk)$, we will use $\sk$ as the encryption randomness.\footnote{This means we can leave out the redundant first element which now equals $\vk$, and each ciphertext will consist of a single group element.} Thus, $P_i^\hot$'s secret share uses the mask $\lhlhash(\ek_i^{\sk}) = \lhlhash(\vk^{\dk_i})$, which allows the corresponding cold party $P_i^\cold$ to decrypt using only a client's verification key $\vk$ and without receiving or storing any additional per-client randomness. We will discuss how to generate the hot shares in \Cref{sec:bls-construction}.

Because the input to $\lhlhash$ is no longer uniformly random, we now need $\lhlhash$ to be a hash function where the Leftover Hash Lemma (see \Cref{sec:lhl}) holds on random inputs. In order for the output of $\lhlhash$ to have sufficient entropy to mask $m$, this requires two group elements as input. Therefore, our construction we will use $\ek := (g^{\dk_1}, g^{\dk_2})$ and $ct := m + \lhlhash(\ek_1^\sk, \ek_2^\sk)$.

Although any function $\lhlhash$ which meets the above requirements suffices, it is desirable to find a very efficient construction since $\lhlhash$ will have to be computed by the cold party at signing time. One such $\lhlhash$ is a random subset sum.
In more detail, $\lhlhash$ first represents its inputs $x_1, x_2 \in \GG$ as $\ell$-bit vectors $\vec{x}_1, \vec{x}_2 \in \{0,1\}^\ell$, where $\ell = \log{p}$. Let $\vec{r}_1, \vec{r}_2 \in \ZZ_p^{2 \ell}$ be (public) uniform vectors with $\vec{r}_k = (\vec{r}_{k,0}, \vec{r}_{k,1})$ for $k=1,2$. We will use bracket notation to index into the vector, i.e., $\vec{r}_{1,0}[i]$ is the $i$th element of $\vec{r}_{1,0}$.
Let $\lhlhash(x_1, x_2) := \lhlhash'(\vec{r}_1,\vec{x}_1) + \lhlhash'(\vec{r}_2,\vec{x}_2)$, where $\lhlhash':\ZZ_p^{2\ell} \times \{0,1\}^\ell \to \ZZ_p$ is the subset sum function
\[
\lhlhash'(\vec{r} := (\vec{r}_0, \vec{r}_1),\vec{x}) := \sum_{b_i \in \vec{x}} \vec{r}_{b_i}[i]
\]
When we want to be specific about the randomness used in $\lhlhash$, we write $\lhlhash(x_1, x_2;\allowbreak \vec{k})$ for $\vec{k} \in \ZZ_p^{4\ell}$. By \Cref{lemma:LHL} in \Cref{sec:lhl}, the output of $\lhlhash$ is statistically indistinguishable from uniform.