There are three entities in a naysayer proof system. We assume that all parties can read and write to a public bulletin board (e.g. a blockchain).
\begin{description}
    \item[Prover] The prover posts a proof $\pi$ to the bulletin board claiming $(x,\witness)\in\mathcal{R}$. 
    \item[Verifier] The verifier does not directly verify the validity of $\pi$, rather, it allows everyone to naysay in a pre-defined time window of duration $T_{\nay}$.
    Optimistically, if no one naysays $\pi$ within time $T_{\nay}$, the verifier accepts it. In the pessimistic case, a party (or multiple parties) naysay the validity of $\pi$ by posting proof(s) $\pi_{\nay}$. The verifier checks the validity of each $\pi_{\nay}$, and if any of them pass, it rejects the original proof $\pi$.
    \item[Naysayer] If $\vrfy(\crs, x,\pi)=0$, then the naysayer posts a naysayer proof $\pi_{\nay}$ to the public bulletin board before $T_{\nay}$ time elapses.
\end{description}

Note that we need to assume a synchronous communication model as we cannot have naysayer proofs in partial synchrony or asynchrony: if the adversary can arbitrarily delay the posting of naysayer proofs, then we cannot enforce soundness of the underlying proofs. Note that synchrony is already assumed by most of the deployed consensus algorithms, e.g., Nakamoto consensus~\cite{Nakamoto08}. Furthermore, we assume that the public bulletin board offers censorship resistance for the writers of the bulletin board. Finally, we assume that there is \emph{at least one honest party} who is ready to create and submit naysayer proofs for invalid proofs.