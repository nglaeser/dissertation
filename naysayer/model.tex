\section{Model}\label{sec:naysayer_model}

There are three entities in a naysayer proof system. We assume that all parties can read and write to a public bulletin board (e.g., a blockchain). Fix a function $f: \mathcal{X} \times \mathcal{W} \to \mathcal{Y}$ and let $\Lang_f$ be the language $\{(x, y): \exists \witness \suchthat y = f(x, \witness)\}$. Let $\Rel_f = \{((x, y), \witness)\}$ be the corresponding relation. We assume $f,x$ are known by all parties. When $f : \mathcal{Y} \times \mathcal{W} \to \mathcal{Y}$ is a state transition function with $y' = f(y, w)$, this corresponds to the rollup scenarios described above.
\begin{description}
    \item[Prover] The prover posts $y$ and a proof $\pi$ to the bulletin board claiming $(x,y) \in \Lang_f$. 
    \item[Verifier] The verifier does not directly verify the validity of $y$ or $\pi$, rather, it waits for time $T_{\nay}$.
    If no one naysays $(y, \pi)$ within that time, the verifier accepts $y$. In the pessimistic case, a party (or multiple parties) naysay the validity of $\pi$ by posting proof(s) $\pi_{\nay}$. The verifier checks the validity of each $\pi_{\nay}$, and if any of them pass, it rejects $y$.
    \item[Naysayer] If $\vrfy(\crs, (x,y),\pi)=0$, then the naysayer posts a naysayer proof $\pi_{\nay}$ to the public bulletin board before $T_{\nay}$ time elapses.
\end{description}

Note that, due to the optimistic paradigm, we must assume a synchronous communication model: in partial synchrony or asynchrony, the adversary can arbitrarily delay the posting of naysayer proofs, and one cannot enforce soundness of the underlying proofs. 
% Synchrony is already assumed by most deployed consensus algorithms, e.g., Nakamoto consensus~\cite{Nakamoto08}. \noemi{Ethereum's consensus uses Gasper~\cite{ARXIV:BHKPQR20}, which requires only asynchrony, or partial synchrony for probabilistic liveness (see \S2.6 therein).} \noemi{note we require synchrony, not liveness, which is weaker since it can be (probabilistically) achieved under asynchrony}
Furthermore, we assume that the public bulletin board offers censorship-resistance, i.e., anyone who wishes to write to it can do so successfully within a certain time bound. Finally, we assume that there is \emph{at least one honest party} who will submit a naysayer proof for any invalid $\pi$.