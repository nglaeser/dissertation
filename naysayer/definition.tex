\subsection{Formal Definitions}\label{sec:naysayer_def}

We formally introduce the notion of a \emph{naysayer proof}, with the following syntax:

\begin{definition}[Naysayer proof] \label{def:naysayer_proof}
Given a non-interactive proof system $\Pi = (\setup, \prove, \vrfy)$ for some NP relation $\Rel$, the corresponding naysayer proof system $\Pi_{\nay}$ is a tuple of PPT algorithms $({\sf N}\setup, {\sf N}\prove, \naysay, \vrfynay)$ defined as follows:
    \begin{description}
        \item[${\sf N}\setup(\secparam, 1^\naysecparam) \rightarrow (\crs,\crs_{\nay})$:] Given security parameters $\secparam$ and $1^\naysecparam$ for $\Pi$ and $\Pi_{\nay}$, respectively, output common reference strings $\crs$ and $\crs_{\nay}$. This algorithm might use private randomness.
        \item[${\sf N}\prove(\crs, x, w) \to \pi$:] Given a statement $x$ and witness $w$ such that $(x, w) \in \Rel$, compute $\pi' \gets \Pi.\prove(\crs, x,w)$ and $\com$ a commitment to the evaluation trace of $\Pi.\vrfy(\crs, x, \pi')$, output $\pi := (\com, \pi')$.
        \item[$\naysay(\crs_{\nay}, (x,\pi), \aux_{\nay}) \rightarrow \pi_{\nay}$:] Given a statement $x$, $\pi = (\com, \pi')$ where $\pi'$ is a corresponding (potentially invalid) proof in proof system $\Pi$, and auxiliary information $\aux_{\nay}$, output a naysayer proof $\pi_{\nay}$ disputing $\pi'$. 
        \item[$\vrfynay(\crs_{\nay}, (x,\pi), \pi_{\nay}) \rightarrow \{0, \perp\}$:] Given a statement-proof pair $(x, \pi=(\com,\allowbreak\pi'))$ and a naysayer proof $\pi_{\nay}$ disputing $\pi$, output a bit indicating whether evidence against $\pi$ is sufficient to reject $\pi$ (0) or inconclusive ($\perp$).
    \end{description}
\end{definition}

%Naysayer proofs are most interesting if $\vrfynay$ takes less time than directly verifying the proof $\pi$, i.e., running the $\vrfy$ algorithm. 
A trivial naysayer proof system always exists in which $\pi_\nay = \varnothing$, $\pi = (\perp, \pi')$, and $\vrfynay$ simply runs the original verification procedure.
We say a proof system $\Pi$ is \emph{efficiently naysayable} if there exists a corresponding naysayer proof system $\Pi_{\nay}$ such that $\vrfynay$ is asymptotically faster than $\vrfy$. If $\vrfynay$ is only concretely faster than $\vrfy$, we say $\Pi_{\nay}$ is a \emph{weakly efficient} naysayer proof. Note that some proof systems already have constant proof size and verification time~\cite{EC:Groth16,C:Schnorr89} and therefore can, at best, admit only weakly efficient naysayer proofs. 
Moreover, if $\aux_\nay = \varnothing$, we say $\Pi_{\nay}$ is a \emph{public} naysayer proof. 

\begin{definition}[Naysayer correctness]
    Given a proof system $\Pi$, a naysayer proof system $\Pi_{\nay}$ is \emph{correct} if, for all honestly generated $\crs, \crs_{\nay}$, all statements $x$, and all invalid proofs $\pi$, $\naysay$ outputs a valid naysayer proof $\pi_{\nay}$:
\begin{equation}
    \Pr\left[
        \vrfynay(\crs_{\nay}, (x,\pi), \pi_{\nay})=0 
        \middle| 
        \begin{array}{c}
            (\crs, \crs_{\nay}) \gets {\sf N}\setup(\secparam, 1^\naysecparam)~\land\\
            \vrfy(\crs, x,\pi)=0~\land\\
            \pi_{\nay} \gets \naysay(\crs_{\nay}, (x, \pi), \aux_{\nay})
        \end{array}
    \right] = 1.
\end{equation}
\end{definition}

\begin{definition}[Naysayer soundness] Given a proof system $\Pi$, a naysayer proof system $\Pi_{\nay}$ is sound if, for all PPT adversaries $\adv$ and for all $x, \aux$, honestly generated $\crs, \crs_\nay$, and correct proofs $\pi$, $\adv$ produces a verifying naysayer proof $\pi_{\nay}$ with at most negligible probability:
\begin{equation}
    \Pr\left[
        \vrfynay(\crs_{\nay}, (x,\pi), \pi_{\nay})=0 
        \middle| 
        \begin{array}{c}
            (\crs, \crs_{\nay}) \gets {\sf N}\setup(\secparam, 1^\naysecparam)~\land\\
            \vrfy(\crs, x,\pi)=1~\land\\
            \pi_{\nay} \gets \adv(\crs_{\nay}, (x, \pi), \aux_{\nay})
        \end{array}
    \right] \leq\mathsf{negl}(\naysecparam).
\end{equation}
\end{definition}

We distinguish between two types of naysayer proofs as follows.
\begin{description}
    \item \textbf{Type 1.} A prover of an NP-relation $\Rel_\Lang$ posts $(x,\pi)$ to the public bulletin board claiming that $x \in \Lang$. If the proof $\pi$ is invalid with respect to the statement $x$, i.e., $\vrfy(\crs, x,\pi)=0$, then naysayer provers convince the resource-constrained verifier by sending a $\pi_{\nay}$ that this is indeed the case, i.e., $\vrfynay(\crs_{\nay}, (x,\pi), \pi_{\nay})=0$. 
    \item \textbf{Type 2.} This family of naysayer proofs is even more efficient in the optimistic case, as the prover \emph{only sends the instance $x$} and no proofs at all, claiming without evidence that $(x,w) \in \Rel$. On the other hand, if the prover's assertion is incorrect, i.e., $(x,w) \notin \Rel$, then a naysayer prover provides the correct statement $x'$ such that $(x',w) \in \Rel$ 
    and a corresponding ``regular'' proof $\pi$ such that $\vrfy(\crs, x', \pi) = 1$. For example, in the case of rollups, the (public) witness $w$ is the set of transactions in the rollup, and the statement $x = (\mathsf{st}, \mathsf{st}')$ is the updated rollup state after applying the batch $w$. Therefore, an incorrect assertion represents an incorrect application of the update $w$. The correction $x'$ is the result of the proper application of $w$.
\end{description}
We conjecture that in most applications, in the worst case, type-2 naysayer proofs are more costly than type-1 naysayer proofs (both compute and storage).
It is an interesting open question which applications are more suited to type-1 or type-2 naysayer proofs considering both optimistic and pessimistic costs. To thoroughly model this question, one must take into account the verifier's compute cost, the (naysayer) proof storage costs, as well as the probability of the prover sending an invalid proof. We leave this problem to future work.
In the rest of this paper, we solely focus on type-1 naysayer proofs.

Finally, we show that for every NP language, there exists a logarithmic-size naysayer proof with constant verification time (i.e., a succinct naysayer proof). %We follow the blueprint of~\cite{FOCS:FeiLapSha90}.

\begin{theorem}\label{thm:naysayingNPCircuit}
    For every NP language $\Lang$ with relation $\Rel_\Lang$, 
    % assuming a binding commitment scheme with constant-size openings, 
    there exists a naysayer proof system $\Pi_{\nay}$ with logarithmic-sized proofs $\pi_{\nay}$ and constant-time verifier.
\end{theorem}
\begin{proof}
    For any NP language $\Lang$, there exists a non-interactive proof system $\Pi = (\setup, \prove, \vrfy)$ for $\Rel_\Lang$ where $\vrfy(\crs, \cdot, \cdot)$ can be represented as a boolean circuit $C$ of size $\mathsf{poly}(\lvert x \rvert)$~\cite{C:LapSha90}. Recall that for all $(x,w) \in \Rel_\Lang$ and $\crs \gets \Pi.\setup(\secparam)$, by correctness, we have that if $\pi' \gets \Pi.\prove(\crs, x, w)$ then $C(x,\pi') = 1$. Therefore, if there is some gate of $C$ for which the wire assignment is inconsistent, then the proof $\pi'$ is incorrect. To naysay, i.e., to show the incorrectness of $\pi'$, the naysayer simply provides the index of the inconsistent gate.
    Recall that the verifier has access to the wire assignments of $C(x,\pi')$ as part of $\pi$ (see \Cref{def:naysayer_proof}). 
    The verifier then checks whether the wire assignments of $C(x,\pi')$ are consistent with a correct evaluation of the gate, which is a constant-time operation assuming constant-time indexing into the wire assignments. 
    Furthermore, the naysayer proof consists only of the gate index, which is logarithmic in the circuit size, i.e., succinct.
\end{proof}

\begin{corollary}
    Every efficient proof system $\Pi$ (i.e., with a polynomial-time verification algorithm) has a succinct naysayer proof.
\end{corollary}
\begin{proof}
Given any proof system $\Pi$, one can represent the $\vrfy(\crs,\cdot,\cdot)$ algorithm as a circuit and apply the above theorem to obtain a succinct naysayer proof.
\end{proof}