\section{Formal Definitions}\label{sec:naysayer_def}

Next, we introduce a formal definition and syntax for naysayer proofs. A naysayer proof system $\Pi_{\nay}$ can be seen as a ``wrapper'' around an underlying proof system $\Pi$. For example, $\Pi_{\sf nay}$ defines a proving algorithm $\Pi_{\nay}.\prove$ which uses the original prover $\Pi.\prove$ as a subroutine.

\begin{definition}[Naysayer proof]\label{def:naysayer_proof}
Given a non-interactive proof system $\Pi = (\setup, \prove, \vrfy)$ for a NP language $\Lang$, the naysayer proof system corresponding to $\Pi$ is a tuple of PPT algorithms $\Pi_{\nay} = (\setup, \prove,\allowbreak \naysay, \vrfynay)$ defined as follows:
    \begin{description}
        \item[$\setup(\secparam, \naysecparam) \randout (\crs,\crs_{\nay})$:] Given (potentially different) security parameters $\secparam$ and $\naysecparam$, output two common reference strings $\crs$ and $\crs_{\nay}$. This algorithm may use private randomness.
        \item[$\prove(\crs, x, \witness) \to \pi'$:] Given a statement $x$ and witness $\witness$ such that $(x,\allowbreak \witness) \in \Rel$, 
        % compute $\pi \gets \Pi.\prove(\crs, x,w)$ and a commitment $\com$ to the evaluation trace of $\Pi.\vrfy(\crs, x, \pi)$. 
        output $\pi' = (\pi, \aux)$, where $\pi \gets \Pi.\prove(\crs, x, \witness)$.
        \item[$\naysay(\crs_\nay, (x,\pi'), \td_{\nay}) \rightarrow \pi_{\nay}$:] Given a statement $x$ and values $\pi' = (\pi,\allowbreak \aux)$ where $\pi$ is a (potentially invalid) proof that $x \in \Lang$ using the proof system $\Pi$, output a naysayer proof $\pi_\nay$ disputing $\pi$. This algorithm may also make use of some (private) trapdoor information $\td_\nay$.
        \item[$\vrfynay(\crs_\nay, (x,\pi'), \pi_\nay) \rightarrow \{0, \perp\}$:] Given a statement-proof pair $(x,\allowbreak \pi')$ and a naysayer proof $\pi_\nay$ disputing $\pi'$, output a bit indicating whether the evidence is sufficient to reject (0) or inconclusive ($\perp$).
    \end{description}
\end{definition}

%Naysayer proofs are most interesting if $\vrfynay$ takes less time than directly verifying the proof $\pi$, i.e., running the $\vrfy$ algorithm. 
A trivial naysayer proof system always exists in which $\pi_\nay = \top$, $\pi' = (\pi, \perp)$, and $\vrfynay$ simply runs the original verification procedure, outputting $0$ if $\Pi.\vrfy(\crs, x, \pi) = 0$ and $\perp$ otherwise.
We say a proof system $\Pi$ is \emph{efficiently naysayable} if there exists a corresponding naysayer proof system $\Pi_{\nay}$ such that $\vrfynay$ is asymptotically faster than $\vrfy$. If $\vrfynay$ is only concretely faster than $\vrfy$, we say $\Pi_{\nay}$ is a \emph{weakly efficient} naysayer proof. Note that some proof systems already have constant proof size and verification time~\cite{EC:Groth16,C:Schnorr89} and therefore can, at best, admit only weakly efficient naysayer proofs. 
Moreover, if $\td_\nay = \perp$, we say $\Pi_\nay$ is a \emph{public} naysayer proof (see \Cref{sec:vshuffle_naysayer} for an example of a non-public naysayer proof). 

\begin{definition}[Naysayer completeness]
    Given a proof system $\Pi$, a naysayer proof system $\Pi_{\nay} = (\setup, \prove, \naysay, \vrfynay)$ is \emph{complete} if, for all honestly generated $\crs, \crs_{\nay}$ and all values of $\aux$,\footnotemark
    % all statements $x$, all invalid proofs $\pi$, and all corresponding auxiliary information $\witness_{\nay}$, 
    given an invalid statement-proof pair $(x,\pi)$, $\naysay$ outputs a valid naysayer proof $\pi_{\nay}$. That is, for all $\secpar, \naysecpar \in \NN$ and all $\aux,x,\pi$,
\footnotetext{We do not place any requirement on $\aux$.}
\begin{equation*}
    \Pr\left[
        \vrfynay(\crs_{\nay}, (x,(\pi, \aux)), \pi_{\nay})=0 
        \middle| 
        \begin{array}{c}
            (\crs, \crs_{\nay}) \gets \setup(\secparam, \naysecparam)~\land\\
            \Pi.\vrfy(\crs, x,\pi) \neq 1~\land\\
            \pi_{\nay} \gets \naysay(\crs_{\nay}, (x, (\pi, \aux)), \perp)
        \end{array}
    \right] = 1.
\end{equation*}
% \noemi{Note this means completeness is only guaranteed to hold if you do not use a trapdoor... is this a problem? Edit: no, more like you can always naysay even without trapdoor}
\end{definition}

\begin{definition}[Naysayer soundness] Given a proof system $\Pi$, a naysayer proof system $\Pi_{\nay}$ is \emph{sound} if, for all PPT adversaries $\adv$, and for all honestly generated $\crs, \crs_\nay$, all $(x,w) \in \Rel_\Lang$, and all correct proofs $\pi'$, $\adv$ produces a passing naysayer proof $\pi_{\nay}$ with at most negligible probability. That is, for all $\secpar, \naysecpar \in \NN$, and all $\td_\nay$,
\begin{equation*}
    \Pr\left[
        \vrfynay(\crs_{\nay}, (x,\pi'), \pi_{\nay})=0 
        \middle| 
        \begin{array}{c}
            (\crs, \crs_\nay) \gets \setup(\secparam, \naysecparam)~\land\\
            % \Pi.\vrfy(\crs, x,\pi)=1~\land\\
            (x,w) \in \Rel_\Lang~\land
            \pi' \gets \prove(\crs, x, w)~\land\\
            \pi_\nay \gets \adv(\crs_\nay, (x, \pi'), \td_\nay)
        \end{array}
    \right] \leq\negl[\naysecpar].\footnotemark
\end{equation*}
\footnotetext{If we assume $\aux$ is computed correctly, the second line of the precondition can be simplified to see that $\Pi_\nay$ is required to be a sound proof system for the language $\Lang_\nay = \{(x,\pi) : x \notin \Lang ~\lor~ \Pi.\vrfy(\crs, x, \pi) \neq 1 \}$.}
\end{definition}

Next, we show that every proof system has corresponding naysayer proof system with a logarithmic-sized (in the size of the verification circuit) naysayer proofs and constant verification time (i.e., a succinct naysayer proof system). %We follow the blueprint of~\cite{FOCS:FeiLapSha90}.

\begin{lemma}\label{lemma:naysayingCSAT}
    A claimed satisfying assignment for a circuit $C: \mathcal{X} \to \{0,1\}$ on input $x \in \mathcal{X}$ is efficiently naysayble. That is, if $C(x) \neq 1$, there is an $O(\log{\sizeof{C}})$-size proof of this fact which can be checked in constant-time, assuming oracle access to the wire assignments of $C(x)$.
    % For every NP language $\Lang$ with relation $\Rel_\Lang$, \todo{change to CSAT specifically}
    % assuming a binding commitment scheme with constant-size openings, 
    % there exists a naysayer proof system $\Pi_{\nay}$ with logarithmic-sized proofs $\pi_{\nay}$ and constant-time verifier.
\end{lemma}
\begin{proof}
    % For any NP language $\Lang$, there exists a non-interactive proof system $\Pi = (\setup, \prove, \vrfy)$ for $\Rel_\Lang$ where $\vrfy(\crs, \cdot, \cdot)$ can be represented as a boolean circuit $C$ of size $\mathsf{poly}(\lvert x \rvert)$~\cite{C:LapSha90}. Recall that for all $(x,w) \in \Rel_\Lang$ and $\crs \gets \Pi.\setup(\secparam)$, by correctness, we have that if $\pi' \gets \Pi.\prove(\crs, x, w)$ then $C(x,\pi') = 1$. Therefore, if there is some gate of $C$ for which the wire assignment is inconsistent, then the proof $\pi'$ is incorrect. To naysay, i.e., to show the incorrectness of $\pi'$, the naysayer simply provides the index of the inconsistent gate.
    % Recall that the verifier has access to the wire assignments of $C(x,\pi')$ as part of $\pi$ (see \Cref{def:naysayer_proof}). 
    Without loss of generality, let $C$ be a circuit of fan-in 2. 
    
    If $C(x) \neq 1$, then there must be some gate of $C$ for which the wire assignment is inconsistent. Let $i$ be the index of this gate (note $\sizeof{i} \in O(\log{\sizeof{C}})$). To confirm that $C(x) \neq 1$, a party can re-evaluate the indicated gate $G_i$ on its inputs $a,b$ and compare the result to the output wire $c$. That is, if $G_i(a,b) \neq c$, the verifier rejects the satisfying assignment.
    % The verifier then checks whether the wire assignments of $C(x,\pi')$ are consistent with a correct evaluation of the gate, which is a constant-time operation assuming constant-time indexing into the wire assignments. 
    % Furthermore, the naysayer proof consists only of the gate index, which is logarithmic in the circuit size, i.e., succinct.
\end{proof}

\begin{theorem}\label{thm:naysayer}
    Every proof system $\Pi$ with $\poly[\sizeof{x}, \sizeof{\witness}]$ verification complexity has a succinct naysayer proof.
\end{theorem}
\begin{proof}
Given any proof system $\Pi$, the evaluation of $\Pi.\vrfy(\crs,\cdot,\cdot)$ can be represented as a circuit $C$. (We assume this circuit description is public.) 
% Let $\COM$ be a deterministic, binding commitment scheme with succinct openings and $\com \gets \COM.\Commit(\Pi.\vrfy(\crs,x,\pi))$ be a commitment to the verification trace of a statement-proof pair $(x,\pi)$. 
Then the following is a complete and sound naysayer proof system $\Pi_\nay$:

\begin{description}
    \item[$\setup(\secparam, \naysecparam)$:] Output $\crs \gets \Pi.\setup(\secparam)$ and $\crs_\nay := \varnothing$.
    % $\crs_\nay \gets \COM.\setup(\naysecparam)$.
    \item[$\prove(\crs, x, w) \to \pi'$:] Let $\pi \gets \Pi.\prove(\crs, x, w)$ and $\aux$ be the wire assignments of $\Pi.\vrfy(\crs, x, \pi)$. Output $\pi' = (\pi, \aux)$.
    \item[$\naysay(\crs_\nay, (x, \pi'), \td_\nay)$:] Parse $\pi' = (\pi, \aux)$\footnote{If $\sizeof{\aux}$ is larger than the number of wires in $C$, truncate it to the appropriate length.} and output $\pi_\nay := \top$ if $\aux = \aux' \concat 0$. Otherwise, evaluate $\Pi.\vrfy(\crs, x, \pi)$. If the result is not 1, search $\aux$ to find an incorrect wire assignment for some gate $G_i \in C$. %(if $\aux = \aux' \concat 0$, let $i = \sizeof{C}-1$). 
    Output $\pi_\nay := i$.
    \item[$\vrfynay(\crs_\nay, (x, \pi'), \pi_\nay)$:] Parse $\pi' = (\cdot, \aux)$ and $\pi_\nay = i$. If $\aux = \aux' \concat 0$, output 0, indicating rejection of the proof $\pi'$. Otherwise, obtain the values $\mathsf{in},\mathsf{out} \in \aux$ corresponding to the gate $G_i$ % (recall that the circuit description is public so $G_i$ is known)
    and check $G_i(\mathsf{in}) \stackrel{?}{=} \mathsf{out}$. If the equality does not hold, output $\perp$; else output 0.
\end{description}

Completeness (if a $\pi$ fails to verify, we can naysay $(\pi, \aux)$) follows by \Cref{lemma:naysayingCSAT}. If $\Pi.\vrfy(\crs,x,\pi) \neq 1$, then we have two cases: If $\aux$ is consistent with a correct evaluation of $\Pi.\vrfy(\crs, x, \pi)$, either $\aux = \aux' \concat 0$ (and $\vrfynay$ rejects) or we can apply the lemma to find an index $i$ such that $G_i(\mathsf{in}) \neq \mathsf{out}$ for $\mathsf{in}, \mathsf{out} \in \aux$, where $G_i \in C$. Alternatively, if $\aux$ is not consistent with a correct evaluation, there must be some gate (with index $i'$) which was evaluated incorrectly, i.e., $G_{i'}(\mathsf{in}) \neq \mathsf{out}$ for $\mathsf{in}, \mathsf{out} \in \aux$.

Soundness follows by the completeness of $\Pi$. If $(x,w) \in \Rel_\Lang$ and $\pi' = (\pi, \aux)$ is computed correctly, completeness of $\Pi$ implies $\Pi.\vrfy(\crs, x, \pi) = 1$. Since $\aux$ is correct, it follows that $\aux \neq \aux \concat 0$ and $G_i(\mathsf{in}) = \mathsf{out}$ for all $i \in \sizeof{C}$ and corresponding values $\mathsf{in},\mathsf{out} \in \aux$. Thus there is no index $i$ which will cause $\vrfynay(\crs_\nay, (x, \pi'), i)$ to output 0.

Succinctness of $\pi_\nay$ follows from the fact that $\sizeof{i} = \log{\sizeof{\Pi.\vrfy(\crs, \cdot, \cdot)}} = 
% \bigOmega{\log{\sizeof{C}}} = 
\bigO{\log(\sizeof{x}, \sizeof{\witness})} \in o(\sizeof{x} + \sizeof{\witness})$ and that the runtime of $\vrfynay$ is constant.
\end{proof}

The proof of \Cref{thm:naysayer} gives a generic way to build a succinct naysayer proof system for any proof system $\Pi$ with polynomial-time verification. For succinct proof systems, the generic construction even allows efficient (sublinear) naysaying, since the runtime of $\naysay$ depends only on the runtime of $\Pi.\vrfy$, which is sublinear if $\Pi$ is succinct.

Notice that although the syntax gives $\pi' = (\pi, \aux)$ as an input to the $\vrfynay$ algorithm, in the generic construction the algorithm does not make use of $\pi$. Thus, if the naysayer rollup from \Cref{fig:naysayer} were instantiated with this generic construction, $\pi$ would not need to be posted on-chain since the on-chain verifier (running the $\vrfynay$ algorithm) will not use this information. In fact, the verifier wouldn't even need most of $\aux$---only the values corresponding to the gate $G_i$, which is determined by $\pi_\nay$. Thus, although $\pi'$ must be available to all potential naysayers, only a small (adaptive) fraction of it must be accessible on-chain. In \Cref{sec:naysayer_storage}, we will discuss how to leverage this insight to reduce the storage costs of a naysayer rollup. 
% First, in \Cref{sec:naysayer_apps}, we show application-specific naysayer constructions which are more efficient than the generic naysayer proof.

% \paragraph{Naysaying a zkVM.} 
% \noemi{Given an underlying zkVM, you can define a corresponding naysayer proof based on the $\mathsf{zkVM}.\vrfy$ circuit --- with the rise of zkVMs, should mention this. This is still ``application-specific'', since the naysayer proof depends on what zkVM is used (on its design/proof system). But in practice it's more ``general-purpose''. However, I think naysayer proofs are inherently application-specific since they rely on the structure of the verification circuit. But if we, as an area, start using only a few common verification circuits (e.g., adopting a few standard zkVMs), then the naysayer proof system can be reused.}
% 
% \noemi{update: not sure if it makes sense to talk about a $\mathsf{zkVM}.\vrfy$ circuit---they're generally just some proof system's verification procedure, e.g. the RISC Zero zkVM is running a STARK. Just talk about general naysaying by committing to a verification trace---and mention zkVMs somewhere as an example---then talk about particular languages where the verification trace is not needed because the structure is repetitive (i.e., conjunctions).}
