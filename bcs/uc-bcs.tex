\section{UC-Secure Blind Conditional Signatures}\label{sec:uc-bcs}

We now model security in the universal composability framework (\Cref{sec:uc}) extended to support a global setup~\cite{TCC:CDPW07} in order to capture concurrent executions. 
We consider \emph{static} corruptions, where the adversary announces at the beginning which parties it corrupts.

In our protocol, we assume the existence of a general-purpose UC-secure 2-party computation (2PC) protocol~\cite{C:HorKat07,STOC:CLOS02}, where two parties interact with the ideal functionality to compute a function $f(x,y)$ over their private inputs $x$ and $y$. 

\begin{figure}[tb]
    \centering
    \begin{idealFunctionality}{Ideal Functionality $\flocks$}
    \noindent\underline{Puzzle Promise:} On input $(\Promise, A)$ from $B$, $\flocks$ 
    proceeds as follows:
    \begin{itemize}
        \item Send $(\mathsf{promise}\mathsf{-req}, B)$ to $H$	 and $\sdv$. % NOTE: We model here that a malicious Tumbler can avoid token check and continue with the protocol.
        \item Receive $(\mathsf{promise}\mathsf{-res}, b)$ from $H$.
        \item If $b = \bot$ then abort.
        \item Sample $\pid, \pid' \sample \bin^\secpar$.
        \item Store the tuple $(\pid, \pid', \bot)$ into $\Plist$.
        \item Send $(\mathsf{promise}, (\pid,\pid'))$ to $B$, $(\mathsf{promise}, \pid)$ to $H$, $(\mathsf{promise}, \pid')$ to $A$, and inform $\sdv$. 
    \end{itemize}
    
    \noindent\underline{Puzzle Solver:} On input $(\Pay, B, \pid')$ from $A$, $\flocks$ proceeds 
    as follows:
    \begin{itemize}
        \item If $(\cdot, \pid', \cdot) \notin \Plist$ then abort.
        \item Send $(\mathsf{solve}\mathsf{-req}, A, \pid')$ to $H$ and $\sdv$.
        \item Receive $(\mathsf{solve}\mathsf{-res}, b)$ from $H$.
        \item If $b = \bot$ then abort.
        \item Update entry to $(\cdot, \pid', \top)$ in $\Plist$.
        \item Send $(\mathsf{solved}, \pid', \top)$ to $A, B$ and $\sdv$.
    \end{itemize}
    
    \noindent\underline{Open:} On input $(\Open, \pid)$ from $B$, $\flocks$ proceeds as 
    follows:
    \begin{itemize}
        \item If $(\pid, \cdot, b) \notin \Plist$ or $b = \bot$ then send $(\mathsf{open},\pid, \bot)$ to $B$ and abort. Else send $(\mathsf{open}, \pid, \top)$ to $B$.
    \end{itemize}
    \end{idealFunctionality}
    %}}
    \caption{Ideal functionality $\flocks$ (corresponds to $\F_{\AAL}$ in~\cite{SP:TaiMorMaf21}). Portions related to griefing protection (i.e., registration) have been removed.}
    \label{fig:ideal-a2l}
    \end{figure}

\subsection{Ideal Functionality}
In \Cref{fig:ideal-a2l}, we describe the ideal functionality $\flocks$ that captures the functionality and security of BCS in the UC framework.
The ideal functionality has three routines, namely for puzzle promise, puzzle solver, and open, which intuitively capture the functionality of BCS as discussed in~\Cref{sec:bcs}. On a high level, $\flocks$ captures \emph{blindness} by sampling the puzzle identifiers $\pid$ and $\pid'$, which correspond to puzzle promise and puzzle solve interactions, locally together, but never revealing them together to the hub. $\flocks$ captures \emph{atomicity} by returning a successful message (not aborting) for $\pid$ during open if and only if it sent a successful $\mathsf{solved}$ message during the puzzle solve interaction for the puzzle identifier $\pid'$ (where $\pid$ and $\pid'$ correspond to each other). Note that the above atomicity guarantee implies the game-based definitions of unlockability and unforgeability.

Our functionality $\flocks$ is taken verbatim from the $\F_{\AAL}$ functionality in \cite{SP:TaiMorMaf21} except that we do not consider user registrations (as done in $\F_{\AAL}$) to tackle \emph{griefing attacks}~\cite{griefing} in the coin mixing layer. These attacks are mounted by Bob starting many puzzle promise operations, each of which
requires Hub to lock coins, whereas the corresponding puzzle solver interactions are never carried out. As
a consequence, all of Hub's coins are locked and no longer
available, which results in a form of denial of service attack. 
We argue that the issue does not concern the functionality or security of BCS as a  cryptographic tool, but only affects the coin mixing protocol at the transaction layer.
We emphasize that griefing attacks can be thwarted at this layer in both the formal model and the construction using the same ideas as in~\cite{SP:TaiMorMaf21}.

\subsection{The \aaluc Protocol}\label{sec:our_protocol}

We now describe our protocol \aaluc that realizes the ideal functionality \flocks.
% blind conditional signatures construction $\Pi_\mathsf{BCS}$. 
We assume the following cryptographic building blocks:
\begin{itemize}
    \item An adaptor signature scheme $\Pi_\ADP$ defined with respect to $\Pi_\DS$ and a hard relation $R_\mathsf{DL}$.
    \item A UC-secure NIZK proof system $\Pi_\NIZK$ for the language $$\mathcal{L} := \{ (\ek, Y,c) : \exists s, \suchthat c \gets \Pi_\ENC.\enc(\ek, s) \land Y = g^s \}. $$ 
    \item A UC-secure 2PC protocol.
    \item A CCA-secure~\cite{STOC:GolMic82} encryption scheme $\Pi_\ENC := (\kgen, \enc, \dec)$ with unique decryption keys.
\end{itemize}
The property of unique decryption keys is formalized below.
\begin{definition}[Unique Decryption Keys]
An encryption scheme $\Pi_\ENC$ has unique decryption keys if the $\kgen$ algorithm is of the following form:
\begin{itemize}
    \item Sample $\dk \sample \{0,1\}^\secpar$.
    \item Run $\ek \gets \kgen(\dk)$.
\end{itemize}
Furthermore, for all $\ek$ output by $\kgen$, there exists a unique $\dk$ such that $\ek = \kgen(\dk)$. In other words, $\kgen$ is injective.
\end{definition}
This property is already satisfied by most natural public-key encryption schemes, but it can be generically achieved by augmenting the encryption key with a perfectly binding commitment $\Commit(\dk)$ to the decryption key $\dk$.

\paragraph{Protocol Description} We assume Alice and Hub have a key pair for the signature scheme $\Pi_\DS$. Specifically, we have the verification-signing key pairs $(\vk_\hb^H, \sk_\hb^H)$ and $(\vk_\ah^A, \sk_\ah^A)$, belonging to Hub and Alice, respectively.
We then have two messages $m := m_\hb$ and $m' := m_\ah$ for which the users wish to generate blind conditional signatures.
The setup and open algorithms are formally described in~\Cref{fig:our_protocol_algorithms}. The puzzle promise and puzzle solver of \aaluc are formally described in~\Cref{fig:our_protocol_hub_bob} and~\Cref{fig:our_protocol_alice_hub}, respectively.
For ease of understanding, we briefly describe below our \aaluc protocol in terms of the differences with the \aal protocol (\Cref{fig:a2l_promise,fig:a2l_payment}).

\begin{itemize}
    \item The setup algorithm (\Cref{fig:our_protocol_algorithms}) of \aaluc generates the keys of Hub, which are the keys for the (CCA-secure) encryption scheme $\Pi_\ENC$. 
    \item In $\Promise$ of \aaluc (\Cref{fig:our_protocol_hub_bob}),
    \begin{itemize}
        \item The NIZK proof system  is UC-secure.
        \item Bob no longer re-randomizes the instance or the ciphertext. Therefore, we drop the re-randomization steps 
        (line 9 and 10) of $\Promise$ in \aal (\Cref{fig:a2l_promise}).
        Simply set the puzzle to $\tau := (m_\hb, \presig_\hb^H, (Y,c))$.
    \end{itemize}
    \item In $\Pay$ of \aaluc (\Cref{fig:our_protocol_alice_hub}),
    \begin{itemize}
        \item Alice no longer sends the ciphertext to Hub 
        (line 5 of~\Cref{fig:a2l_payment}). 
        We therefore remove the local decryption step 
        (line 6 of~\Cref{fig:a2l_payment}), 
        and replace it with a 2PC protocol (line 6 of \Cref{fig:our_protocol_alice_hub}).
        \item  At the end of the 2PC protocol, Alice receives $\bot$, while Hub receives the value $z$. Hub additionally  checks if $Y' = g^z$ (line 7) and uses $z$ to adapt the pre-signature $\presig_\ah^A$ to signature $\sigma_\ah^A$.
        \item We add a check for Alice (line 10) that $\sigma_\ah^A$ is a valid signature before extracting the witness $z'$ in line 12. 
    \end{itemize}
    \item The $\open$ algorithm (\Cref{fig:our_protocol_algorithms}) is the same as in~\Cref{fig:a2l_openverif}
    of \aal, except we skip removing the randomness factor. The algorithm in~\Cref{fig:our_protocol_algorithms} now simply adapts a pre-signature $\presig$ to a valid signature $\sigma$ which it returns as output.
\end{itemize}

\begin{figure}
    \centering
    \begin{pchstack}[boxed]
        \procedure[space=auto,codesize=\small]{$\Setup(\secparam)$}{
    (\ek_H, \dk_H) \gets \Pi_\ENC.\kgen(\secparam)\\
    \mathbf{set}\ \tilde \pk := \ek_H, \tilde \sk := \dk_H\\
    \pcreturn (\tilde \pk, \tilde \sk)
    }
    
    \pchspace
    
    \procedure[space=auto,codesize=\small]{$\open(\tau, s)$}{
    \mathbf{parse}\ \tau := ( \cdot, \presig, \cdot)\\
    \sigma \gets \Pi_\ADP.\adapt(\tilde \sigma, s)\\
    \pcreturn \sigma
    }
    \end{pchstack}
    \caption{Setup and Open algorithms of our conditional puzzle construction}
    \label{fig:our_protocol_algorithms}
\end{figure}
\begin{figure}[tbh]
\hspace{-10em}
\begin{minipage}{\textwidth}
\begin{pchstack}[boxed]
    \procedure{Public parameters: group description $(\GG, g, q)$, message $m_\hb$}{
 	\text{ $\Promise \langle H( \dk_H, \sk_\hb^H), \cdot \rangle$} \< \< \text{$\Promise \langle \cdot, B(\ek_H, \vk_\hb^H) \rangle$} \\[][\hline]
	\pcln s \sample \ZZ_p, Y := g^s \< \< \\ [-0.15\baselineskip][]
	\pcln c \gets \Pi_\ENC.\enc(\ek_H, s) \< \< \\ [-0.15\baselineskip][]
	\pcln \pi_s \gets \NIZK.\prove( (\ek_H, Y, c), s)  \< \< \\ [-0.15\baselineskip][]
	\pcln \presig_{\hb}^H \gets \Pi_\ADP.\presig( \sk_\hb^H, m_\hb, Y ) \< \< \\ [-0.85\baselineskip][]
	\pcln \< \sendmessageright*{Y, c, \pi_s, \presig_\hb^H}\< \\ [-0.55\baselineskip][]
	\pcln \< \< {\mathbf{If}\ \NIZK.\vrfy((\ek_H,Y,c), \pi_s) \neq 1\ \mathbf{ then}\ \pcreturn \bot} \\ [-0.15\baselineskip][]
	\pcln \< \< \mathbf{If}\ \Pi_\ADP.\prevrfy(\vk_\hb^H, m_\hb, Y, \presig_\hb^H) \neq 1\ \mathbf{then} \\ [-0.15\baselineskip][]
	\pcln \< \< \quad \pcreturn \bot \\ [-0.15\baselineskip][]
	\pcln \< \< \mathbf{set}\ \tau := (m_\hb, \presig_\hb^H, (Y, c)) \\ [-0.15\baselineskip][]
	\pcln \pcreturn \bot \< \< \pcreturn \tau
}
\end{pchstack}
\end{minipage}
\caption{Puzzle promise protocol of \aaluc}
\label{fig:our_protocol_hub_bob}
\end{figure}
\begin{figure}[tbh]
\hspace{-8em}
\begin{minipage}{\textwidth}
\begin{pchstack}[boxed]
    \procedure{Public parameters: group description $(\GG, g, q)$, message $m_\ah$}{
 	\text{$\Pay \langle A(\sk_\ah^A, \ek_H, \tau), \cdot\rangle$} \< \< \text{$\Pay \langle \cdot, H(\dk_H, \vk_\ah^A) \rangle$} \\[][\hline]
 	\pcln \pcparse \tau := ( \cdot, \cdot, (Y, c)) \< \< \\ [-0.15\baselineskip][]
 	\pcln r \sample \ZZ_p, Y' := Y \cdot g^r \< \< \\ [-0.15\baselineskip][]
 	\pcln \tilde{\sigma}_\ah^A \gets \Pi_\ADP.\presig( \sk_\ah^A, m_\ah, Y') \< \< \\ [-0.85\baselineskip][] 
 	\pcln \< \sendmessageright*{Y', \tilde{\sigma}_\ah^A} \< \\ [-0.55\baselineskip][]
 	\pcln \< \< \mathbf{If}\ \Pi_\ADP.\prevrfy(\vk_\ah^A, m_\ah, Y', \tilde{\sigma}_\ah^A) \neq 1\ \mathbf{then} \\ [-0.15\baselineskip][]
 	\< \< \quad \pcreturn \bot\\ [-0.15\baselineskip][]
 	\pcln \< \begin{subprocedure}
 	\dbox{\procedure[linenumbering]{$\twoPC((r,c), (\dk_H))$}{
 	\pcif \ek_H \neq \Pi_\ENC.\mathsf{Gen}(\dk_H)\\
 	\pcthen \text{abort}\\
 	s^* \gets \Pi_\ENC.\dec(\dk_H, c)\\
 	z := s^* + r\\
 	\pcreturn ((\bot), (z))
 	}}
 	\end{subprocedure}
 	\< \\ [-0.15\baselineskip][]
 	\pcln \< \< \mathbf{If}\ Y' \neq g^z\ \mathbf{then}\ \pcreturn \bot\\ [-0.15\baselineskip][]
 	\pcln \< \< \sigma_\ah^A \gets \Pi_\ADP.\adapt(\tilde{\sigma}_\ah^A, z)\\ [-0.85\baselineskip][]
 	%\pcln \< \< \sigma_\ah^H \gets \Pi_\sign(\sk_\ah^H, m)\\ [-0.85\baselineskip][]
 	\pcln \< \sendmessageleft*{{\sigma}_\ah^A} \< \\ [-0.55\baselineskip][]
 	\pcln \mathbf{If}\ \Pi_\ADP.\vrfy(\vk_\ah^A, m_\ah, \sigma_\ah^A) \neq 1\ \mathbf{then} \< \< \\ [-0.15\baselineskip][]
 	\pcln \quad \pcreturn \bot \< \< \\ [-0.15\baselineskip][]
 	\pcln z' \gets \Pi_\ADP.\ext(\tilde{\sigma}_\ah^A, \sigma_\ah^A, Y')\< \< \\ [-0.15\baselineskip][]
 	\pcln s := z'-r \< \< \\ [-0.15\baselineskip][]
 	\pcln \pcreturn (\sigma_\ah^A, s) \< \< \pcreturn \sigma_\ah^A
}
\end{pchstack}
\end{minipage}
\caption{Puzzle solver protocol of \aaluc}
\label{fig:our_protocol_alice_hub}
\end{figure}

\subsection{Security Analysis}

We now show that \aaluc satisfies UC-security. 
In favor of a simpler analysis, we assume that the verification keys of all parties are honestly generated. In practice, this can be enforced by augmenting keys with NIZKs that certify their validity~\cite{PKC:Boldyreva03,EC:LOSSW06}.

\begin{theorem}\label{thm:a2luc}
Let $\Pi_\ENC$ be a CCA-secure encryption scheme, $\Pi_\ADP$ a secure adaptor signature scheme, $\twoPC$ a UC-secure two-party computation protocol, and $\Pi_\NIZK$ a UC-secure NIZK for the language $\LL$ above. Then the \aaluc protocol UC-realizes $\flocks$.
\end{theorem}
\begin{proof}
We proceed by describing the UC simulator and arguing about indistinguishability from the real execution of the protocol. We consider the cases where the adversary corrupts a different subset of parties separately. We describe the simulator for a single session and the security of the overall interaction is established via a standard hybrid argument.

\paragraph{H Corrupted}
We first give a simulator $\Sim_H$, then give a series of hybrid experiments that gradually change the real experiment (i.e., the construction in ~\Cref{fig:our_protocol_hub_bob,fig:our_protocol_alice_hub}) into the ideal experiment given by the interaction of the corrupted $H$ and the simulator $\Sim_H$, which has access to $\flocks$.

\smallskip
\noindent\underline{Simulator $\Sim_H$:} Upon receiving $(\mathsf{promise}\mathsf{-req}, B)$ from $\flocks$, $\Sim_H$ proceeds as follows:
\begin{enumerate}
    \item Ask the attacker to initiate a session and receive in return $(Y, c, \pi_s, \presig_\hb^H)$. If $\Pi_\ADP.\prevrfy(\vk_\hb^H, m_\hb, Y, \presig_\hb^H) = 1$ and $\NIZK.\mathsf{V}((\ek_H, Y,c), \pi_s) = 1$, proceed as in the protocol and send $(\mathsf{promise}\mathsf{-res}, \top)$ to $\flocks$. Otherwise, abort and send $(\mathsf{promise}\mathsf{-res}, \bot)$.
    \item Receive $(\mathsf{promise}, \pid)$ from $\flocks$.
    \item Upon receiving $(\mathsf{solve}\mathsf{-req}, A, \pid')$ from $\flocks$ at some later point, sample $y' \sample \ZZ_q$ and generate keys $(\vk_\ah^A, \sk_\ah^A) \gets \Pi_\ADP.\kgen(\secparam)$. Compute $Y' \gets g^{y'}$, $\presig_\ah^A \gets \Pi_\ADP.\presign(\sk_\ah^A, m_\ah, Y')$ and send them to the attacker.
    \item When the attacker initiates the 2PC, run the 2PC simulator to recover its input $\dk_H$. If $\ek_H \neq \Pi_\ENC.\kgen(\dk_H)$, program the output of the 2PC to $\bot$, otherwise to $y'$.
    \item Receive $\sigma_\ah^A$ in response from the attacker and check if $\Pi_\ADP.\vrfy(\vk_\ah^A,\allowbreak m_\ah, \sigma_\ah^A) = 1$. Additionally check if $\Pi_\ADP.\ext(\presig_\ah^A, \sigma_\ah^A, Y') = y'$. If both checks pass, send $(\mathsf{solve}\mathsf{-res}, \top)$ to $\flocks$, compute $s \gets \Pi_\ENC.\dec(\dk_H,\allowbreak c)$, and output $(\sigma_\ah^A, s)$ as in the protocol; otherwise, send $(\mathsf{solve}\mathsf{-res}, \bot)$ and abort.
    \item If, at any point before the successful completion of step 4, the attacker produces a valid signature $\sigma_\ah^A$, or at any point in the protocol (including after step 4), a valid signature on a message $m_\ah' \neq m_\ah$, send $(\mathsf{solve}\mathsf{-res}, \bot)$ to $\flocks$ and abort.
\end{enumerate}

\medskip

\noindent\underline{Hybrid $\hybrid_0$:} This corresponds to the real protocol (\Cref{fig:our_protocol_hub_bob,fig:our_protocol_alice_hub}).

\smallskip
\noindent\underline{Hybrid $\hybrid_1$:} Simulate the 2PC (Fig.~\ref{fig:our_protocol_alice_hub}, line 6) and send the output $z$ to $H$.

\smallskip
\noindent\underline{Hybrid $\hybrid_2$:} Replace $Y'$ with $Y'' := g^{y'}$ where $y' \sample \ZZ_q$ (Fig.~\ref{fig:our_protocol_alice_hub}, line 2). If $\Pi_\ENC.\kgen(\dk_H)=\ek_H$, send $y'$ to $H$ instead of $z$; otherwise, send $\bot$.

\smallskip
\noindent\underline{Hybrid $\hybrid_3$:} Abort if $z' \neq y'$ (after line 12 of Fig.~\ref{fig:our_protocol_alice_hub}).

\smallskip
\noindent{\underline{Hybrid $\hybrid_4$:} Abort if any valid signature $\sigma_\ah^A$ is received on a different message $m_\ah' \neq m_\ah$ or on any message before the 2PC has successfully completed.}

\begin{claim}
For all PPT distinguishers $\env$,
\[
    \exec_{\hybrid_0,\adv,\env} \approx \exec_{\hybrid_1,\adv,\env}
\]
\end{claim}
\begin{proof}
This follows directly from the security of the 2PC protocol.
\end{proof}

\begin{claim}
For all PPT distinguishers $\env$,
\[
    \exec_{\hybrid_1,\adv,\env} \approx \exec_{\hybrid_2,\adv,\env}
\]
\end{claim}
\begin{proof}
By the uniqueness of the decryption key and correctness of $\Pi_\ENC$, $\ek_H = \Pi_\ENC.\kgen(\dk_H)$ implies $\Pi_\ENC.\dec(\dk_H,\Pi_\enc.\enc(\ek_H, m)) = m$ for all $m$ in the message space. Thus, the output of the 2PC $z$ is necessarily $s+r$, where $s \in \ZZ_q$ such that $c = \Pi_\ENC.\enc(\ek_H,s) ~\wedge~ Y=g^s$ (this is guaranteed by the NIZK). 
Since $r$ is uniformly random, $y'$ is identically distributed to $z=s+r$. The same holds for $Y''$ and $Y' = Y \cdot g^r$. Furthermore, it still holds that $y'$ is the discrete logarithm of $Y''$ (cf. $z$ and $Y'$).
\end{proof}

\begin{claim}
For all PPT distinguishers $\env$,
\[
    \exec_{\hybrid_2,\adv,\env} \approx \exec_{\hybrid_3,\adv,\env}
\]
\end{claim}
\begin{proof} 
If $z' \neq y'$, by the uniqueness of dlog witnesses $g^{z'} \neq Y''$. By the witness extractability of $\Pi_\ADP$, $\Pr[g^{z'} \neq Y'' ~\wedge~ \Pi_\ADP.\vrfy(\vk_\ah^A,m_\ah, \sigma_\ah^A)=1]$ is negligible, so the abort only happens with negligible probability.
\end{proof}

\begin{claim}
For all PPT distinguishers $\env$,
\[
    \exec_{\hybrid_3,\adv,\env} \approx \exec_{\hybrid_4,\adv,\env}
\]
\end{claim}
\begin{proof}
    {Any distinguishing advantage implies a case in which $\adv$ outputs some valid signature $\sigma_\ah^A$ for some message $m_\ah'$ for which it has potentially been given a presignature $\tilde\sigma_\ah^A$ and corresponding statement $Y$. This signature is a winning instance in the unforgeability experiment for $\Pi_\ADP$, but by assumption this only occurs with negligible probability, and so the distinguishing advantage must be negligible. Therefore the experiments are statistically close.}
\end{proof}

\begin{claim}
For all PPT distinguishers $\env$,
\[
    \exec_{\hybrid_4,\adv,\env} \equiv \exec_{\flocks,\Sim,\env}
\]
\end{claim}
\begin{proof}
    $\hybrid_4$ is identical to the ideal world.
\end{proof}

\paragraph{A,B Corrupted} Again, we give a simulator $\Sim_{AB}$ that interacts with $\flocks$ and show by a series of hybrids that our protocol is indistinguishable from ideal experiment in which the corrupted parties interact with the simulator $\Sim_{AB}$.

\smallskip
\noindent\underline{Simulator $\Sim_{AB}$:} When a recipient Bob indicates he would like to initiate a transaction, $\Sim_{AB}$ proceeds as follows:
\begin{enumerate}
    \item Send $(\Promise, A)$ to $\flocks$.
    \item Upon receiving $(\mathsf{promise}, (\pid,\pid'))$ from $\flocks$, sample a uniform value $s \sample \ZZ_q$ and compute $Y \gets g^s$. Generate keys $(\ek_H,\dk_H) \gets \Pi_\ENC.\kgen(\secparam)$ and $(\vk_\hb^H, \sk_\hb^H) \gets \Pi_\ADP.\kgen(\secparam)$; let $c \gets \Pi_\ENC.\enc(\ek_H, 0)$ and $\presig_\hb^H \gets \Pi_\ADP.\presign(\sk_\hb^H, m_\hb, Y)$. Simulate the NIZK $\pi_s \gets \NIZK.\mathsf{Sim}(\td,\allowbreak (\ek_H, Y,c))$. Finally, pre-compute $\sigma_\hb^H \gets \Pi_\ADP.\adapt(\presig_\hb^H, s)$ and save $((\pid,\pid'),(Y,c,s,\sigma_\hb^H),\perp)$ into a table $\Plist$. Send $(Y,c,\pi_s,\presig_\hb^H)$ to the attacker (who is impersonating Bob).
    \item At a later point in time, the attacker sends $(Y', \presig_\ah^A)$ on behalf of Alice. If $\Pi_\ADP.\prevrfy(\vk_\ah^A, m_\ah, Y', \presig_\ah^A) \neq 1$, abort.
    \item When the attacker initiates the 2PC, run the 2PC simulator to recover its inputs $(c^*, r^*)$; compute the result ($\perp$) and return it to the attacker.
    \item Depending on whether or not $c^* \in \Plist$ do the following:
    \begin{enumerate}
        \item If $c^*\in\Plist$, retrieve the corresponding $Y$, $s$, and $\pid'$. Check that $Y' = Y \cdot g^{r^*}$ (if not, abort); send $\Pi_\ADP.\adapt(\presig_\ah^A, s+r^*)$ to the attacker masquerading as Alice and $(\Pay, B, \pid')$ to $\flocks$. Update the last element of the entry in $\Plist$ to $\top$.
        \item If $c^* \notin \Plist$, compute $z' \gets \Pi_\ENC.\dec(\dk_H, c^*) + r^*$. Check that $Y' = g^{z'}$ (if not, abort) and send $\Pi_\ADP.\adapt(\presig_\ah^A, z')$ to the attacker. Send nothing to $\flocks$. (Note that this corresponds to the case where some party Alice is paying Hub without Bob initiating the interaction, which is something that she can do at any time.)
    \end{enumerate}
    \item When the attacker outputs some valid signature $\sigma_\hb^H$, check that the following conditions hold: $\Pi_\ADP.\vrfy(\vk_\hb^H, m_\hb, \sigma_\hb^H)=1$ and $((\pid,\cdot),\allowbreak(\cdot, \cdot, \cdot, \sigma_\hb^H),\top) \in \Plist$. If so, send $(\Open, \pid)$ to $\flocks$; otherwise, abort.
\end{enumerate}

\medskip

%\todo{May be we describe the hybrids a bit more? A few sentences? I am afraid if a UC reviewer complains.}

\smallskip
\noindent\underline{Hybrid $\hybrid_0$:} This corresponds to the real protocol (\Cref{fig:our_protocol_hub_bob,fig:our_protocol_alice_hub}).

\smallskip
\noindent\underline{Hybrid $\hybrid_1$:} Replace the honestly-computed NIZK $\pi_s$ (\Cref{fig:our_protocol_hub_bob}, line 4) with a simulated proof.

\smallskip
\noindent\underline{Hybrid $\hybrid_2$:} Simulate the 2PC (\Cref{fig:our_protocol_alice_hub}, line 6).

\smallskip
\noindent\underline{Hybrid $\hybrid_3$:} Add the list $\Plist$ and step 5 of the simulator (in particular, case 5a) to \Cref{fig:our_protocol_alice_hub}, line 7-10.

\smallskip
\noindent\underline{Hybrid $\hybrid_4$:} Replace $c$ (\Cref{fig:our_protocol_hub_bob}, line 2) with an encryption of zero.

\smallskip
\noindent\underline{Hybrid $\hybrid_5$:} When Bob outputs a valid signature, abort if $(\cdot,(\cdot,\cdot,\cdot,\sigma_\hb^H),b) \in \Plist$ and $b \neq \top$.

\begin{claim}
For all PPT distinguishers $\env$,
\[
    \exec_{\hybrid_0,\adv,\env} \approx \exec_{\hybrid_1,\adv,\env}
\]
\end{claim}
\begin{proof}
This follows directly from the zero-knowledge property  of the NIZK.
\end{proof}

\begin{claim}
For all PPT distinguishers $\env$,
\[
    \exec_{\hybrid_1,\adv,\env} \approx \exec_{\hybrid_2,\adv,\env}
\]
\end{claim}
\begin{proof}
This follows directly from the UC-security of the 2PC protocol.
\end{proof}

\begin{claim}
For all PPT distinguishers $\env$,
\[
    \exec_{\hybrid_2,\adv,\env} \equiv \exec_{\hybrid_3,\adv,\env}
\]
\end{claim}
\begin{proof}
By definition, for $c^* \in \Plist$, the corresponding $s, Y$ in $\Plist$ are $\Pi_\ENC.\dec(\dk_H,\allowbreak c^*)$ and $g^s$, respectively. Therefore $z' = s + r^*$ and the case of $c^* \in \Plist$ is handled in the same way as all cases were in the previous hybrid experiment.
\end{proof}

\begin{claim}
For all PPT distinguishers $\env$,
\[
    \exec_{\hybrid_3,\adv,\env} \approx \exec_{\hybrid_4,\adv,\env}
\]
\end{claim}
\begin{proof}
Suppose towards a contradiction that $\env$ can distinguish the two executions with nonnegligible probability. We give a reduction to the CCA-security game of $\Pi_\ENC$. The reduction sets $m_0 := s$ and $m_1 := 0$, sends them to the CCA game, and receives $c$. It then acts as hub in its interaction with $\env$, computing everything as in Hybrid 3, except for $c$, which it sets to the ciphertext it received from the game. When it needs to decrypt $c^*$ it uses the CCA decryption oracle. At the end of the execution, based on $\env$'s guess, it outputs a bit to the CCA game (0 if $\env$ guesses $\hybrid_3$, 1 otherwise), which will be correct with nonnegligible advantage. This violates the CCA-security of $\Pi_\ENC$, so the two executions must be indistinguishable. 
\end{proof}

\begin{claim}
For all PPT distinguishers $\env$,
\[
    \exec_{\hybrid_4,\adv,\env} \approx \exec_{\hybrid_5,\adv,\env}
\]
\end{claim}
\begin{proof}
If $b \neq \top$, Alice did not receive the completed signature $\sigma_\ah^A$ for that session and thus cannot recover the secret $s$ to send to Bob. This means Bob's signature $\sigma_\hb^H$ was created without knowing the witness for the pre-signature $\presig_\hb^H$, which, by aEUF-CMA of $\Pi_\ADP$, can only happen with negligible probability. Thus the abort also only happens with negligible probability and the two experiments are indistinguishable.
% \[
%     \Pr[\text{abort in step 6} \wedge \Pi_\ADP.\vrfy(\pk_\hb^H,m_\hb,\sigma_\hb^H)] = 1
% \]
% is negligible.
\end{proof}

\begin{claim}
For all PPT distinguishers $\env$,
\[
    \exec_{\hybrid_5,\adv,\env} \equiv \exec_{\flocks,\Sim,\env}
\]
\end{claim}
\begin{proof}
    $\hybrid_5$ is identical to the ideal world.
\end{proof}

\paragraph{A,H Corrupted} This case is trivial, as $B$ has no secret information and the simulator therefore simply follows the protocol.

\paragraph{H,B Corrupted} The simulator in this case follows the protocol honestly. If hub publishes a valid signature $\sigma_\ah^A$ on a transaction $m$ that is not in the simulator's (acting as Alice) transcript, the simulator aborts. This means that the adversary was able to forge a signature $\sigma_\ah^A$ on some transaction $m$ for which it did not previously receive a pre-signature $\presig_\ah^A$. By EUF-CMA of the adaptor signature scheme, this case occurs with negligible probability and thus for all PPT distinguishers $\env$, the real world (an honest execution of the protocol) and the ideal world (an interaction with the simulator) are indistinguishable.
\end{proof}