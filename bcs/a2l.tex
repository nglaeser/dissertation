\section{The \aal Protocol}\label{sec:description_a2l}

We now recall the \aal system~\cite{SP:TaiMorMaf21}, which is defined over the following cryptographic schemes:
\begin{itemize}
    \item A digital signature scheme $\Pi_\DS$, a hard relation $\Rel_\DL$ for a group $(\GG, g, p)$ with generator $g$ and prime order $p$, and the corresponding adaptor signature scheme $\Pi_\ADP$.
    \item A linearly homomorphic re-randomizable CPA-secure encryption scheme $\Pi_\ENC$.\footnote{Technically, \cite{SP:TaiMorMaf21} uses a different abstraction called ``randomizable puzzle''. However, it is not hard to see that a re-randomizable linearly homomorphic encryption scheme satisfies this notion. For completeness, we show this in~\Cref{sec:randpuzzle}.}
    \item A NIZK proof system $\Pi_\NIZK := (\Setup, \prove, \vrfy) $ for the language $$\Lang := \{ (\ek, Y,c) : \exists s \suchthat c \gets \Pi_\ENC.\enc(\ek, s) \land Y = g^s \}.$$
\end{itemize}
The protocol has three parties: Alice, Bob, and Hub. At the beginning of the system, Hub runs the setup (as described in~\Cref{fig:our_protocol_algorithms}) to generate its keys, which are the keys for the (CPA-secure) encryption scheme $\Pi_\ENC$. The protocol then consists of a promise phase and a solving phase. To conclude, Alice runs the open algorithm given in~\cref{fig:a2l_openverif}.

\begin{figure*}[!t]
    \hspace{-8em}
    \begin{minipage}{\textwidth}
    \begin{pchstack}[boxed]
    \procedure{Public parameters: group description $(\GG, g, q)$, message $m_\hb$}{
    % 	\text{$\Promise \langle H((\pp, \td)), \sk_{\hb}^H, \cdot \rangle$} \< \< \text{$\Promise \langle \cdot,B(\pp, \vk_\hb^H) \rangle$} \\[][\hline]
        \text{$\Promise \langle H(\dk_H, \sk_{\hb}^H), \cdot \rangle$} \< \< \text{$\Promise \langle \cdot,B(\ek_H, \vk_\hb^H) \rangle$} \\[][\hline]
        %\pcln \< \< \sigma_\hb^B \gets \Pi_\DS.\sign(\sk_\hb^B, m_\hb) \\ [-0.85\baselineskip][]
        %\pcln \< \sendmessageleft*{\sigma_\hb^B} \< \\ [-0.55\baselineskip][]
        \pcln s \sample \Z_p, Y := g^s \< \< \\ [-0.15\baselineskip][]
        \pcln c \gets \Pi_\ENC.\enc(\ek_H, s) \< \< \\ [-0.15\baselineskip][]
    % 	\pcln \pi_s \gets \NIZK.\prove(\{\exists s \mid \Pi_\ENC.\dec(\sk_H, c) = s \}, s) \< \< \\ [-0.15\baselineskip][]
        \pcln \pi_s \gets \NIZK.\prove((\ek_H, Y, c), s) \< \< \\ [-0.15\baselineskip][]
        \pcln \presig_\hb^H \gets \Pi_\ADP.\presig(\sk_\hb^H, m_\hb, Y) \\ [-0.85\baselineskip][]
        \pcln \< \sendmessageright*{Y,c, \pi_s, \presig_\hb^H} \< \\ [-0.55\baselineskip][]
        \pcln \< \< {\mathbf{If}\ \NIZK.\vrfy((\ek_H,Y,c),\pi_s) \neq 1\ \mathbf{ then}\ \pcreturn \bot} \\ [-0.15\baselineskip][]
        \pcln \< \< \mathbf{If}\ \Pi_\ADP.\prevrfy(\vk_\hb^H, m_\hb, Y, \presig_\hb^H) \neq 1\ \mathbf{then} \\ [-0.15\baselineskip][]
        \pcln \< \< \quad \pcreturn \bot \\ [-0.15\baselineskip][]
        \pcln \< \< r \sample \Z_q, Y' := Y \cdot g^r \\ [-0.15\baselineskip][]
        \pcln \< \< c' \gets \Pi_\ENC.\mathsf{Rand}(c,r) \\ [-0.15\baselineskip][]
        %\pcln \< \< \mathbf{Send } Y', c' \mathbf{ to } A \\ [-0.15\baselineskip][]
        \pcln \< \< \mathbf{Set}\ \tau := (r, m_\hb, \presig_\hb^H, (Y,c),(Y',c')) \\ [-0.15\baselineskip][]
        \pcln \pcreturn \bot \< \< \pcreturn \tau
    % 	\pcln \pcreturn (\adapt(\presig_\hb^H, s),\sigma_\hb^B) \< \< \pcreturn (\Pi, ((Y,c),(Y',c')))
    }
    \end{pchstack}
\end{minipage}
    \caption{Puzzle promise protocol of \aal}
    \label{fig:a2l_promise}
    \end{figure*}

\paragraph{Puzzle Promise} 
In the promise phase (\Cref{fig:a2l_promise}), 
Hub generates a pre-signature $\presig_\hb^H$ on a common message $ m_\hb$ with respect to a uniformly sampled instance $Y := g^s$. Hub also encrypts the witness $s$ in the ciphertext $c \gets \Pi_\ENC.\enc(\ek_H,s)$ under its own encryption key $\ek_H$. Hub gives Bob the tuple $(Y, c, \pi, \presig_\hb^H)$, where $\pi$ is a NIZK proof that certifies the ciphertext $c$ encrypts $s$.
Bob verifies that the NIZK proof and the pre-signature are indeed valid. If so, he chooses a random $r \sample \ZZ_q$ and re-randomizes the instance $Y$ to $Y' := Y \cdot g^r$ and also re-randomizes the ciphertext $c$ as $c' \gets \Pi_\ENC.\mathsf{Rand}(c,r)$. The puzzle is set to $\tau := (r, m_\hb, \presig_\hb^H, (Y,c), (Y', c'))$. 

\begin{figure*}
    \hspace{-5em}
    \begin{minipage}{\textwidth}
    \begin{pchstack}[boxed]
    \procedure{Public parameters: group description $(\GG, g, q)$, message $m_\ah$}{
        \text{$\Pay \langle A(\sk_\ah^A, \ek_H, \tau), \cdot \rangle$} \< \< \text{$\Pay \langle \cdot, H(\dk_H, \vk_\ah^A) \rangle$} \\[][\hline]
        \pcln \mathbf{Parse}\ \tau := (\cdot, \cdot, \cdot, \cdot,(Y',c')) \\ [-0.15\baselineskip][]
        \pcln r' \sample \Z_p, Y'' := Y' \cdot g^{r'} \< \< \\ [-0.15\baselineskip][]
        \pcln c'' \gets \Pi_\ENC.\mathsf{Rand}(c',r') \< \< \\ [-0.15\baselineskip][]
        \pcln \presig_\ah^A \gets \Pi_\ADP.\presig(\sk_\ah^A, m_\ah, Y'') \< \< \\ [-0.85\baselineskip][]
        \pcln \< \sendmessageright*{Y'',c'', \presig_\ah^A} \< \\ [-0.55\baselineskip][]
        \pcln \< \< s'' \gets \Pi_\ENC.\dec(\dk_H, c'') \\ [-0.15\baselineskip][]
        \pcln \< \< \sigma_\ah^A \gets \Pi_\ADP.\adapt(\presig_\ah^A, s'') \\ [-0.15\baselineskip][]
    %	\pcln \< \< \sigma_\ah^H \gets \Pi_\DS.\sign(\sk_\ah^H, m_\ah) \\ [-0.15\baselineskip][]
        \pcln \< \< \mathbf{If}\ \Pi_\ADP.\vrfy(\vk_\ah^A, m_\ah, \sigma_\ah^A) \neq 1\ \mathbf{then} \\ [-0.15\baselineskip][]
        \pcln \< \< \quad \pcreturn \bot \\ [-0.15\baselineskip][]
        %\pcln \< \< \text{Else publish } (\sigma_\ah^A, \sigma_\ah^H) \\ [-0.85\baselineskip][]
        \pcln \< \sendmessageleft*{\sigma_\ah^A} \< \\ [-0.55\baselineskip][]
        \pcln s'' \gets \Pi_\ADP.\ext(\sigma_\ah^A, \presig_\ah^A, Y'')  \\ [-0.15\baselineskip][]
        \pcln \mathbf{If}\ s'' = \bot\ \mathbf{then}\ \pcreturn \bot \\ [-0.15\baselineskip][]
        \pcln s' := s'' - r'  \< \< \\ [-0.15\baselineskip][]
        \pcln \pcreturn (\sigma_\ah^A, s') \< \< \pcreturn \sigma_\ah^A
    }
    \end{pchstack}
    \end{minipage}
    \caption{Puzzle solver protocol of \aal}
    \label{fig:a2l_payment}
    \end{figure*}

\paragraph{Puzzle Solve}
Bob sends the puzzle $\tau$ privately to Alice, who now executes the puzzle solve protocol with Hub 
(\Cref{fig:a2l_payment}).
Alice samples a random $r'$ and further re-randomizes the instance $Y'$ as $Y'' := Y' \cdot g^{r'}$ and the ciphertext $c'$ as $c'' \gets \Pi_\ENC.\mathsf{Rand}(c', r')$. 
She then generates a pre-signature $\presig_\ah^A$ on a common message $m_\ah$ with respect to the instance $Y''$. She sends the tuple $(Y'', c'', \presig_\ah^A)$ to Hub, who decrypts $c''$ using the decryption key $\dk_H$ to obtain $s''$. 
Hub then adapts the pre-signature $\presig_\ah^A$ to $ \sigma_\ah^A$ using $s''$ and ensures its validity. It then sends the signature $\sigma_\ah^A$ to Alice, who extracts the witness for $Y''$ as $ s'' \gets \Pi_\ADP.\ext(\presig_\ah^A, \sigma_\ah^A, Y'')$. 
Alice removes the re-randomization factor to obtain the solution $s' := s'' - r'$ for the instance $Y'$. Alice finally sends $s'$ privately to Bob, who opens the puzzle $\tau$ by computing the witness $s := s' - r$ and adapting the pre-signature $\presig_\hb^H$ (given by Hub in the promise phase) to the signature $\sigma_\hb^H$.

\begin{figure}[tb]
    \centering
    \begin{pchstack}[boxed]
         \procedure[mode=text]{{$\Open(\tau, s')$}}{
     $\mathbf{Parse}\ \tau := (r, \cdot, \presig, \cdot,\cdot)$\\
     $s := s' - r $\\
     $\sigma \gets \Pi_\ADP.\adapt(\presig, s)$\\ 
     $\pcreturn \sigma$
    }
    \end{pchstack}
    \caption{Open algorithm of \aal}
    \label{fig:a2l_openverif}
    \end{figure}

% \chapter{Randomizable Puzzles and Homomorphic Encryption}
\subsection{Randomizable Puzzles and Homomorphic Encryption}
\label{sec:randpuzzle}

Here we recall the definitions of randomizable puzzles~\cite{SP:TaiMorMaf21} and we show that they are trivially satisfied by a CPA-secure homomorphic encryption sceme (over $\mathbb{Z}_p$), with statistical circuit privacy~\cite{C:OstPasPas14}. We recall the syntax as defined in~\cite{SP:TaiMorMaf21}.

\begin{definition}[Randomizable Puzzle]
A randomizable puzzle scheme $\RP = (\psetup, \pgen, \psolve, \prand)$ with a solution space $\solspace$ (and a function $\phi$ acting on $\solspace$) consists of four algorithms defined as:
\begin{description}
	\item[$(\pparam, \td) \gets \psetup(\secparam)$:] is a PPT algorithm that on input security parameter $\secparam$, outputs public parameters $\pparam$ and a trapdoor $\td$.
	\item[$Z \gets \pgen(\pparam, \zeta)$:] is a PPT algorithm that on input public parameters $\pparam$ and a puzzle solution $\zeta$, outputs a puzzle $Z$.
	\item[$\zeta := \psolve(\td, Z)$:] is a deterministic polynomial-time algorithm that on input a trapdoor $\td$ and puzzle $Z$, outputs a puzzle solution $\zeta$.
	\item[$(Z', r) \gets \prand(\pparam, Z)$:] is a PPT algorithm that on input public parameters $\pparam$ and a puzzle $Z$ (which has a solution $\zeta$), outputs a randomization factor $r$ and a randomized puzzle $Z'$ (which has a solution $\phi(\zeta, r)$).
	%\item[$\pverif(\pparam, Z, \zeta)$:] is a $\dpt$ algorithm that on input the public parameters $\pparam$, a puzzle $Z$ and a puzzle solution $\zeta$, outputs a bit $b$.
\end{description}
\end{definition}
It is not hard to see that a linearly homomorphic encryption scheme $(\kgen,\allowbreak \enc, \dec)$ matches the syntax of a randomizable puzzle, setting $\pparam$ to the encryption key and $\td$ to be the decryption key. For the $\prand$ algorithm, we can sample a random $r \sample \mathbb{Z}_p$ and compute
$$
\enc(\ek, \zeta) \circ \enc(\ek, r) = c 
$$
which is an encryption of $\phi(\zeta, r) = \zeta + r$. Next we recall the definition of security for randomizable puzzles.
\begin{definition}[Security]
\label{def:rp-sec}
A randomizable puzzle scheme $\RP$ is secure, if there exists a negligible function 
$\negl[]$, such that
\[
	\Pr\left[
	\zeta \gets \adv(\pparam, Z)
	~\middle\vert
	\begin{array}{l}
	(\pparam, \td) \gets \psetup(\secparam) \\
	\zeta \sample \solspace, Z \gets \pgen(\pparam, \zeta)
	\end{array}
	\right] \leq \negl[\secpar].
\]
\end{definition}
This follows as an immediate application of CPA-security (in fact, even the weaker one-wayness suffices) of the encryption scheme. Finally we recall the notion of privacy.
\begin{definition}[Privacy]
\label{def:rp-privacy}
A randomizable puzzle scheme $\RP$ is private if for every PPT adversary $\adv$ there exists a negligible function $\negl[]$ such that:
$$\Pr[\RPUnlink_{\adv, \RP}(\secpar) = 1] \leq 1/2 + \negl[\secpar]$$ where the experiment 
$\RPUnlink_{\adv, \RP}$ is defined as follows: 
\begin{itemize}
    \item $(\pparam,\td) \gets \psetup(\secparam)$
	\item $((Z_0, \zeta_0), (Z_1, \zeta_1)) \gets \adv(\pparam, \td)$
	\item $b \sample \bin$
	\item $(Z_0', r_0) \gets \prand(\pparam, Z_0)$ 
	\item $(Z_1', r_1) \gets \prand(\pparam, Z_1)$
	\item $b' \gets \adv(\pparam, \td, Z_b')$
	\item Return $\psolve(\td, Z_0) = \zeta_0 \land \psolve(\td, Z_1) = \zeta_1 \\ \land b = b'$
\end{itemize}
\end{definition}
Recall that circuit privacy implies that the distribution induced by 
$\enc(\ek,\allowbreak \zeta) \circ \enc(\ek, r)$ is statistically close to that induced by a a fresh encryption $\enc(\ek, \zeta + r)$. This implies that privacy is satisfied in a statistical sense. Thus we can state the following.
\begin{lemma}
Assuming that $(\kgen, \enc, \dec)$ is a linearly homomorphic encryption with statistical circuit privacy, the there exists a randomizable puzzle with statistical privacy.
\end{lemma}