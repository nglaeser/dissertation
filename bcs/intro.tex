Bitcoin and cryptocurrencies sharing Bitcoin's core principles have attained huge prominence as decentralized and publicly verifiable payment systems. 
They have attracted not only cryptocurrency enthusiasts but also banks~\cite{bankDigitalEuro2021}, leading  IT companies (e.g., Facebook and PayPal), and payment providers such as Visa~\cite{ARXIV:CEGKKM21}. 
At the same time, the initial perception of payment unlinkability based on pseudonyms has been refuted in numerous academic research works~\cite{IMC:MPJLMV16,ARXIV:SHMM21}, and the blockchain surveillance industry~\cite{ARXIV:HSRK21} demonstrates this privacy breach in practice. 
This has led to a large amount of work devoted to providing a privacy-preserving overlay to Bitcoin in the form of \emph{coin mixing} protocols~\cite{FC:BBSU12,ARES:GheFdhWei22}. 

Decentralized coin mixing protocols such as CoinJoin~\cite{coinjoin} or CoinShuffle~\cite{ESORICS:RufMorKat14,EPRINT:RufMorKat16,FCW:RufMor17} allow a set of mutually distrusting users to mix their coins to achieve \emph{unlinkability}: that is, the coins cannot be linked to their initial owners even by malicious participants. 
These protocols suffer from a common drawback, the \emph{bootstrapping problem}, i.e., how to find a set of participants to execute the protocol. In fact, while a high number of participants is desirable to improve the anonymity guarantees provided by the coin mixing protocol, such a high number is at the same time undesirable as it results in poor scalability and  makes bootstrapping harder.

An alternative mechanism is one in which a third party, referred to as the \emph{hub}, alleviates the bootstrapping problem by connecting users that want to mix their coins. 
Moreover, the hub itself can provide a coin mixing service by acting as a tumbler. In more detail, users send their coins to the hub, which, after collecting all the coins, sends them back to the users in a randomized order, thereby providing unlinkability for an observer of such transfers (e.g., an observer of the corresponding Bitcoin transactions).

\paragraph{Synchronization Puzzles} There are numerous reported cases of ``exit scams'' by mixing services which took in new payments but stopped providing the mixing service~\cite{exitscam}. 
This has prompted the design of numerous cryptographic protocols~\cite{FC:BNMCKF14,FCW:ValRow15,coinswap,FCW:HeiBalGol16} to remove trust from the hub, providing a trade-off between trust assumptions, minimum number of transactions, and Bitcoin compatibility~\cite{NDSS:HABSG17}. Of particular interest is the work by Heilman et al.~\cite{NDSS:HABSG17}, which lays the groundwork for the core cryptographic primitive which can be used to build a mixing service. This primitive, referred to as a \emph{\syncpuzzle}, enables unlinkability from even the view of a corrupt hub. However, Heilman et al. only present informal descriptions of the security and privacy notions of interest. Furthermore, the protocol proposed (TumbleBit) relies on hashed time-lock contracts (HTLCs), a smart contract incompatible with major cryptocurrencies such as Monero, Stellar, Ripple, MimbleWimble, and Zerocash (shielded addresses), lowering the interoperability of the solution. 

The recent work of Tairi et al.~\cite{SP:TaiMorMaf21} attempts to overcome both of these limitations. It gives formal security notions for a \syncpuzzle in the universal composability (UC) framework~\cite{FOCS:Canetti01}. It also  provides an instantiation of the \syncpuzzle (called \aal) that is simultaneously more efficient and more interoperable than TumbleBit, requiring only timelocks and digital signature verification from the underlying cryptocurrencies. 

In this work, we identify a gap in their security analysis, and we substantiate the issue by presenting two concrete counterexamples: there exist two encryption schemes (secure under standard cryptographic assumptions) that satisfy the prerequisites of their security notions, yet yield completely insecure systems. This shows that our understanding of {\syncpuzzle}s as a cryptographic primitive is still inadequate. Establishing firm foundations for this important cryptographic primitive requires us to rethink this object from the ground up.