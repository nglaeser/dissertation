\section{Blind Conditional Signatures}\label{sec:bcs-defs}

Next, we formally define and instantiate blind conditional signatures, the central cryptographic notion for coin mixing services. 
Our goal here is to give a simple and easy-to-understand formalization of a \syncpuzzle. 

% \subsection{Definitions}
A blind conditional signature (BCS) is executed among users Alice, Bob, and Hub. The interfaces and associated security properties are defined below.
% \todo{Notation note: \aal paper uses $m'$ for the promise and $m$ for the solver, which is the opposite of what we do here. Personally I think something like $m_\hb$ and $m_\ah$ (or $m^\hb,m^\ah$) is clearer.}

\begin{definition}[Blind Conditional Signature]
A blind conditional signature $\Pi_\mathsf{BCS} :=(\Setup, \Promise, \Pay, \open)$ is defined with respect to a signature scheme $\Pi_\DS := (\kgen, \sign, \vrfy)$ and consists of the following efficient algorithms.

\begin{itemize}
    \item {$(\tilde{\ek}, \tilde{\dk})\gets \Setup(\secparam)$}: The setup algorithm takes as input the security parameter $\secparam$ and outputs a key pair $(\tilde{\ek}, \tilde{\dk})$.
    \item {$(\bot, \{\tau, \bot\}) \gets \Promise \left\langle \begin{matrix} H \left(\tilde{\dk}, \sk^H, m_\hb \right)\\ \bob \left(\tilde{\ek}, \vk^H, m_\hb \right) \end{matrix} \right\rangle $}: The puzzle promise algorithm is an interactive protocol between two users $H$ (with inputs the decryption key $\tilde \dk$, the signing key $\sk^H$, and a message $m_\hb$) and $\bob$ (with inputs the encryption key $\tilde \ek$, the verification key $\vk^H$, and a message $m_\hb$)  and returns $\bot$ to $H$ and either a puzzle $\tau$ or $\bot$ to $B$.
    \item {$(\{(\sigma^*, s), \bot\}, \{\sigma^*, \bot\}) \gets \Pay \left\langle \begin{matrix} \alice \left(\sk^A, \tilde{\ek}, m_\ah, \tau\right)\\ H \left(\tilde{\dk}, \vk^A, m_\ah\right) \end{matrix} \right\rangle $}: The puzzle solving algorithm is an interactive protocol between two users $\alice$ (with inputs the signing key $\sk^A$, the encryption key $\tilde \ek$, a message $m_\ah$, and a puzzle $\tau$) and $H$ (with inputs the decryption key $\tilde \dk$, the verification key $\pk^A$, and a message $m_\ah$) and returns to both users either a signature $\sigma^*$ ($\alice$ additionally receives a secret $s$) or $\bot$.
    \item {$\{\sigma, \bot\}\gets \open(\tau, s)$}: The open algorithm takes as input a puzzle $\tau$ and a secret $s$ and returns a signature $\sigma$ or $\bot$.
\end{itemize}
\end{definition}

Next, we define correctness.

\begin{definition}[Correctness]
A blind conditional signature $\Pi_\mathsf{BCS}$ is correct if for all $\secpar\in \mathbb{N}$, all $(\tilde{\ek}, \tilde{\dk})$ in the support of $\Setup(\secparam)$, all $(\vk^H, \sk^H)$ and $(\vk^A, \sk^A)$ in the support of $\Pi_\DS.\kgen(\secparam)$, and all pairs of messages $(m_\hb, m_\ah)$, it holds that
$$
\Pr\left[\vrfy(\vk^H,m_\hb, \open(\tau, s)) = 1\right] = 1
$$
and
$$
\Pr\left[\vrfy(\vk^A,m_\ah, \sigma^*) = 1\right] = 1
$$
where 
\begin{itemize}
    \item $\tau \gets \Promise \left\langle \begin{matrix} H \left(\tilde{\dk}, \sk^H, m_\hb \right)\\ \bob \left(\tilde{\ek}, \vk^H, m_\hb\right) \end{matrix} \right\rangle$ and
    \item $((\sigma^*, s), \sigma^*) \gets \Pay \left\langle \begin{matrix} \alice \left(\sk^A,  \tilde{\ek}, m_\ah, \tau\right)\\ H \left(\tilde{\dk}, \vk^A, m_\ah\right) \end{matrix} \right\rangle$.
\end{itemize}
\end{definition}

We now present the security guarantees of BCS in the game-based setting.
Our definition of blindness is akin to the strong blindness notion of standard blind signatures~\cite{C:Chaum82}, in which the adversary picks the keys (as opposed to the weak version in which they are chosen by the experiment)\footnote{We opt for this stronger version since we want to provide anonymity even in the case of a fully malicious hub, which can pick its keys adversarially to try to link a sender/receiver pair.}. Roughly speaking, it says that two promise/solve sessions cannot be linked together by the hub.\footnote{We do not consider the case in which Hub colludes with either Alice or Bob, since deanonymization is trivial (Alice (resp. Bob) simply reveals the identity of Bob (resp. Alice) to Hub); this is in line with \cite{SP:TaiMorMaf21}.}

\begin{definition}[Blindness]
A blind conditional signature $\Pi_\mathsf{BCS}$ is blind if there exists a negligible function $\negl$ such that for all $\secpar \in \NN$ and all PPT adversaries $\adv$, the following holds:
\[ \prob{\expUnlink^{\adv}_{\Pi_{\mathsf{puzzle}}}(\secpar) = 1} \le \frac{1}{2} + \negl\]
where $\expUnlink$ is defined in~\Cref{fig:exp_unlinkability}.\footnote{In previous works, descriptions of unlinkability assume an explicit step for blinding the puzzle $\tau$ between $\Promise$ and $\Pay$. Here, we assume that $\Pay$ performs this blinding functionality.}
\end{definition}

\begin{figure}[tbh]
    \centering
    \begin{pchstack}[boxed]
    \procedure[space=auto,codesize=\small]{$\expUnlink^{\adv}_{\Pi_{\mathsf{BCS}}}(\secpar)$}{
    \hfill\\[-0.5\baselineskip]
   (\tilde{\ek}, \vk^H_0, \vk^H_1, (m_{\hb,0}, m_{\ah,0}), (m_{\hb,1}, m_{\ah,1})) \gets \adv(\secparam)\\
    (\vk_0^A, \sk_0^A) \gets \kgen(\secparam)\\
    (\vk_1^A, \sk_1^A) \gets \kgen(\secparam)\\
    \tau_0 \gets \Promise \left\langle  \adv(\vk_0^A, \vk_1^A), \bob(\tilde{\ek}, \vk^H_0, m_{\hb,0}) \right\rangle\\
    \tau_1 \gets \Promise \left\langle  \adv(\vk_0^A, \vk_1^A), \bob(\tilde{\ek}, \vk^H_1, m_{\hb,1}) \right\rangle\\
    b \gets \{0,1\}\\
    (\sigma^*_0, s_0) \gets \Pay \left\langle  \alice \left(\sk_0^A, \tilde{\ek}, m_{\ah,0}, \tau_{0 \oplus b}\right), \adv \right\rangle \\
    (\sigma^*_1, s_1) \gets \Pay \left\langle  \alice \left(\sk_1^A, \tilde{\ek}, m_{\ah,1}, \tau_{1 \oplus b}\right), \adv \right\rangle \\
    \mathbf{if~}(\sigma^*_0 = \bot) \lor (\sigma^*_1 = \bot) \lor (\tau_0 = \bot) \lor (\tau_1 = \bot)\\
    \quad \sigma_0 := \sigma_1 := \bot\\
    \mathbf{else}\\
    \quad \sigma_{0 \oplus b} \gets \open(\tau_{0 \oplus b}, s_0)\\
    \quad \sigma_{1 \oplus b} \gets \open(\tau_{1 \oplus b}, s_1)\\
    b' \gets \adv(\sigma_0, \sigma_1)\\
    \mathbf{return}\ (b = b')
    }
\end{pchstack}
\caption{Blindness experiment \label{fig:exp_unlinkability}}
\end{figure}

Next, we define unlockability, which says that it should be hard for Hub to create a valid signature on Alice's message that does not allow Bob to unlock the full signature in the corresponding promise session.

\begin{definition}[Unlockability]
A blind conditional signature $\Pi_\mathsf{BCS}$ is unlockable if there exists a negligible function $\negl$ such that for all $\secpar \in \NN$ and all PPT adversaries $\adv$, the following holds:
\[ \prob{\expUnlock^{\adv}_{\Pi_{\mathsf{BCS}}}(\secpar) = 1} \le \negl\]
where $\expUnlock$ is defined in~\Cref{fig:exp_unlockability}.
\end{definition}

\begin{figure}[tbh]
    \centering
    \begin{pchstack}[boxed]
    \procedure[space=auto,codesize=\small]{$\expUnlock^{\adv}_{\Pi_{\mathsf{BCS}}}(\secpar)$}{
    \hfill\\[-0.5\baselineskip]
    (\tilde{\ek}, \vk^H, m_\hb, m_\ah) \gets \adv(\secparam)\\
    (\vk^A, \sk^A) \gets \kgen(\secparam)\\
    \tau \gets \Promise \left\langle  \adv(\vk^A), \bob(\tilde{\ek}, \vk^H, m_\hb) \right\rangle\\
    \textbf{if } \tau = \bot\\
    \quad (\hat{\sigma}, \hat{m}) \gets \adv\\ 
    \quad b_0 := (\vrfy(\vk^A,\hat{\sigma}, \hat{m}) =1)\\
    \textbf{if } \tau \neq \bot\\
        \quad (\sigma^*,s) \gets \Pay \left\langle  \alice \left(\sk^A, \tilde{\ek}, m_\ah,\tau\right), \adv \right\rangle\\
    \quad (\hat{\sigma}, \hat{m}) \gets \adv\\ 
    \quad b_1 := (\vrfy(\vk^A,\hat{\sigma}, \hat{m}) =1) \land (\hat{m} \neq m_\ah)\\
     \quad b_2 := (\vrfy(\vk^A,\sigma^*, m_\ah) =1)\\
    \quad b_3 := (\vrfy(\vk^H, m_\hb, \open(\tau,s)) \neq1) \\
     \mathbf{return}\ b_0 \lor b_1 \lor (b_2 \land b_3)
    }
\end{pchstack}
    \caption{Unlockability experiment} 
    \label{fig:exp_unlockability}
\end{figure}


Our definition of unforgeability is inspired by the unforgeability of blind signatures~\cite{C:Chaum82}. We require that Alice and Bob cannot recover $q$ signatures from Hub while successfully querying the solving oracle at most $q-1$ times. Since each successful query reveals a signature from Alice's key (which in turn corresponds to a transaction from Alice to Hub), this requirement implicitly captures the fact that Alice and Bob cannot steal coins from Hub. The winning condition $b_0$ captures the scenario where the adversary forges a signature of the hub on a message previously not used in any promise oracle query.
The remaining conditions $b_1, b_2$ and $b_3$ together capture the scenario in which the adversary outputs $q$ valid distinct key-message-signature tuples while having queried for solve only $q-1$ times. Hence, in the second condition, the attacker manages to \emph{complete} $q$ promise interactions with only $q-1$ solve interactions, whereas in the first winning condition, the adversary computes a \emph{fresh} signature that is not the completion of any promise interaction. These conditions are technically incomparable: an attacker that succeeds under one condition does not imply an attacker succeeding on the other.
It is important to note that this is different from the unforgeability notion of blind signatures (where the attacker only has access to a single signing oracle), since in our case the hub is offering the attacker two oracles: promise and solve.

\begin{definition}[Unforgeability]
A blind conditional signature $\Pi_\mathsf{BCS}$ is \emph{unforgeable} if there exists a negligible function $\negl$ such that for all $\secpar \in \NN$ and all PPT adversaries $\adv$, the following holds:
\begin{equation*}
     \prob{\expSec^{\adv}_{\Pi_{\mathsf{BCS}}}(\secpar) = 1} \le \negl
\end{equation*}
where $\expSec$ is defined in~\Cref{fig:exp_security_ab}.
\end{definition}

\begin{figure}[tbh]
    \centering
    \begin{pchstack}[boxed]
    \begin{pcvstack}
    \procedure[space=auto,codesize=\small]{$\expSec^{\adv}_{\Pi_{\mathsf{BCS}}}(\secpar)$}{
    \hfill\\[-0.5\baselineskip]
\LL := \emptyset, Q := 0 \\
(\tilde{\ek}, \tilde{\dk}) \gets \Setup(\secparam)\\
(\vk_1^H, m_1, \sigma_1), \ldots, (\vk_q^H, m_q, \sigma_q)  \gets \adv^{\oracle\mathsf{PP}(\cdot), \oracle\mathsf{PS}(\cdot)}(\tilde{\ek})\\
b_0 := \exists i \in [q] \suchthat (\vk_i^H,\cdot) \in \LL \land (\vk_i^H,m_i) \notin \LL \\
\qquad \qquad  \land \vrfy(\vk_i^H,m_i,\sigma_i) = 1\\
b_1 := \forall i \in [q], (\vk_i^H,m_i) \in \LL \land \vrfy(\vk_i^H, m_i, \sigma_i) = 1\\
b_2 := \bigwedge_{i,j \in [q], i \ne j}  (\vk_i^H, m_i, \sigma_i) \ne (\vk_j^H, m_j, \sigma_j) \\
b_3 := (Q \le q-1) \\
\mathbf{return}\ b_0 \lor (b_1 \land b_2 \land b_3)
}
\pcvspace

\procedure[space=auto,codesize=\small]{$\oracle\mathsf{PP}(m)$}{
\hfill\\[-0.5\baselineskip]
(\vk^H, \sk^H) \gets \Pi_\ADP.\kgen(\secparam)\\
\LL := \LL \cup \{ (\vk^H,m) \}\\
\bot \gets \Promise\langle H(\tilde{\dk}, \sk^H, m), \adv(\vk^H) \rangle \\
}


\procedure[space=auto,codesize=\small]{$\oracle\mathsf{PS}(\vk^A, m')$}{
\hfill\\[-0.5\baselineskip]
\sigma^* \gets \Pay\langle \adv, H(\tilde{\dk}, \vk^A, m') \rangle \\
\mathbf{if}\ \sigma^* \ne \bot\ \mathbf{then}~ Q:= Q+ 1
}
    \end{pcvstack}
\end{pchstack}
    \caption{Unforgeability experiment}
    \label{fig:exp_security_ab}
\end{figure}

We define security as the collection of all properties.
\begin{definition}[Security]
A blind conditional signature $\Pi_\mathsf{BCS}$ is secure if it is blind, unlockable, and unforgeable.
\end{definition}