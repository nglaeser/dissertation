\section{Additional Preliminaries}

In this chapter, we require that any digital signature scheme $\Pi_\DS := (\kgen, \sign,\allowbreak \vrfy)$ satisfies the standard notion of strong existential unforgeability~\cite{GolMicRiv88}. We require any non-interactive proof system $\NIZK$ (see \Cref{sec:nizks}) to be (1) \emph{zero-knowledge}, i.e., there exists a simulator $\pi \gets \mathsf{Sim}(\td, x)$ that computes valid proofs without the knowledge of the witness, (2) \emph{sound}, i.e., it is infeasible for an adversary to output a valid proof for a statement $x \notin \Lang$ (see \Cref{def:soundness}), and (3) \emph{UC-secure}, i.e., one can efficiently extract from the proofs computed by the adversary a valid witness (with the knowledge of the setup trapdoor $\td$), even in the presence of simulated proofs. For formal security definitions, we refer the reader to~\cite{C:DeSMicPer87,AC:CamKreSho11}.

% \paragraph{Hard Relations} 
% We recall the notion of a hard
% relation $\Rel$ with statement-witness pairs $(Y, y)$. We denote by $\Lang_\Rel$ the associated language defined as $\Lang_\Rel := \{ Y : \exists y,\ (Y, y) \in \Rel \}$. The relation is called a hard relation if the following holds: 
In this chapter, we will deal with so-called \emph{hard relations} $\Rel$ of statement-witness pairs $(Y,y)$ which are decidable in polynomial time (see \Cref{sec:nizks}) but where, for all PPT adversaries $\adv$, the probability of $\adv$ on input $Y$ outputting a witness $y$ is negligible. We will also require a PPT sampling algorithm $\genr(\secparam)$ that outputs a statement-witness pair $(Y, y) \in \Rel$.
In particular, we use the discrete log relation $\Rel_\DL$ defined with respect to a group $\GG$ with generator $g$ and order $p$. The corresponding language is defined as $\Lang_\DL := \{ Y : \exists y \in \ZZ_p,\ Y = g^y \}$.

\subsection{Adaptor Signatures} 
Adaptor signatures~\cite{AC:AEEFHMMR21} let users generate a pre-signature on a message $m$ which by itself is not a valid signature, but can later be adapted into a valid signature using knowledge of some secret value. More precisely, an adaptor signature scheme $\Pi_\ADP:= (\kgen, \presign, \prevrfy,\adapt,\allowbreak \vrfy,\ext)$ is defined with respect to a signature scheme $\Pi_\DS$ and a hard relation $\Rel$. The key generation algorithm is the same as in $\Pi_\DS$ and outputs a key pair $(\vk, \sk)$. The pre-signing algorithm $\presign(\sk, m, Y)$ returns a pre-signature $\presig$ (we sometimes also refer to this as a partial signature). The pre-signature verification algorithm $\prevrfy(\vk, m, Y, \presig)$ verifies if the pre-signature $\presig$ is correctly generated. 
The adapt algorithm $\adapt(\presig, y)$ transforms a pre-signature $\presig$ into a valid signature $\sigma$ given the witness $y$ for the instance $Y$ of the language $\Lang_\Rel$.
The verification algorithm $\vrfy$ is the same as in $\Pi_\DS$. Finally, we have the extract algorithm $\ext(\presig, \sigma, Y)$ which, given a pre-signature $\presig$, a signature $\sigma$, and an instance $Y$, outputs the witness $y$ for $Y$. This can be formalized as \emph{pre-signature correctness}.

\begin{definition}[Pre-signature Correctness]
\label{def:as-pre-correct}
	An adaptor signature scheme $\Pi_\ADP$ satisfies pre-signature correctness 
	if for every $\secpar \in \NN$, every message $m \in \bin^*$, and every statement/witness pair 
	$(Y, y) \in \Rel$, the following holds:
	$$
	\prob{
	\begin{array}{c}
	\prevrfy(\vk, m, Y, \presig) = 1\\
	\land \\
	\vrfy(\vk, m, \sigma) = 1\\
	\land \\
	(Y, y') \in R
	\end{array}
	\middle\vert
	\begin{array}{l}
	(\sk, \vk) \gets \kgen(\secparam)\\
	\presig \gets \presign(\sk, m, Y) \\
	\sigma := \adapt(\presig,y)\\
	y' := \ext(\sigma, \presig, Y)
	\end{array}
	 }=1.
	$$
\end{definition}

In terms of security, we want standard unforgeability even when the adversary is given access to pre-signatures with respect to the signing key $\sk$.

\begin{definition}[Unforgeability]
\label{def:aeufcma}
	An adaptor signature scheme $\Pi_\ADP$ is $\aeufcma$ secure if for every PPT adversary $\adv$ there exists a negligible function $\negl[]$ such that
	$$\prob{\ASigForge_{\adv, \Pi_\ADP}(\secpar) = 1} \leq \negl[\secpar],$$
	where the experiment $\ASigForge_{\adv,\Pi_\ADP}$ is defined as in~\Cref{fig:as-unforge}.
    \begin{figure}
        \centering
        \begin{pchstack}[boxed]
            \procedure[space=keep]{$\ASigForge_{\adv,\Pi_\ADP}(\secpar)$}
            {
                    \queryM := \emptyset\\
                    (\sk,\vk) \gets \kgen(\secparam)\\
                    m \gets \adv^{\oracle^\sign(\cdot), \oracle^\presign(\cdot,\cdot)}(\vk) \\
                    (Y, y) \gets  \genr(\secparam)\\
                    \presig \gets \presign(\sk, m,Y)\\
                    \sigma \gets \adv^{\oracle^\sign(\cdot), \oracle^\presign(\cdot,\cdot)}(\presig, Y)\\
                    \pcreturn \left(m \not \in \queryM \land \vrfy(\vk, m, \sigma) \right)}
                
            \pchspace
            
            \begin{pcvstack}
                \begin{subprocedure}
                    \procedure[space=keep]{$\oracle^\sign(m)$}{
                            \sigma \gets \sign(\sk, m)\\
                            \queryM := \queryM \cup \{m\}\\
                            \pcreturn \sigma 	
                    }
                \end{subprocedure}
                \pcvspace[1pt]
                \begin{subprocedure}
        \procedure[space=keep]{$\oracle^\presign(m, Y)$}{
                            \presig \gets \presign(\sk, m, Y)\\
                            \queryM := \queryM \cup \{m\}\\
                            \pcreturn \presig 
                    }
                \end{subprocedure}
            \end{pcvstack}
    \end{pchstack}
    \caption{Unforgeability experiment of adaptor signatures\label{fig:as-unforge}}
\end{figure}
\end{definition}

We also require that, given a pre-signature and a witness for the instance, one can always adapt the pre-signature into a valid signature (\emph{pre-signature adaptability}). 
 
\begin{definition}[Pre-signature Adaptability]
\label{def:as-adaptability}
	An adaptor signature scheme $\Pi_\ADP$ satisfies pre-signature adaptability 
	if for any $\secpar \in \NN$, any message $m \in \bin^*$, any statement/witness pair $(Y, 
	y) \in R$, any key pair $(\sk,\vk) \gets \kgen(\secparam)$, and any pre-signature 
	$\presig \gets \{0,1\}^*$ with $\prevrfy(\vk, m, Y, \presig) = 1$, we have: 
	\[\prob{\vrfy(\vk,m, \adapt(\presig,y)) = 1} = 1. \]
\end{definition}

Finally, we require that, given a valid pre-signature and a signature with respect to the same instance, one can efficiently extract the corresponding witness (\emph{witness extractability}). 

\begin{definition}[Witness Extractability]
\label{def:as-extractability}
	An adaptor signature scheme $\Pi_\ADP$ is witness extractable if 
	for every PPT adversary $\adv$, there exists a negligible function $\negl[]$ such that 
	\[\prob{\AExtractability_{\adv, \Pi_\ADP}(\secpar) = 1} 
	\leq \negl[\secpar],\] where the experiment $\AExtractability_{\adv, 
	\Pi_\ADP}$ is defined as in \Cref{fig:as-witness-ext}.
    \begin{figure}[ht]
    \centering

    \begin{pchstack}[boxed]
        \procedure[space=keep]{$\AExtractability_{\adv, \Pi_\ADP}(\secpar)$}{
                \queryM := \emptyset\\
                (\sk,\vk) \gets \kgen(\secparam)\\
                (m,Y) \gets \adv^{\oracle^\sign(\cdot), \oracle^\presign(\cdot,\cdot)}(\vk) \\
                \presig \gets \presign(\sk,m,Y)\\
                \sigma \gets \adv^{\oracle^\sign(\cdot), \oracle^\presign(\cdot,\cdot)}(\presig)\\
                y' := \ext( \sigma, \presig, Y)\\
                \pcreturn (m \not \in \queryM \land (Y,y')\not \in R \\
                \land \vrfy(\vk, m, \sigma))
        }
        
        \pchspace
        
        \begin{pcvstack}
            \begin{subprocedure}
                \procedure[space=keep]{$\oracle^\sign(m)$}{
                        \sigma \gets \sign(\sk, m)\\
                        \queryM := \queryM \cup \{m\}\\
                        \pcreturn \sigma 	
                }
            \end{subprocedure}
        
            \pcvspace[5pt]
            \begin{subprocedure}
                \procedure[space=keep]{$\oracle^\presign(m, Y)$}{
                        \presig \gets \presign(\sk, m, Y)\\
                        \queryM := \queryM \cup \{m\}\\
                        \pcreturn \presig 	
                }
            \end{subprocedure}
        \end{pcvstack}
    \end{pchstack}
\caption{Witness extractability experiment for adaptor signatures\label{fig:as-witness-ext}}
\end{figure}
\end{definition}

Combining the three properties described above, we can define a secure adaptor signature scheme 
as follows.

\begin{definition}[Secure Adaptor Signature Scheme]
	An adaptor signature scheme $\Pi_\ADP$ is secure if it is $\aeufcma$ 
	secure, pre-signature adaptable, and witness extractable.
\end{definition}

\subsection{Linear-Only Homomorphic Encryption} 

A public-key encryption scheme $\Pi_\ENC:= (\kgen, \enc, \dec)$ allows one to generate a key pair $(\ek, \dk) \gets \kgen(\secparam)$ that allows anyone to encrypt messages as $c \gets \enc(\ek, m)$ and allows only the owner of the decryption key $\dk$ to decrypt ciphertexts as $m \gets \dec(\dk, c)$. We require that  $\Pi_\ENC$ satisfies perfect correctness and the standard notion of CPA-security~\cite{STOC:GolMic82}.
We say that an encryption scheme is \emph{linearly homomorphic} if there exists some efficiently computable operation $\circ$ such that $\enc(\ek, m_0) \circ \enc(\ek, m_1) \in \enc(\ek, m_0 + m_1)$, where addition is defined over $\mathbb{Z}_p$. The $\alpha$-fold application of $\circ$ is denoted by $\enc(\ek, m)^\alpha$. 

Linear-only encryption (LOE) is an idealized model introduced by Groth~\cite{TCC:Groth04} as ``generic homomorphic cryptosystem''. Here, homomorphic encryption is modeled by giving access to oracles instead of their corresponding algorithms. A formal description of the oracles is given  in~\Cref{o:kgen}. We note that although we do not model such an algorithm explicitly, this model allows for (perfect) ciphertext re-randomization by homomorphically adding $0$ to the desired ciphertext.
\begin{figure}
    \centering

    \begin{pchstack}[boxed]
    \begin{pcvstack}
    \begin{pchstack}
    \procedure[space=auto,codesize=\small]{$\oracle^\kgen(i)$}{
    \ek_i \sample \{0,1\}^\secpar\\
    \text{Enter $(i, \ek_i)$ into table $K$}\\
    \pcreturn \ek_i
    }
    
        \pchspace
        
    \procedure[space=auto,codesize=\small]{$\oracle^\enc(\ek_i, m)$}{
    c_j \sample \{0,1\}^\secpar\\
    \text{Enter $(m,c_j)$ into table $M_i$}\\
    \pcreturn c_j
    }
    \end{pchstack}

    \pcvspace
    
    \procedure[space=auto,codesize=\small]{$\oracle^\dec(\ek_i, c)$}{
    \mathbf{if}\ (\cdot, c) \notin M_i\ \mathbf{then}\ \pcreturn \perp\\
    %\quad \pcreturn \perp\\
    \mathbf{else}\\
    \quad \text{Look up $m$ corresponding to $c$ in $M_i$}\\
    \quad \pcreturn m
    }
    
    \pcvspace
    
    \procedure[space=auto,codesize=\small]{$\oracle^\add(\ek_i,c_0,c_1)$}{
    \text{look up $m_0, m_1$ corresponding to $c_0, c_1$ in table $M_i$}\\
    \tilde{c} \sample \{0,1\}^\secpar\\
    \text{Enter $(m_0+m_1,\tilde{c})$ into table $M_i$}\\
    \pcreturn \tilde{c}
    }
    
    \end{pcvstack}
    \end{pchstack}
    \caption{Linear-only encryption oracles}
    \label{o:kgen}
    \label{o:enc}
    \label{o:dec}
    \label{o:add}
\end{figure}

% \paragraph{One-More DL} 
\subsection{One-More Discrete Logarithm Assumption} 
We recall the one-more discrete logarithm (OMDL) assumption \cite{JC:BNPS03,EPRINT:BauFucPlo21}.

\begin{definition}[One-More Discrete Logarithm (OMDL) Assumption]\label{def:omdl}
Let $\GG$ be a uniformly sampled cyclic group of prime order $p$ and $g$ a random generator of $\GG$. The one-more discrete logarithm (OMDL) assumption states that for all $\secpar \in \NN$ there exists a negligible function $\negl$ such that for all PPT adversaries $\adv$ making at most $q = \poly$ queries to $\DL(\cdot)$, the following holds:
\begin{align*}
\Pr\left[ \begin{array}{c}
   \forall i: x_i = r_i
\end{array}
\middle|
\begin{array}{c}
    r_1, \ldots, r_{q+1} \sample \ZZ_p\\
    \forall i \in [q+1], h_i \gets g^{r_i}\\
    \{x_i \}_{i \in [q+1]} \gets \adv^{\DL(\cdot)}(h_1, \ldots, h_{q+1})
\end{array}
\right]
\le \negl,
\end{align*}
where the $\DL(\cdot)$ oracle takes as input an element $h \in \GG$ and returns $x$ such that $h = g^x$.
\end{definition}