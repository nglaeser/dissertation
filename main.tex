\documentclass{article}
\usepackage{graphicx} % Required for inserting images
\usepackage{xcolor}
\usepackage{dashbox}
\usepackage{tikz}
\usetikzlibrary{patterns}
\usepackage{xspace}
\usepackage{booktabs,multicol}
\usepackage{mdframed}
\usepackage{hyperref}
\hypersetup{
    colorlinks=true,
    linkcolor=blue,
    filecolor=magenta,
    urlcolor=blue,
    citecolor=blue,
}
\usepackage{amsthm}
\newtheorem{definition}{Definition}
\newtheorem{theorem}{Theorem}
\usepackage{amsmath,amsfonts,amssymb}
\usepackage{wasysym}
\usepackage{cleveref}
\usepackage[lambda]{cryptocode}
\usepackage{hyphenat}
\hyphenation{inter-active}

% protocol names
\newcommand{\AAL}{A$^2$L\xspace}
\newcommand{\AALplus}{A$^2$L$^+$\xspace}
\newcommand{\AALUC}{A$^2$L$^{\rm UC}$\xspace}
\newcommand{\COCO}{C\ensuremath{\emptyset}C\ensuremath{\emptyset}}

% checkmarks / x's
\usepackage{pifont}
\newcommand{\cmark}{\ding{51}}%
\newcommand{\xmark}{\ding{55}}%

% variables
% \renewcommand{\epsilon}{\varepsilon}

% math
\newcommand{\ZZ}{\ensuremath{\mathbb{Z}}}

% primitives
\newcommand{\ENC}{\ensuremath{\mathsf{E}}}
\newcommand{\ADP}{\ensuremath{\mathsf{ADP}}}
\newcommand{\DS}{\ensuremath{\mathsf{ADP}}}
\newcommand{\NIZK}{\ensuremath{\mathsf{NIZK}}}
\newcommand{\BCS}{\ensuremath{\mathsf{BCS}}}

%%% cryptography
\newcommand{\secpar}{\ensuremath{\lambda}}
\newcommand{\secparam}{\ensuremath{1^\secpar}}
\newcommand{\sample}{\ensuremath{\stackrel{\$}{\gets}}}
\newcommand{\Rel}{\ensuremath{\mathcal{R}}}
\newcommand{\Lang}{\ensuremath{\mathcal{L}}}
\newcommand{\F}{\ensuremath{\mathcal{F}}}
% oracles
\newcommand{\oracle}{\ensuremath{\mathcal{O}}}
\newcommand{\oaal}{\ensuremath{\oracle^{\sf A^2L}}}
% common algorithms
\newcommand{\kgen}{\ensuremath{\mathsf{KGen}}}
\newcommand{\enc}{\ensuremath{\mathsf{Enc}}}
\newcommand{\dec}{\ensuremath{\mathsf{Dec}}}
\newcommand{\ext}{\ensuremath{\mathsf{Ext}}}
\newcommand{\vrfy}{\ensuremath{\mathsf{Vrfy}}}
% keys
\newcommand{\vk}{\ensuremath{\mathsf{vk}}}
\newcommand{\sk}{\ensuremath{\mathsf{sk}}}
\newcommand{\ek}{\ensuremath{\mathsf{ek}}}
\newcommand{\dk}{\ensuremath{\mathsf{dk}}}
% adaptor sigs
\newcommand{\presign}{\ensuremath{\mathsf{PreSig}}}
\newcommand{\prevrfy}{\ensuremath{\mathsf{PreVrfy}}}
\newcommand{\adapt}{\ensuremath{\mathsf{Adapt}}}
\newcommand{\presig}{\ensuremath{\tilde{\sigma}}}
\newcommand{\copied}[1]{\textcolor{red}{[copied] #1}}
\newcommand{\noemi}[1]{\textcolor{magenta}{Noemi: #1}}
\newcommand{\todo}[1]{\textcolor{red}{todo: #1}}

\title{Cryptography for Private and Secure Blockchain Applications}
% \title{Cryptography for Blockchain Applications}
% \title{Privacy-Enhancing Technologies for Blockchains}
% \title{Improving Privacy and Security of Blockchain with Cryptography}
% \title{Private and Responsible Decentralized Payments} % from MPI contract
\author{Noemi Glaeser}
\date{}

\begin{document}

\maketitle
\begin{abstract}
Blockchains were introduced in 2008 by Satoshi Nakamoto as a way to implement a trusted but decentralized append-only ledger. This simple functionality has given rise to a plethora of decentralized applications utilizing the blockchain as a public bulleting board. In recent years, it has become clear that this basic functionality is not enough to prevent widespread attacks on both the privacy and security of blockchain users, as evidenced by the blockchain analytics industry and the billions of dollars stolen via cryptocurrency exploits to date. 

This work explores the role cryptography has to play in the blockchain ecosystem to both enhance user privacy and secure user funds. I discuss how to generically add universally composable security to any non-interactive zero-knowledge proof in a way that is compatible with an updatable common reference string, thus showing how to strengthen security for any system relying on ZKPs while maintaining minimal trust assumptions; improve the security of a class of coin mixing protocols by giving a formal security treatment of this class and patching the security of an existing, insecure protocol; and show how to construct efficient, non-interactive, and private on-chain protocols for a large class of elections and auctions. Finally, I describe two proposed works: a new threshold wallet construction in a strong threat model and a comparison of on-chain key management approaches.

% Blockchains are inherently public, [but sometimes we want privacy. We need to use crypto to do this. And so on and so forth\dots] \noemi{actually probably need to expand to also include ``security'', since the threshold sigs project isn't really about privacy}
% A key challenge in [my area of research] is [problem I halfway figured out how to solve]. (And so on and so forth\dots)
\end{abstract}

\tableofcontents
\newpage

\textit{(Parts of this section are taken/adapted from \cite{EPRINT:GSZB23}.)}

\paragraph{Special Notation} We will use $n$ as the number of users, $m$ as the number of candidates, and $w$ as the maximum weight to be allocated to any one candidate in a ballot/bid ($n,m,w \in \mathbb{N}$). For simplicity and without loss of generality, we assume the user identities are unique integers $i \in [n]$.
We generally use $i \in [n]$ to index users and $j \in [m]$ for candidates.

The application layer can provide more varied functionalities than simply adding anonymity to payments. This includes on-chain auctions and voting, which are are becoming increasingly requested web3 applications.
Decentralized marketplaces run auctions to sell digital goods like non-fungible tokens (NFTs)~\cite{opensea_auction} or domain names~\cite{ARXIV:XWYLLX21}, while decentralized autonomous organizations (DAOs) deploy voting schemes to enact decentralized governance~\cite{optimismgov}. 
Most auction or voting schemes currently deployed on blockchains, e.g., NFT auctions on OpenSea or Uniswap governance~\cite{ARXIV:FMW22}, lack bid/ballot privacy. This can negatively influence user behavior, for example, by vote herding or discouraging participation~\cite{FC:ElkLip04,WTSC:GalYou18,FC:SuzYok03}. The lack of privacy can cause surges in congestion and transaction fees as users try to outbid each other to participate, a negative externality for the entire network.
    
%-----------------------------BEGIN Approaches Table---------------------------
\begin{table*}[tb]
    \footnotesize
    \centering
    \makebox[\linewidth]{
     \setlength{\tabcolsep}{3pt}
     \setlength{\belowbottomsep}{6pt}
    %  \newcolumntype{R}{>{\begin{turn}{90}\begin{minipage}}r%
    % <{\end{minipage}\end{turn}}%
    % }
    %  \newcolumntype{v}{>{\rotatebox{270}}r<{}}}
     \begin{tabular}{lccccc} 
     \toprule
      \textbf{Approach} & NI        & No TTPs   &Efficient & Tally privacy & Ev. ballot privacy \\
    \midrule 
    Commit-reveal~\cite{AUSC:FOO93,WTSC:GalYou18}              & \xmark                 & \cmark   &\cmark    &\cmark & \xmark     \\ %\color{green}{One-way functions} \\
    Zero-knowledge proofs~\cite{maci}              & \xmark                 & \xmark   &\cmark    &\cmark & \cmark     \\ %\color{green}{One-way functions} \\
    Fully homomorphic TLPs~\cite{C:MalThy19}  & \cmark                 & \cmark   &\xmark    &\cmark & \xmark     \\ %\color{red}{Indistinguishability Obfuscation} \\
    FHE~\cite{STOC:Gentry09,PQCRYPTO:CGGI16,CCS:DLNS17}                              & \cmark                 & \xmark   &\xmark    &\cmark & \cmark     \\ %\color{green}{Bounded Distance Decoding Problem} \\
    %Functional encryption~\cite{boneh2011functional}&\color{green}{$1$}&\cmark&\cmark&\xmark& \\
    Multi-party computation~\cite{FC:BDJNPT06,C:AOZZ15}                        & \xmark                 & \LEFTcircle &\cmark    &\cmark & \cmark     \\ %\color{red}{Trusted third parties}& \\
    TLPs + homomorphic encryption~\cite{ESORICS:CJSS21}                    & \cmark                 & \LEFTcircle   &\cmark    &\cmark & \xmark     \\\midrule
    HTLPs (our approach)                                    & \cmark                 & \cmark$^{*}$   &\cmark     & \cmark & \LEFTcircle \\ %\color{yellow}{Sequentiality of repeated squaring}~\cite{rotem2020generically}\\[1ex]
     \bottomrule
     $^{*}$ when using class groups &&&&&
    \end{tabular}
    }
    \caption{Qualitative comparison of major cryptographic approaches for designing private auction/voting schemes. 
    % (H)TLP stands for (homomorphic) time-lock puzzle. 
    %Our approach can support everlasting privacy while remaining non-interactive and efficient. 
    % No TTPs refer to the absence of trusted third parties. 
    % An asterisk indicates that those schemes can be instantiated with a transparent setup using class groups, cf.~\Cref{sec:feasability}.}
    NI = non-interactive, Ev. = everlasting. \cite{C:AOZZ15,ESORICS:CJSS21} require a trusted setup but no TTP. Everlasting ballot privacy can be added to our approach via an extension discussed in \cite{EPRINT:GSZB23}.
    }
    \label{table:approaches}
    \end{table*}
    %------------------------------END Approaches Table---------------------------

Existing private voting protocols~\cite{maci,plume,rln} achieve privacy at the cost of introducing a trusted authority who is still able to view all submissions.
Alternatively, the only private \emph{and} trustless auction in deployment we are aware of~\cite{ARXIV:XWYLLX21} uses a two-round commit-reveal protocol: in the first round, every party commits to their bid, and in the second round they open the commitments and the winner can be determined.
Other protocols relying on more heavyweight cryptographic building blocks have been proposed in the literature.
We summarize the various approaches for private voting and auctions in \Cref{table:approaches}.
Unfortunately, all of them suffer from at least one of the following limitations, hindering widespread adoption:

\begin{description}
    \item[Interactivity.] Interactivity is a usability hurdle that often causes friction in the protocols' execution. Mandatory bid/ballot reveals are also a target for censorship.     
    %\item[Lack of censorship-resistance]  Voting and auction schemes can be easily censored due to their public nature.
    A malicious party can bribe the block proposers to exclude certain bids or ballots until the auction/voting ends~\cite{ARXIV:PaiResFox23}. %With bid/ballot privacy the cost of censorship can be increased substantially.
    %
    \item[Trusted third party (TTP).] Many schemes use a trusted coordinator to tally submissions during the voting/bidding phase. This introduces a strong assumption which is at odds with the trustless ethos of the blockchain ecosystem.
    %
    \item[Inefficiency.] Introducing more complicated cryptographic primitives such as fully-homomorphic encryption (FHE) introduces additional overheads which can incur extra gas fees or at least extra time for all participants.
\end{description}

% \subsection{Cicada: A framework for private non-interactive on-chain auctions and voting}
\subsection{Our Results}

We introduce Cicada, a general framework for practical, privacy-preserving, and trust-minimized protocols for both auctions and voting. 
Cicada uses time-lock puzzles (TLPs)~\cite{RSW96} to achieve \emph{privacy and non-interactivity} in a trustless and efficient manner.
A TLP allows a party to ``encrypt'' a message to the future. Specifically, to recover the solution, one needs to perform a computation that is believed to be inherently sequential, with a parameterizable number of steps.
Intuitively, the TLPs play the role of commitments to bids/ballots that any party can open after a predefined time, avoiding the reliance on a second \emph{reveal} round. 

% \paragraph{(Homomorphic) Time-lock puzzles}
% A time-lock puzzle (TLP) allows a party to ``encrypt'' a message to the future. Specifically, to recover the solution, one needs to perform a computation that is believed to be inherently sequential, with a parameterizable number of steps.

% \begin{definition}[Time-lock puzzle~\cite{RSW96}] A time-lock puzzle scheme $\sf TLP$ consists of the following three efficient algorithms:
%     \begin{description}
%         \item[$\mathsf{TLP.Setup}(\secparam, \Ttime) \randout \pp$.] The (potentially trusted) setup algorithm takes as input a security parameter $\secparam$ and a difficulty (time) parameter $\Ttime$, and outputs public parameters $\pp$. % (usually a group $\mathbb{G}$ with $\lambda$ bits of security). 
%         % Typically $\mathbb{G}$ is a group of unknown order, e.g., the group $\mathbb{Z}^{*}_N$. 
%         \item[$\mathsf{TLP.Gen}(\pp, s) \randout Z$.] Given a solution $s\in\ZZ$, the puzzle generation algorithm efficiently computes a time-lock puzzle $Z\in\mathbb{G}$.
%         \item[$\mathsf{TLP.Solve}(\pp, Z) \rightarrow s$.] Given a TLP $Z$, the puzzle solving algorithm requires at least $\Ttime$ sequential steps to output the solution $s$.
%     \end{description}
% \end{definition}
% Informally, we say that a TLP scheme is \emph{correct} if $\mathsf{TLP.Gen}$ is efficiently computable and $\tlp.{\sf Solve}$ always recovers the original solution $s$ to a validly constructed puzzle. A TLP scheme is \emph{secure} if $Z$ hides the solution $s$ and no adversary can compute $\mathsf{TLP.Solve}$ in fewer than $\Ttime$ steps with non-negligible probability. For the formal definitions, we refer the reader to~\cite{C:MalThy19}.

Since solving a TLP is computationally intensive, ideally, an efficient protocol would require solving only a sublinear number of TLPs (in the number of voters/bidders). Cicada achieves this via \emph{homomorphic} TLPs (HTLPs): bids/ballots encoded as HTLPs can be ``squashed'' into a sublinear number of TLPs. Fully homomorphic TLPs are not practically efficient, but Malavolta and Thyagarajan~\cite{C:MalThy19} introduced efficient additively and multiplicatively homomorphic TLP constructions which we will describe below. This is clearly enough for simple constructions like first-past-the-post (FPTP) voting, but it remained an open problem how to apply HTLPs to realize more complicated auction and voting protocols, e.g., cumulative voting.
% $\htlp$s allow an evaluator to compute a function $f$ homomorphically on TLPs  $Z_0,\dots,Z_k$ to obtain a puzzle $Z^* = f(Z_0,\dots,Z_k)$ such that for the corresponding puzzle solutions $s_0, \dots, s_k, s^*$, it holds that $f(s_0,\dots,s_k)=s^{*}$. Concretely efficient $\htlp$ constructions are known for addition, multiplication, and XOR. These are already powerful building blocks for first-past-the-post, range and approval voting, which only require adding together votes. 
% Linearly homomorphic TLPs are practical by instantiating them with an elegant scheme derived from the Paillier-encryption scheme~\cite{paillier1999public}.
% ~\cite{malavolta2019homomorphic} showed how to conduct private auctions using a fully-homomorphic TLP. Sadly, fully homomomorphic TLPs are currently not practically efficient and therefore are primarily of theoretical interest. They left devising efficient auction and voting protocols using their techniques to future work. 

We show how to use HTLPs to build practically efficient, private, and non-interactive protocols for a special class of auction and voting schemes where the tallying procedure can be expressed as a linear function. We show that this limited class nonetheless includes many schemes of interest:
\begin{description}
    \item[Majority, approval, range, and cumulative voting.] In majority/FPTP voting, users can cast a $1$ (support) vote for a single candidate (or cause) and give $0$ (oppose) to the others. Approval voting is a slight generalization of binary voting, where users can submit several binary votes for multiple candidates, i.e., the cast ballot $s$ can be seen as $s\in\{0,1\}^{m}$, where $m$ is the number of causes. In a range voting scheme (or score voting), users can give each candidate some weight between $0$ and $w$. A similar scheme is cumulative voting, where users can distribute $w$ votes (points) among the candidates. In each case, each candidate's points are tallied and the candidate with the highest number is declared the winner.
    \item[Ranked-choice voting.] In a ranked-choice voting scheme, voters can signal more fine-grained preferences among $m$ candidates by listing them in order of preference. There are multiple approaches to determining the winner, including single transferrable vote (STV) and instant runoff voting (IRV). In this work, we focus on the simpler Borda count version~\cite{Emerson13}, where each voter can cast $m-1$ points to their first-choice candidate, $m-2$ points to their second-choice candidate, etc., and the candidate with the most points is the winner. Our protocols can be adapted to similar counting functions, such as the Dowdall system~\cite{FraGro14}, via minor modifications.
    \item[Quadratic voting.] In quadratic voting~\cite{AEA:LalWey18}, each user's ballot is a vector $\vec{b} = (b_1, \dots, b_m)$ such that $\langle \vec{b}, \vec{b} \rangle = \lVert \vec{b} \rVert^2_2 = \sum_i b_i^2 \leq w$. Once again, the winner is determined by summing all the ballots and determining the candidate with the most points. 
    % However, proving ballot well-formedness efficiently in this particular case benefits greatly from the novel application of the residue numeral system (RNS) to private voting (see~\Cref{sec:packing}).
\end{description}

    % \item[One-round protocol.] We argue that usability is one of the major challenges of deploying privacy-preserving voting and auction protocols in practice. Multi-round protocols, e.g., commit-reveal-style protocols, have incentive compatibility issues and considerable usability hurdles. We solve these pressing issues with efficient one-round protocols.  

    % \item[Ballot/bid privacy.] Last but not least, we want to achieve ballot/bid privacy. Our approach naturally provides ballot/bid privacy till the end of the voting/bidding phase. Additionally, we show novel cryptographic techniques in~\Cref{sec:everlasting_ballot_privacy} how we can achieve everlasting ballot and bid privacy without sacrificing our two previously stated goals. 

% \def\sp{\hspace{1em}}
% \newcommand{\enf}[1]{\textcolor{green}{#1}} % constraint to enforce
\newcommand{\enf}[1]{#1}
%%------------------- BEGIN Ballot domain TABLE ----------------------
\begin{table*}[tb]
    \centering
\makebox[\linewidth]{
    \setlength{\tabcolsep}{6pt}
    \setlength{\belowbottomsep}{6pt}
    % \begin{tabular}{l|ccc|ccc}
    % \begin{tabular}{r@{\hspace{3pt}}l@{\hspace{12pt}}ccc}
    \begin{tabular}{l@{\hspace{12pt}}ccc}
        % \multirow{2}*{Protocol}& \multirow{2}*{Ballot domain}    & \multirow{2}*{Hamming wt}& \multirow{2}*{$\ell_1$ norm}& \multicolumn{3}{c}{Proof system}\\
        %                        &                                 &                       &                              & no packing & PNS & RNS \\\hline
        \toprule
                        & \textbf{Submission domain}    & \textbf{Hamming wt}   & \textbf{Norm} \\\midrule
        \emph{Voting schemes}  &&& \\
        % \multirow{6}{*}{\rotatebox[origin=c]{90}{\textit{Voting}}} &
        \sp First-past-the-post & \enf{$[0,1]^m$} & \enf{1}       & 1 \\%& \todo{\sf PoKS} \& \cite[\S3]{groth2005voting} & &\\
        \sp Approval    & \enf{$[0,1]^m$} & \enf{$\leq m$}& $\leq m$ \\%& '' && \\
        \sp Range       & \enf{$[0,w]^m$} & $\leq m$& $\leq wm$ \\%& range proof~\cite{arun2022dew} && \\
        \sp Cumulative  & $[0,w]^m$ & $\leq m$      & \enf{$\leq w$} \\%& {\sf PoPos} \& \todo{norm? \cite{groth2005voting}}&& \\
        \sp Ranked-choice (Borda) 
                        & \enf{$\pi([0,m-1])$}& $m-1$     & $m(m-1)/2$ \\%& Borda shuffle proof~\cite[\S6]{groth2005voting} &&\\
        % 
        \sp Quadratic (\Cref{sec:voting_quadratic})   & $[0,\sqrt{w}]^m$& $\leq m$& \enf{$\lVert \vec{b} \rVert_2^2 = \langle \vec{b}, \vec{b} \rangle = w$} \\%& {\sf PoSqDLog} (\cref{fig:squares_proof}) \& \pokcsmon (\cref{fig:pokcsmon_protocol}) && \\\hline
        \midrule
        % \emph{Sealed-bid auction} &&& \\%                             &&& \\
        Single-item sealed-bid auction & $[0, w]$  & \enf{1}     & \enf{$\leq w$} \\%& range proof~\cite{arun2022dew} & - & -\\
        \midrule
        % \emph{Other}              &&& \\%                             &&& \\
        Bayesian truth serum (\ref{sec:voting_bayesian_truth})
                        & $[0,1]^m \times \mathbb{N}^m$ & $1, 1$ & $1, \leq m$ \\
        \bottomrule
    \end{tabular}
}
    \caption{Requirements for the domain, Hamming weight, and norm of a vector $\vec{b}$ in order for it to be a valid submission in various voting/auction schemes. %and the proof system used to enforce them.
    $\pi(S)$ denotes the set of permutations of $S$. The norm is an $\ell_1$ norm unless otherwise specified. $m$ is the number of candidates and $w$ is the maximum weight which can be assigned to any one candidate. %The \enf{green column} indicates the predicate enforced by the proof.
    }
    \label{tab:voting_schemes}
\end{table*}
%%------------------- END Ballot domain TABLE ----------------------


% Malavolta and Thyagarajan~\cite{C:MalThy19} introduce \emph{homomorphic} TLPs (HTLPs). An HTLP is defined with respect to a class $\mathcal{C}$ of circuits which can be homomorphically evaluated over puzzle solutions, i.e., $\htlp.{\sf Eval}(\pp, C, Z_1, \dots, Z_m) \rightarrow Z_*$ where $Z_*$ contains the application of $c \in \mathcal{C}$ to the solutions in $Z_1, \dots, Z_m$. Moving forward, 
% we will use $\boxplus$ for homomorphic addition and $\cdot$ for scalar multiplication of HTLPs. For the homomorphic application of a linear function $f$, we write $f(Z_1, \dots, Z_m)$.

\paragraph{Homomorphic time-lock puzzles.} Malavolta and Thyagarajan~\cite{C:MalThy19} construct two HTLPs with, respectively, linear and multiplicative homomorphisms in groups of unknown order. For our purposes we are only interested in the former, which is based on the Paillier cryptosystem~\cite{EC:Paillier99}. It uses $N=pq$ a strong semiprime, $g \sample \ZZ_N^*$ and $h = g^{2^\Ttime}$, and has solution space $\ZZ_N$. A puzzle $Z$ is constructed as
\begin{equation}\label{eq:paillierHTLP}
(g^r, h^{r \cdot N} (1+N)^s) \in \mathbb{J}_N \times \ZZ_{N^2}^*
\end{equation}
where $\JJ_N$ is the subgroup of $\subseteq \ZZ_N^*$ of elements with Jacobi symbol +1.
To recover $s$, a solver must recompute $h^r = (g^r)^{2^\Ttime}$, which is believed to be an inherently sequential computation in a group of unknown order.

As an alternative, we introduce a novel linear HTLP based on the exponential ElGamal cryptosystem~\cite{EC:CraGenSch97} over a group of unknown order. This construction is more efficient for a small solution space $\mathcal{S} \subset \ZZ_N$, i.e., $\mathcal{S} = \{ s : s \in \JJ_N \land\ s \ll N \}$. Here a puzzle $Z$ is constructed as
\begin{equation}\label{eq:exp_elgamalHTLP}
(g^r, h^r y^s) \in (\ZZ_N^*)^2
\end{equation}
where $g,y \sample \ZZ_N^*$ and again $h = g^{2^\Ttime}$. This scheme is only practical for small $\mathcal{S}$ since, in addition to recomputing $h^r$, recovering $s$ requires brute-forcing the discrete modulus of $y^s$. We discuss the efficiency trade-off between these two constructions in \cite{EPRINT:GSZB23}.

Introducing a homomorphism raises the issue of puzzle malleability, i.e., the possibility of ``mauling'' one puzzle (whose solution may be unknown) into a puzzle with a related solution. This could lead to issues when HTLPs are deployed in larger systems, prompting research into non-malleable TLPs~\cite{TCC:FKPS21}. In our case, we define and enforce non-malleability at the system level (see below).

We will use NIZKs to enforce well-formedness of user submissions while maintaining their secrecy. This prevents users from ``poisoning'' the aggregate HTLP maintained by the on-chain coordinator. For efficiency, we make use of custom NIZKs in groups of unknown order following the approach of \cite{C:BonBunFis19} (see \cite{EPRINT:GSZB23}). 

% Since submissions will be instantiated as HTLPs in our application and all known HTLP constructions use groups of unknown order, our proofs of well-formedness must also operate over these groups.
% Previous ballot correctness proofs~\cite{ACNS:Groth05} and $\Sigma$-protocols generally operate in groups of prime order and cannot directly be applied in groups of unknown order~\cite{PKC:BanCamMau05}.
% To circumvent these impossibility results, we follow the blueprint of \cite{C:BonBunFis19} and instantiate our protocols in the generic groups of unknown order~\cite{EC:DamKop02} with a common reference string.
% We detail our constructions in \Cref{app:sigmas}.

\paragraph{Syntax of Time-Locked Voting and Auction Protocols}
We now introduce a generic syntax for a time-locked voting/auction protocol. Any such protocol is defined with respect to a base \emph{scoring function} $\Score: \X^n \to \Y$ (e.g., second-price auction, range voting), which takes as input $n$ submissions (bids/ballots) $s_1, \dots, s_n$ in the submission domain $\X$ and computes the election/auction result $\Score(s_1, \dots, s_n) \in \Y$. It is useful to break down the scoring function into the ``tally'' or aggregation function $\tally: \X^n \to \X'$ and the finalization function $\final: \X' \to \Y$, i.e., $\Score = \final \circ \tally$.
For example, in first-past-the-post voting, the tally function $t$ is addition, and the finalization function $f$ is $\argmax$ over the final tally/bids.
%$\Sigma : \mathcal{X}^n \rightarrow \Y$ which operates over submissions in the clear. 

% \noemi{I think the following is actually unnecessary and I'm not even sure what it was going for...} In the case of an auction, $\Y\subseteq [n]\times\mathbb{Z}^m$. Intuitively, $\Y$ provides the aggregate result of how users voted for the candidates or the number of a specific item they obtained as a result of the auction. %Although it is in theory possible to efficiently open the aggregated $\htlp$(s) given every participant's private puzzle randomness, in practice this is unrealistic. We therefore omit the optimistic opening procedure in the $\Open$ algorithm below.

\begin{definition}[Time-locked voting/auction protocol]\label{def:syntax}
A time-locked voting/auction protocol $\Pi_\Sigma = (\Setup, \Seal,\allowbreak \Aggr, \Open, \Finalize)$ is defined with respect to a base voting/auction protocol $\Sigma = \final \circ \tally$, where $\tally : \X^n \to \X'$ and $\final : \X' \to \Y$. %where $n$ is the number of participants.

    \begin{description}
        \item[$\Setup(\secparam, \Ttime) \randout (\pp, \Z)$.] Given a security parameter $\secpar$ and a time parameter $\Ttime$, output public parameters $\pp$ and an initial list of HTLP(s) $\Z$ that corresponds to the running tally or bid computation.
        \item[$\Seal(\pp, i, s) \randout (\Z_i, \pi_i)$.] User $i\in[n]$ wraps its submission $s \in \X$ in a (list of) HTLP(s) $\Z_i$. It also outputs a proof of well-formedness $\pi_i$.
        \item[$\Aggr(\pp, \Z, i, \Z_i, \pi_i) \to \Z'$.] Given a list of (tally) HTLPs $\Z$, time-locked submission $\Z_i$ of user $i$, and proof $\pi_i$, the transparent contract potentially aggregates the sealed submission homomorphically into $\Z$ to get an updated (tally) $\Z' = \tally(\Z, \Z_i)$.
        \item[$\Open(\pp, \Z) \to (\mathcal{S}, \pi_\open)$.] Open $\Z$ to solution(s) $\mathcal{S}$, requiring $\Ttime$ sequential steps, and compute a proof $\pi_\open$ to prove correctness of $\mathcal{S}$.
        \item[$\Finalize(\pp, \Z, \mathcal{S}, \pi_\open) \to \{y, \perp\}$.] Given proposed solution(s) $\mathcal{S}$ to $\Z$ with proof $\pi_{\sf open}$, the coordinator may reject $\mathcal{S}$ or compute the final result $y = \final(\mathcal{S}) \in \Y$. %, which specifies the winner and, in the case of an auction, the amount to be paid as well as the appropriate items to be transferred.
    \end{description}
\end{definition}

We note that the $\Setup(\cdot)$ algorithm in our protocols may use private randomness. In particular, our constructions use cryptographic groups (RSA and Paillier groups) that cannot be efficiently instantiated without a trusted setup (an untrusted setup would require gigantic moduli~\cite{ICICS:Sander99}). This trust can be minimized by generating the group via a distributed trusted setup, e.g.,~\cite{JACM:BonFra01,SP:CHIKMRsVW21,TCC:DamMik10}.
Alternatively, the HTLPs in our protocols could be instantiated in class groups~\cite{CCS:TCLM21}, which do not require a trusted setup; however, HTLPs in class groups are less efficient and verifying them on-chain would be prohibitively costly.

A time-locked voting/auction protocol $\Pi_\Sigma$ must satisfy the following informal security properties, which we define formally in \cite{EPRINT:GSZB23}:
\begin{description}
    \item[Correctness.] $\Pi_\Sigma$ is \emph{correct} if, assuming setup, submission of $n$ puzzles, aggregation of all $n$ submissions, and opening are all performed honestly, the finalization procedure outputs a winner consistent with the base protocol $\Sigma$.
    \item[Submission privacy.] The scheme satisfies \emph{submission privacy} if the adversary cannot distinguish between two submissions, i.e., bids or ballots. Note that this property is only ensured up to time $\Ttime$.
    \item[Non-malleability.] Notice that submission privacy alone does not suffice for security: even without knowing the contents of other puzzles, an adversary could submit a value that depends on other participants' (sealed) submissions. For example, in an auction, one could be guaranteed to win by homomorphically computing an HTLP containing the sum of all the other participants' bids plus a small value $\epsilon$. Therefore, we also require \emph{non-malleability}, which requires that no participant can take another's submission and replay it or ``maul'' it into a valid submission under its own name.
\end{description}

\textit{A note on anonymity.} We consider user anonymity an orthogonal problem. In the applications we have in mind, users can increase their anonymity by using zero-knowledge mixers or other privacy-enhancing overlays, e.g., zero-knowledge sets~\cite{zksetsemaphore}. Additionally, users can decouple their identities from their ballots by applying a verifiable shuffle~\cite{CCS:Neff01}, although the on-chain verification of a shuffle proof might be prohibitively costly for larger elections. 
%In~\Cref{sec:everlasting_ballot_privacy} we describe how our protocols can be extended to achieve bid privacy even after the election ends, thus disclosing nothing besides a user's (non-)participation.

\paragraph{Efficient vector encoding for HTLPs.}
In many voting schemes, a ballot consists of a vector indicating the voter's relative preferences or point allocations for all $m$ candidates. To avoid solving many HTLPs, it is desirable to encode this vector into a single HTLP, which requires representing the vector as a single integer. % This motivates the following definition.

\begin{definition}[Packing scheme]\label{def:packing}
A setup algorithm $\PSetup$ and pair of efficiently computable bijective functions $(\pack,\unpack)$ is called a \emph{packing scheme} and has the following syntax:
    \begin{itemize}
        \item $\PSetup(\ell, w) \to \pp$. Given a vector dimension $\ell$ and maximum entry $w$, output public parameters $\pp$.
        \item $\pack(\pp, \Vec{a}) \to s$. Encode $\Vec{a} \in (\ZZ^+)^\ell$ as a positive integer $s \in \ZZ^+$. 
        \item $\unpack(\pp, s) \to \Vec{a}$. Given $s \in \ZZ^+$, recover a vector $\Vec{a} \in (\ZZ^+)^\ell$. 
    \end{itemize}
For \emph{correctness} we require $\unpack(\pack(\Vec{a}))=\Vec{a}$ for all $\Vec{a}\in (\ZZ^+)^\ell$.
\end{definition}

The classic approach to packing~\cite{ACNS:Groth05,EC:HirSak00} uses a \emph{positional numeral system (PNS)} to encode a vector of entries bounded by $w$ as a single integer in base $M := w$.
%More specifically, the vector $\vec{a}=(a_1, \dots, a_m)$ with $\forall j \in [m]:a_j < w$ is encoded as a sum of powers of $M$: the ballot contains a single integer $s := \sum_{j=1}^m a_j M^{j-1}$. Then $a_j$ can be obtained as $s \mod{M^{j-1}}$, 
Instead, we will set $M:= nw+1$ to accommodate the homomorphic addition of all $n$ users' vectors: each voter submits a length-$m$ vector with entries $\leq w$. Summing over $n$ voters, the result is a length-$m$ vector with a maximum entry value $nw$; to prevent overflow, we set $M = nw+1$.

% \begin{construction}[Packing from Positional Numeral System]\label{con:packingPNS}
% \hfill
% \begin{itemize}%[topsep=2pt]
%     \item $\PSetup(\ell, w) \to M:$ Return $M := w + 1$.
%     \item $\pack(M, \Vec{a}) \to s:$ Output $s := \sum_{j=1}^{\sizeof{\vec{a}}} a_j M^{j-1}$.
%     \item $\unpack(M, s) \to \Vec{a}:$ Let $\ell := \ceil{\log_M{s}}$. For $j \in [\ell]$, compute the $j$th entry of $\Vec{a}$ as $a_j := s \mod{M^{j-1}}$.
% \end{itemize}
% \end{construction}
% \end{mdframed}

We also introduce an alternative approach in \Cref{con:packingRNS} which is based on the \emph{residue numeral system (RNS)}. The idea of the RNS packing is to interpret the entries of $\vec{a}$ as prime residues of a single unique integer $s$, which can be found efficiently using the Chinese Remainder Theorem (CRT). In other words, for all $j \in [\ell]$, $s$ captures $a_j$ as $s \bmod p_j$.

\begin{construction}[Packing from Residue Numeral System]\label{con:packingRNS}
\hfill
\begin{itemize}%[topsep=2pt]
    % \textit{Public parameters:}~$\pp:=(p_1,\dots,p_m)$ primes s.t. $\min\limits_{j\in[m]}p_j\geq M(=nw+1)$.\\
    \item $\PSetup(\ell, w) \to \vec{p}:$ Let $M := w + 1$ and sample $\ell$ distinct primes $p_1, \dots, p_\ell$ s.t. $p_j \geq M\ \forall j \in [\ell]$. Return $\vec{p} := (p_1, \dots, p_\ell)$.
    \item $\pack(\vec{p}, \Vec{a}) \to s$: Given $\Vec{a} \in (\ZZ^+)^\ell$, use the CRT to find the unique $s \in \ZZ^+$ s.t. $s\equiv a_j \pmod{p_j}~\forall j\in[\ell]$.
    \item $\unpack(\vec{p}, s) \to \Vec{a}$: return $(a_1, \dots, a_\ell)$ where $a_j \equiv s \mod{p_j}\ \forall j \in [\ell]$.
\end{itemize}
\end{construction}

A major advantage of this approach is that, in contrast to the PNS approach, which is only homomorphic for SIMD (single instruction, multiple data) addition, the RNS encoding is fully SIMD homomorphic: the sum of vector encodings $\sum_{i \in [n]} s_i$ encodes the vector $\vec{a}_{+} = \sum_{i \in [n]} \vec{a}_i$, and the product $\prod_{i \in [n]} s_i$ encodes the vector $\vec{a}_{\times} = \prod_{i \in [n]} \vec{a}_i$. Note that as in the PNS approach, we set $M = nw + 1$ to accommodate homomorphic addition of submissions; homomorphic multiplication, however, would require $M = w^n+1$, and the primes in $\vec{p}$ would therefore be larger as well.
Although the RNS has found application in error correction~\cite{KPTOC22,TaiCha14}, side-channel resistance~\cite{TCHES:PFPB18}, and parallelization of arithmetic computations~\cite{AsiHosKon17,BajDuqMel06,GomTyaNam11,VNLVC20}, to our knowledge it has not been applied to voting schemes. We show in \cite{EPRINT:GSZB23} that RNS is in fact a natural fit for some voting schemes, in particular quadratic voting, where it results in more efficient proofs of ballot correctness. 

%%-------------BEGIN Cicada Framework FIGURE---------------
\begin{figure*}[tbh]
\begin{mdframed}
\begin{center}
    \textbf{The Cicada Framework}
\end{center}
Let $\Sigma: \X^n \rightarrow \Y$ be an linear voting/auction scheme
where 
% $\Sigma = f \circ g$ for some linear function $g$ and 
$\X = [0,w]^m$, $\htlp$ a linear HTLP, $\Ttime \in \NN$ be a time parameter representing the election/auction length, and a packing scheme $(\PSetup, \pack, \unpack)$.
Let $\NIZK$ be a NIZKPoK for 
submission correctness (the language depends on $\Sigma$ and $\htlp$)
% the language $\{(i, Z) : \exists x \text{ s.t. } Z \in {\sf Im}(\htlp.{\sf Gen}(\pack(x))) \land\ x \in \X\}$ 
and \poe\ a proof of exponentiation~\cite{ITCS:Pietrzak19b,EC:Wesolowski19}. 
%\noemi{is this better?}\istvan{LFG to me, but I think Joe wanted to have a list IIRC.}

\hrulefill
\begin{description}
    \item[$\Setup(\secparam, \Ttime, \ell) \randout (\pp, \Z)$.] 
    Set up the public parameters $\pp_{\NIZK} \sample \NIZK.\Setup(\secparam)$, $\pp_{\sf tlp} \sample \htlp.\Setup(\secparam, \Ttime)$, and $\pp_{\sf pack} \gets \PSetup(\ell, w)$. 
    Let $\Z = \{Z_j\}_{j \in [m/\ell]}$ where $Z_j \sample \htlp.{\sf Gen}(0)$. Output $\pp := (\pp_{\sf tlp}, \pp_{\sf pack}, \pp_\NIZK)$ and $\Z$.
    \item[$\Seal(\pp,i, \vec{v}_i) \randout (\Z_i, \pi_i)$.] Parse $\vec{v}_i := \vec{v}_{i,1} || \dots || \vec{v}_{i,m/\ell}$. Compute $Z_{i,j} \gets \htlp.{\sf Gen}(\pack(\vec{v}_{i,j}))~\forall j \in [m/\ell]$ and $\pi_i \gets \NIZK.\prove((i, \Z_i), \vec{v}_i)$.
    % $s_{i,j} \gets \pack(\vec{v}_{i,j})$ for all $j \in [m/\ell]$. 
    Output $(\Z_i := \{Z_{i,j}\}_{j \in [m/\ell]}, \pi_i)$ 
    \item[$\Aggr(\pp,\Z,i,\Z_i,\pi_i) \rightarrow \Z'$.] If $\NIZK.vrfy((i, \Z_i), \pi_i) = 1$, update $\Z$ to $\Z \boxplus \Z_i$. %, where $\boxplus$ is applied pairwise to elements of $\Z,\Z_i$.
    \item[$\Open(\pp,\Z) \rightarrow (\mathcal{S}, \pi_{\sf open})$.] Parse $\Z := \{Z_j\}_{j \in [m/\ell]}$ and solve for the encoded tally $\mathcal{S} = \{s_j\}_{j \in [m/\ell]}$ where $s_j \gets \htlp.{\sf Solve}(Z_j)$. Prove the correctness of the solution(s) as $\pi_{\sf open} \gets \poe.{\sf Prove}(\mathcal{S}, \Z, 2^\Ttime)$ and output $(\mathcal{S}, \pi_{\sf open})$.
    \item[$\Finalize(\pp, \Z, \mathcal{S}, \pi_{\sf open}) \rightarrow \{y,\perp\}$.] If $\poe.{\sf Verify}(\mathcal{S}, \Z, 2^\Ttime, \pi_{\sf open}) \neq 1$, return $\perp$. Otherwise, parse $S := \{s_j\}_{j \in [m/\ell]}$ and let $\Vec{v} := \vec{v}_1 || \dots || \vec{v}_{m/\ell}$, where $\vec{v}_j \gets \unpack(s_j)~\forall j \in [m/\ell]$. Output 
    % $y = f(\vec{v})$.
    $y$ such that $y = \Sigma(\vec{v})$.
\end{description}
\end{mdframed}
\caption{The Cicada framework for non-interactive private auctions and elections.}
\label{fig:cicada}
\end{figure*}
%%-------------END Cicada Framework FIGURE---------------

\paragraph{The Cicada framework.}
We present Cicada, our framework for non-interactive private auctions/elections, in \Cref{fig:cicada}. Cicada can be applied to voting and auction schemes where the scoring function $\Score$ has a linear tally function $t$. 
This includes the following schemes:

\textit{Additive voting.}
FPTP, approval, range, and cumulative voting are all examples of schemes with a linear tally function: each ballot (a length-$m$ vector) is simply added to the tally, and the finalization function $f$ is applied to the tally after the voting phase has ended to determine the winner. Borda-count ranked-choice voting~\cite{Emerson13} can also be expressed as a linear scheme, where each candidate is given a descending number of points based on preference and the vectors are again added. All of these schemes differ only in what qualifies as a ``proper'' ballot, which will be enforced by the NIZK.

\textit{Sealed-bid auctions.}\footnote{Locking up collateral is necessary for every (private) auction scheme. We treat the problem of collateral lock-up as an important but orthogonal problem and refer to~\cite{CCS:TAFWBM23} for an extensive discussion.}
The Cicada framework can also be used to implement a sealed-bid auction with a number of HTLPs which is independent of the number of participants $n$. 
% The following slight modification works as long as $M^n < \sizeof{\gp}$, where $M$ is the largest bid.
Assuming bids are bounded by $M$, we use an HTLP with solution space $\mathcal{S}$ such that $\sizeof{\mathcal{S}} > M^n$.
Each user $i$ submits $Z_i \gets \htlp.{\sf Gen}({bid}_i)$ and $\pi_i$, where $\pi_i$ proves $0 \leq {bid}_i \leq M$. A packing of the bids is computed at aggregation time, with $\Aggr$ updating $Z$ to $Z \boxplus (M^{i-1} \cdot Z_i)$. After the bidding phase, the final ``tally'' is opened to $s^*$ and the bids are recovered as ${\sf Bids} := \{ s^* \mod{M^{i-1}} \}_{i \in [n]}$. Any payment and allocation function can now be computed over the bids; in the simplest case, the winner is $\argmax_i ({\sf Bids})$ and their payment is $\max_i ({\sf Bids})$. Notice that the full set of bids is revealed after the auction concludes. This cannot be avoided when using Cicada with linear HTLPs, since $\max_i$ is a nonlinear function, i.e., it cannot be computed homomorphically.
% \noemi{This reminds me of the fact that our auction schemes are not bid-private since the `tally'' reveals all the bids, which we have to do because the finalize function $\max$ is not linear}

Cicada is instantiated with a linear HTLP, vector packing scheme, and matching NIZK for membership in $\X$ to ensure the correctness of submissions. 
We note that Cicada introduces a crucial design choice via the packing parameter $\ell\in[m]$, which defines a storage-computation trade-off that we discuss in \cite{EPRINT:GSZB23}. %The on-chain footprint of a protocol is minimized by using a ballot correctness proof system $\sf NIZK$ with low verification cost. % this seems kind of obvious? And if anything it belongs in the implementation section

% \noemi{In Cicada we need to talk about the function that is applied to the final tally, not the ballots -- maybe we could introduce a definition of ``linear'' protocols $\Sigma : \X^n \to \Y$ which are those functions that have an associated function $\Sigma_{\sf winner}: \X' \to \Y$ such that $\Sigma_{\sf winner}(t) = \Sigma(s_1, \dots, s_n)$ where $t = f(s_1, \dots, s_n)$ for some linear function $f$.}
%\noemi{difference between $\Sigma$ and $\Sigma_{\sf winner}$ is the domain, $\X^n$ and $\X$, resp.}


\begin{theorem}\label{thm:cicada}
    Given a linear scoring function $\Sigma$, a secure NIZKPoK $\sf NIZK$, a secure $\htlp$, and a packing scheme $({\sf PSetup}, \pack, \unpack)$, the Cicada protocol $\Pi_\Sigma$ (\Cref{fig:cicada}) is a secure time-locked voting/auction protocol. % i.e., it satisfies correctness, bid/ballot privacy, and non-malleability.
\end{theorem}

Intuitively, submission privacy follows from the security of the HTLP and zero-knowledge of the NIZK used: the submission can't be opened before time $\Ttime$ and none of the proofs leak any information about it. Non-malleability is enforced by requiring the NIZK to be a proof of knowledge and including the user's identity $i$ in the instance to prove, e.g., including it in the hash input of the Fiat-Shamir transform. This prevents a malicious actor from replaying a different user's ballot correctness proof.
% \istvan{Correctness is due to the correctness of the underlying NIZKs. Submission privacy is reduced to the HTLP sequentiality, while non-malleability is ensured by the proof of knowledge property of our applied NIZKs.} 
We delegate the full proof to \cite{EPRINT:GSZB23}.

% \subsection{Application to voting schemes}\label{sec:voting}

% \todo{revisit this} The main reason we are interested in building protocols for cardinal voting schemes is that they allow voters to express more fine-grained preferences. Put differently, they bypass Arrow's impossibility theorem~\cite{arrow1950difficulty}, i.e., cardinal voting schemes satisfy non-dictatorship, unrestricted domain, independence of irrelevant alternatives and Pareto efficient. 

\paragraph{Implementation.}
We provide open-source, freely available implementations tailored to the popular Ethereum Virtual Machine (EVM) with word size $256$ bits.\footnote{\url{https://github.com/a16z/cicada}} Our most efficient protocols work in $\ZZ_N^*$ for $N\approx 2^{1024}$, groups which are not natively supported by EVM. We implement several gas-efficient libraries to support modular arithmetic in such groups of unknown order. These protocols can be run today on Ethereum Layer 1. Our non-interactive protocols are particularly well-suited to the EVM since, unlike prior works, we do not need to keep bids, ballots, and proofs in persistent storage as they are not required for any subsequent rounds.

\section{Privacy-Enhancing Building Blocks}\label{sec:building-blocks}

Although cryptocurrencies are often treated as fully anonymous digital currencies, numerous works have shown how to link transactions or even fully deanonymize users on many popular blockchains \cite{CCS:BirKhoPus14,EuroSP:BirTik19,FC:KosKosMcD14,PoPETS:MSHLHSHHMNC18,USENIX:KYMM18}. Numerous mitigations have been suggested, including adding privacy as an overlay on top of existing blockchains (e.g., via cryptocurrency mixers, described in \cref{sec:mixers}), or by building new privacy-first cryptocurrencies like Zcash~\cite{zcash} and Monero~\cite{monero}. Many of these overlays and new blockchains rely on zero-knowledge proofs.

\subsection{Zero-Knowledge Proofs and their trust assumptions}

Non-interactive zero-knowledge proofs (NIZKs) are ubiquitous building blocks for achieving both privacy and scalability. Zcash~\cite{zcash} relies heavily on a type of NIZK called a zk-SNARK (zero-knowledge succinct non-interactive argument of knowledge) to prove that a party has sufficient funds to make a payment without revealing anything more about those funds~\cite{SP:BCGGMT14}. On Ethereum, so-called (zk-)rollups enhance scalability by leveraging the succinctness of (zk-)SNARKs, though they may or may not offer the zero-knowledge property.

To achieve such a high level of succinctness, SNARKs rely on a trusted setup to generate a \emph{common reference string (CRS)}. In keeping with the primary innovation of the blockchain, which is the elimination of a trusted third party, practitioners use various approaches to minimize trust in the CRS generation. Zcash uses a multi-party computation ceremony~\cite{zcash-ceremony} with many independent participants to distribute the trust among several parties. Another trust-minimizing approach consists of using SNARKs with universal and updatable CRS~\cite{C:GKMMM18,CCS:MBKM19,EC:CHMMVW20,EPRINT:GabWilCio19}. A universal CRS can be reused across applications, avoiding a new complicated setup ceremony for every use. Updatable CRSs allow any participant in a system to contribute randomness to the CRS at any point, including once the CRS is in production use, to enable a ``one-out-of-many'' trust scenario in which the user must only trust themselves to contribute (and then delete) good randomness to the CRS in order for the whole system to be secure.

An orthogonal concern is maintaining the security of SNARKs when they are composed with other protocols in the complex blockchain ecosystem. Formally, this is modeled by universally composable security via the universal composability (UC) framework~\cite{FOCS:Canetti01}. Unfortunately, most SNARKs in deployment today are not provably UC-secure. Although compilers to transform any SNARK or NIZK into a UC variant exist~\cite{EPRINT:KZMQCP15,EC:GKOPTT23}, these are not compatible with the aforementioned trust-minimizing properties like updatability. A generic compiler which adds UC-security while maintaining updatability would help ensure confidence in both the trusted setup and the operational security of deployed NIZKs.

\renewcommand{\cmark}{\CIRCLE}
\renewcommand{\xmark}{\Circle}
\begin{table}[tbh]
    \centering
    \begin{tabular}{l@{\hspace{1em}} cc cc c}
        \toprule
        & \multicolumn{2}{c}{UC} & \multicolumn{2}{c}{succinctness-preserving}    & \\ \cmidrule(r{3pt}){2-3} \cmidrule(l{3pt}){4-5}
        & SE     & BBE    & in $\lvert C \rvert$ & in $\lvert w \rvert$ & upd. CRS \\
        \midrule
        \COCO~\cite{EPRINT:KZMQCP15}    & \cmark & \cmark & \cmark           & \xmark           & \xmark\\
        DS~\cite{DCC:DerSla19}          & \cmark & \xmark & \cmark           & \cmark           & \xmark\\
        \textsc{Lamassu}~\cite{CCS:AbdRamSla20} & \cmark & \xmark & \cmark           & \cmark           & \cmark\\
        \midrule
        This work \cite{CSF:AGRS24}     & \cmark & \cmark & \cmark           & \xmark           & \cmark\\
        Concurr. work~\cite{EC:GKOPTT23} & \cmark & \cmark & \cmark           & \cmark           & \xmark\\
        \bottomrule
    \end{tabular}
    \caption{Comparison with concurrent and previous work. SE = simulation extractability, BBE = black-box extractability.}\label{tab:comparison}
\end{table}

\subsubsection{Circuit-succinct universally composable NIZKs with updatable CRS}

\textit{(Parts of this section are taken/adapted from \cite{CSF:AGRS24}.)}

\usetikzlibrary{fit}
\begin{figure}[tbh]
    \begin{center}
        \begin{tikzpicture}
            \node[draw] (crs_enc) {$\crs_\mathit{enc}$};
            \node[draw] (crs_snark) [right=3em of crs_enc] {$\crs_\mathit{SNARK}$};
            \node[draw] (crs_sig) [right=3em of crs_snark] {$\crs_\mathit{sig}$};
            %
            \node[draw,double] (ucrs_enc) [above=2em of crs_enc] {${\sf u}\crs_\mathit{enc}$};
            \node[draw,double] (ucrs_snark) [above=2em of crs_snark] {${\sf u}\crs_\mathit{SNARK}$};
            \node[draw,double] (ucrs_sig) [above=2em of crs_sig] {${\sf u}\crs_\mathit{sig}$};
            %
            \node[draw] (r_enc) [below=2em of crs_enc] {$\land \mathcal{R}_\mathit{enc}$};
            % pattern=north east lines,fill opacity=0.5
            \node[draw,fill=gray!20] (r_snark) [below=2em of crs_snark] {$\mathcal{R}_\mathcal{L}$};
            \node[draw] (r_sig) [below=2em of crs_sig] {$\lor \mathcal{R}_\mathit{sig}$};
            %
            \node[] (center) [right=1em of r_snark] {};
            \node[] (dspadding) [below=1em of center] {};
            \node[draw,fill=blue,opacity=0.2,inner sep=7pt,fit=(crs_snark) (crs_sig) (r_snark) (r_sig) (dspadding)] (ds) {};
            \node[above] at (ds.south) {DS: non-BB SE};
            %
            \node[] (lamassupadding) [below=2.5em of center] {};
            \node[draw,fill=green,opacity=0.2,inner xsep=15pt,inner ysep=10pt,fit=(ucrs_snark) (ucrs_sig) (r_snark) (r_sig) (lamassupadding)] (lamassu) {};
            \node[above] at (lamassu.south) {\textsc{Lamassu}: upd. non-BB SE};
            %
            % \node[] (topmargin) [above=.5em of crs_snark] {};
            \node[] (cocopadding) [below=3.5em of r_snark] {};
            \node[draw,fill=yellow,opacity=0.2,inner xsep=25pt,inner ysep=13pt,fit=(crs_enc) (crs_sig) (r_enc) (r_sig) (cocopadding)] (coco) {};
            \node[above] at (coco.south) {\COCO: BB SE (UC)};
            %
            \node[] (bblamassupadding) [below=5em of r_snark] {};
            \node[draw,inner xsep=35pt,inner ysep=15pt,fit=(ucrs_enc) (ucrs_sig) (r_enc) (r_sig) (bblamassupadding)] (bblamassu) {};
            \node[above] at (bblamassu.south) {This work: upd. BB SE (UC)};
            %
            \draw[->] (crs_enc)--(ucrs_enc);
            \draw[->] (crs_snark)--(ucrs_snark);
            \draw[->] (crs_sig)--(ucrs_sig);
        \end{tikzpicture}
    % trim=left bottom right top
    % \includegraphics[trim=5.5cm 2.5cm 5cm 2cm, clip, scale =0.5]{overview_new}
    \end{center}
    \caption{Overview of our approach including previous work from Table~\ref{tab:comparison}. ${\sf u}\crs$ denotes an updatable CRS.}
    \label{fig:ucse-overview}
\end{figure}

In \cite{CSF:AGRS24} we show how to build such a compiler by giving the first \emph{fully black-box approach to generically build} circuit-succinct UC-secure NIZKs with updatable CRS from any zk-SNARK, circumventing the aforementioned problems. \textsc{BB-Lamassu} is a framework for black-box (BB) simulation-extractable (SE) (i.e., universally-composable) circuit-succinct NIZKs with updatable CRS. It can be seen as a hybrid of \COCO~\cite{EPRINT:KZMQCP15} and \textsc{Lamassu}~\cite{CCS:AbdRamSla20}, combining the BB extractability of the former with the updatable CRS compatibility of the latter (see \Cref{tab:comparison}). 
% We follow an approach similar to \COCO\ (grey box) to achieve BB extractability, which as mentioned above requires relaxing the succinctness of SNARKs to that of circuit-succinct NIZKs~\cite{EC:KNYY20} since we need to encrypt the witness in the proof.

\COCO\ (yellow box in \Cref{fig:ucse-overview}) combines two tricks from the literature to achieve BBE and SE. First, to avoid non-black-box extractors that rely on rewinding or knowledge assumptions, they extend the CRS by a public key and include an encryption of the witness in the proof. This also requires extending the original statement to show that the correct witness was encrypted ($\mathcal{R}_{enc}$)~\cite{FOCS:DeSPer92}. Now the extractor can extract the witness by simply decrypting the ciphertext in the CRS. Second, they use the classical OR trick (i.e., an alternate clause $\mathcal{R}_{sig}$, to be used only by the simulator, which checks for a valid signature) to enable unbounded simulation of proofs~\cite{C:DDOPS01}, i.e., SE with a straight-line extractor. Together, these additions augment the underlying scheme with SE and UC-security, but the new ciphertext in the CRS ($\crs_{enc}$) adds a linear overhead of $\sizeof{w}$, so it does not fully preserve succinctness and can only give UC SE \emph{NIZKs}.

\textsc{Lamassu} (green box) revisits the \COCO\ framework, tailoring it to updatable NIZKs and removing the $\sizeof{w}$ overhead. This is achieved by using the non-black-box extractor of the underlying scheme instead of an encryption of the witness, thus preserving succinctness (modulo some small constant overhead). To make unbounded proof simulation compatible with an updatable CRS, \textsc{Lamassu} adapts the simulation technique of \cite{DCC:DerSla19} (blue box), which used the OR trick to combine the underlying SNARK's non-BB extractor with key-homomorphic signatures. To support an updatable CRS, \textsc{Lamassu} swaps the signature for an updatable signature (US). The result is a generic framework for CRS-updatable SE succinct NIZKs, but it sacrifices UC-security due to the non-black-box extractor.

\paragraph{BB-Lamassu.} Our new framework BB-\textsc{Lamassu} (outside box) uses \textsc{Lamassu} as a starting point and adds back in an encryption of the witness ($\mathcal{R}_{enc}$) for BB-extractability. To be compatible with updatability, we instantiate this with a novel public-key encryption (PKE) primitive which we call \emph{extractable key-updatable PKE (EKU-PKE)}, for which we show an efficient construction. 
We still have to overcome the hurdle of providing BB extraction for the US and the public key of the EKU-PKE in the CRS ($crs_{enc}$). In brief, this is done by using an efficient (but not necessarily succinct) BBE NIZK \emph{without a CRS} to prove updates of the CRS elements, i.e., updates of $crs_{SNARK}$, the US public key $crs_{sig}$, and the EKU-PKE public key $crs_{enc}$. 
% \copied{We choose to base these proofs on $\Sigma$-protocols converted to NIZK proofs using either the Fiat-Shamir (FS)~\cite{C:FiaSha86}, Fischlin~\cite{C:Fischlin05} or Unruh~\cite{EC:Unruh15} approach. While this requires that the updates of all components are $\Sigma$-protocol friendly, this holds true for the relations in all known constructions. Interestingly, a byproduct of this approach is that the update proofs for the underlying SNARK CRS become much more efficient to verify (and typically also much smaller). This improvement also carries over to the original \textsc{Lamassu} framework~\cite{CCS:AbdRamSla20} and can be used to improve their CRS update proofs as well.}

\paragraph{UC security proof.}
Since BB-\textsc{Lamassu} is BB SE, it is also UC-secure and should therefore realize the NIZK ideal functionality $\fnizk$ of \cite{AC:Groth06}. However, so far this ignores the updatable CRS aspect. 
To formally confirm this intuition, we introduce a new ideal functionality $\fupcrs$ for the updatable CRS generation and then prove that BB-\textsc{Lamassu} realizes the functionality $\fnizk$ in the $\fupcrs$-hybrid model.
Our analysis is carried out in the local ROM, which can be realized in practice by domain separation in the hash function. We note that the use of an RO arises from a building block (the proof of CRS update) and not from the construction of our compiler. %While we currently consider only the local ROM, we expect that an analysis in the global ROM is possible when relying on Fischlin for the update proofs via the techniques in \cite{TCC:LysRos22}.

\begin{figure}[tb]
    \includegraphics[width=\linewidth]{ucse-benchmarks.png}
    \caption{Runtimes of the Prove and Verify algorithms for Pedersen and SHA-256 preimages using our BB SE succinct NIZK compared to the non-BB SE zk-SNARK obtained via \textsc{Lamassu}~\cite{CCS:AbdRamSla20} and the base non-SE zk-SNARK Sonic~\cite{CCS:MBKM19}. In the lower part we plot the overhead of our transformation to add BB SE, which decreases as the witness size increases.}
    % We do not plot Helped Verify times because they are very small (on the order of hundreds of $\mu$s) and any pattern is overshadowed by noise.
    \label{fig:ucse-eval}
\end{figure}

\paragraph{Performance evaluation.} To demonstrate the applicability of BB-\textsc{Lamassu}, we provide a detailed analysis of the induced overheads. 
For concrete instantiations, we estimate overheads of 32 bytes for the CRS, 170 bytes for the CRS update, and 256 bytes plus the size of the witness for the proof. This is a reduction in both storage and runtime overheads compared to \textsc{Lamassu}~\cite{CCS:AbdRamSla20}.
For witness sizes observed in practical applications such as Zcash, BB-\textsc{Lamassu} adds well below 10,000 additional constraints.

As a concrete example, we apply BB-\textsc{Lamassu} to Sonic~\cite{CCS:MBKM19}, a zk-SNARK with updatable CRS.\footnote{\url{https://github.com/nglaeser/sonic-ucse/}} 
We then experimentally evaluate the overhead introduced by BB-\textsc{Lamassu}. For a SHA-256 preimage, which is interesting for Merkle-tree membership proofs, the prover and verifier overhead, respectively, is $\approx 1.2\times$ and $1.07\times$. Our evaluation shows that as the circuits become larger and more complex, proving and verifying the original circuit dominates the overall performance costs and the overhead added by BB-\textsc{Lamassu} converges to the size of the witness (see \Cref{fig:ucse-eval}).

\section{Privacy-Enhancing Applications}\label{sec:applications}

Because the blockchain itself does not guarantee the privacy of users, privacy is often added at the application layer. For example, many applications serve as a privacy overlay to the underlying payment layer. Other applications, such as private voting, enable a wholly new functionality while also adding privacy.

\section{Blind Conditional Signatures}\label{sec:bcs-defs}

Next, we formally define and instantiate blind conditional signatures, the central cryptographic notion for coin mixing services. 
Our goal here is to give a simple and easy-to-understand formalization of a \syncpuzzle. 

% \subsection{Definitions}
A blind conditional signature (BCS) is executed among users Alice, Bob, and Hub. The interfaces and associated security properties are defined below.
% \todo{Notation note: \aal paper uses $m'$ for the promise and $m$ for the solver, which is the opposite of what we do here. Personally I think something like $m_\hb$ and $m_\ah$ (or $m^\hb,m^\ah$) is clearer.}

\begin{definition}[Blind Conditional Signature]
A blind conditional signature $\Pi_\mathsf{BCS} :=(\Setup, \Promise, \Pay, \open)$ is defined with respect to a signature scheme $\Pi_\DS := (\kgen, \sign, \vrfy)$ and consists of the following efficient algorithms.

\begin{itemize}
    \item {$(\tilde{\ek}, \tilde{\dk})\gets \Setup(\secparam)$}: The setup algorithm takes as input the security parameter $\secparam$ and outputs a key pair $(\tilde{\ek}, \tilde{\dk})$.
    \item {$(\bot, \{\tau, \bot\}) \gets \Promise \left\langle \begin{matrix} H \left(\tilde{\dk}, \sk^H, m_\hb \right)\\ \bob \left(\tilde{\ek}, \vk^H, m_\hb \right) \end{matrix} \right\rangle $}: The puzzle promise algorithm is an interactive protocol between two users $H$ (with inputs the decryption key $\tilde \dk$, the signing key $\sk^H$, and a message $m_\hb$) and $\bob$ (with inputs the encryption key $\tilde \ek$, the verification key $\vk^H$, and a message $m_\hb$)  and returns $\bot$ to $H$ and either a puzzle $\tau$ or $\bot$ to $B$.
    \item {$(\{(\sigma^*, s), \bot\}, \{\sigma^*, \bot\}) \gets \Pay \left\langle \begin{matrix} \alice \left(\sk^A, \tilde{\ek}, m_\ah, \tau\right)\\ H \left(\tilde{\dk}, \vk^A, m_\ah\right) \end{matrix} \right\rangle $}: The puzzle solving algorithm is an interactive protocol between two users $\alice$ (with inputs the signing key $\sk^A$, the encryption key $\tilde \ek$, a message $m_\ah$, and a puzzle $\tau$) and $H$ (with inputs the decryption key $\tilde \dk$, the verification key $\pk^A$, and a message $m_\ah$) and returns to both users either a signature $\sigma^*$ ($\alice$ additionally receives a secret $s$) or $\bot$.
    \item {$\{\sigma, \bot\}\gets \open(\tau, s)$}: The open algorithm takes as input a puzzle $\tau$ and a secret $s$ and returns a signature $\sigma$ or $\bot$.
\end{itemize}
\end{definition}

Next, we define correctness.

\begin{definition}[Correctness]
A blind conditional signature $\Pi_\mathsf{BCS}$ is correct if for all $\secpar\in \mathbb{N}$, all $(\tilde{\ek}, \tilde{\dk})$ in the support of $\Setup(\secparam)$, all $(\vk^H, \sk^H)$ and $(\vk^A, \sk^A)$ in the support of $\Pi_\DS.\kgen(\secparam)$, and all pairs of messages $(m_\hb, m_\ah)$, it holds that
$$
\Pr\left[\vrfy(\vk^H,m_\hb, \open(\tau, s)) = 1\right] = 1
$$
and
$$
\Pr\left[\vrfy(\vk^A,m_\ah, \sigma^*) = 1\right] = 1
$$
where 
\begin{itemize}
    \item $\tau \gets \Promise \left\langle \begin{matrix} H \left(\tilde{\dk}, \sk^H, m_\hb \right)\\ \bob \left(\tilde{\ek}, \vk^H, m_\hb\right) \end{matrix} \right\rangle$ and
    \item $((\sigma^*, s), \sigma^*) \gets \Pay \left\langle \begin{matrix} \alice \left(\sk^A,  \tilde{\ek}, m_\ah, \tau\right)\\ H \left(\tilde{\dk}, \vk^A, m_\ah\right) \end{matrix} \right\rangle$.
\end{itemize}
\end{definition}

We now present the security guarantees of BCS in the game-based setting.
Our definition of blindness is akin to the strong blindness notion of standard blind signatures~\cite{C:Chaum82}, in which the adversary picks the keys (as opposed to the weak version in which they are chosen by the experiment)\footnote{We opt for this stronger version since we want to provide anonymity even in the case of a fully malicious hub, which can pick its keys adversarially to try to link a sender/receiver pair.}. Roughly speaking, it says that two promise/solve sessions cannot be linked together by the hub.\footnote{We do not consider the case in which Hub colludes with either Alice or Bob, since deanonymization is trivial (Alice (resp. Bob) simply reveals the identity of Bob (resp. Alice) to Hub); this is in line with \cite{SP:TaiMorMaf21}.}

\begin{definition}[Blindness]
A blind conditional signature $\Pi_\mathsf{BCS}$ is blind if there exists a negligible function $\negl$ such that for all $\secpar \in \NN$ and all PPT adversaries $\adv$, the following holds:
\[ \prob{\expUnlink^{\adv}_{\Pi_{\mathsf{puzzle}}}(\secpar) = 1} \le \frac{1}{2} + \negl\]
where $\expUnlink$ is defined in~\Cref{fig:exp_unlinkability}.\footnote{In previous works, descriptions of unlinkability assume an explicit step for blinding the puzzle $\tau$ between $\Promise$ and $\Pay$. Here, we assume that $\Pay$ performs this blinding functionality.}
\end{definition}

\begin{figure}[tbh]
    \centering
    \begin{pchstack}[boxed]
    \procedure[space=auto,codesize=\small]{$\expUnlink^{\adv}_{\Pi_{\mathsf{BCS}}}(\secpar)$}{
    \hfill\\[-0.5\baselineskip]
   (\tilde{\ek}, \vk^H_0, \vk^H_1, (m_{\hb,0}, m_{\ah,0}), (m_{\hb,1}, m_{\ah,1})) \gets \adv(\secparam)\\
    (\vk_0^A, \sk_0^A) \gets \kgen(\secparam)\\
    (\vk_1^A, \sk_1^A) \gets \kgen(\secparam)\\
    \tau_0 \gets \Promise \left\langle  \adv(\vk_0^A, \vk_1^A), \bob(\tilde{\ek}, \vk^H_0, m_{\hb,0}) \right\rangle\\
    \tau_1 \gets \Promise \left\langle  \adv(\vk_0^A, \vk_1^A), \bob(\tilde{\ek}, \vk^H_1, m_{\hb,1}) \right\rangle\\
    b \gets \{0,1\}\\
    (\sigma^*_0, s_0) \gets \Pay \left\langle  \alice \left(\sk_0^A, \tilde{\ek}, m_{\ah,0}, \tau_{0 \oplus b}\right), \adv \right\rangle \\
    (\sigma^*_1, s_1) \gets \Pay \left\langle  \alice \left(\sk_1^A, \tilde{\ek}, m_{\ah,1}, \tau_{1 \oplus b}\right), \adv \right\rangle \\
    \mathbf{if~}(\sigma^*_0 = \bot) \lor (\sigma^*_1 = \bot) \lor (\tau_0 = \bot) \lor (\tau_1 = \bot)\\
    \quad \sigma_0 := \sigma_1 := \bot\\
    \mathbf{else}\\
    \quad \sigma_{0 \oplus b} \gets \open(\tau_{0 \oplus b}, s_0)\\
    \quad \sigma_{1 \oplus b} \gets \open(\tau_{1 \oplus b}, s_1)\\
    b' \gets \adv(\sigma_0, \sigma_1)\\
    \mathbf{return}\ (b = b')
    }
\end{pchstack}
\caption{Blindness experiment \label{fig:exp_unlinkability}}
\end{figure}

Next, we define unlockability, which says that it should be hard for Hub to create a valid signature on Alice's message that does not allow Bob to unlock the full signature in the corresponding promise session.

\begin{definition}[Unlockability]
A blind conditional signature $\Pi_\mathsf{BCS}$ is unlockable if there exists a negligible function $\negl$ such that for all $\secpar \in \NN$ and all PPT adversaries $\adv$, the following holds:
\[ \prob{\expUnlock^{\adv}_{\Pi_{\mathsf{BCS}}}(\secpar) = 1} \le \negl\]
where $\expUnlock$ is defined in~\Cref{fig:exp_unlockability}.
\end{definition}

\begin{figure}[tbh]
    \centering
    \begin{pchstack}[boxed]
    \procedure[space=auto,codesize=\small]{$\expUnlock^{\adv}_{\Pi_{\mathsf{BCS}}}(\secpar)$}{
    \hfill\\[-0.5\baselineskip]
    (\tilde{\ek}, \vk^H, m_\hb, m_\ah) \gets \adv(\secparam)\\
    (\vk^A, \sk^A) \gets \kgen(\secparam)\\
    \tau \gets \Promise \left\langle  \adv(\vk^A), \bob(\tilde{\ek}, \vk^H, m_\hb) \right\rangle\\
    \textbf{if } \tau = \bot\\
    \quad (\hat{\sigma}, \hat{m}) \gets \adv\\ 
    \quad b_0 := (\vrfy(\vk^A,\hat{\sigma}, \hat{m}) =1)\\
    \textbf{if } \tau \neq \bot\\
        \quad (\sigma^*,s) \gets \Pay \left\langle  \alice \left(\sk^A, \tilde{\ek}, m_\ah,\tau\right), \adv \right\rangle\\
    \quad (\hat{\sigma}, \hat{m}) \gets \adv\\ 
    \quad b_1 := (\vrfy(\vk^A,\hat{\sigma}, \hat{m}) =1) \land (\hat{m} \neq m_\ah)\\
     \quad b_2 := (\vrfy(\vk^A,\sigma^*, m_\ah) =1)\\
    \quad b_3 := (\vrfy(\vk^H, m_\hb, \open(\tau,s)) \neq1) \\
     \mathbf{return}\ b_0 \lor b_1 \lor (b_2 \land b_3)
    }
\end{pchstack}
    \caption{Unlockability experiment} 
    \label{fig:exp_unlockability}
\end{figure}


Our definition of unforgeability is inspired by the unforgeability of blind signatures~\cite{C:Chaum82}. We require that Alice and Bob cannot recover $q$ signatures from Hub while successfully querying the solving oracle at most $q-1$ times. Since each successful query reveals a signature from Alice's key (which in turn corresponds to a transaction from Alice to Hub), this requirement implicitly captures the fact that Alice and Bob cannot steal coins from Hub. The winning condition $b_0$ captures the scenario where the adversary forges a signature of the hub on a message previously not used in any promise oracle query.
The remaining conditions $b_1, b_2$ and $b_3$ together capture the scenario in which the adversary outputs $q$ valid distinct key-message-signature tuples while having queried for solve only $q-1$ times. Hence, in the second condition, the attacker manages to \emph{complete} $q$ promise interactions with only $q-1$ solve interactions, whereas in the first winning condition, the adversary computes a \emph{fresh} signature that is not the completion of any promise interaction. These conditions are technically incomparable: an attacker that succeeds under one condition does not imply an attacker succeeding on the other.
It is important to note that this is different from the unforgeability notion of blind signatures (where the attacker only has access to a single signing oracle), since in our case the hub is offering the attacker two oracles: promise and solve.

\begin{definition}[Unforgeability]
A blind conditional signature $\Pi_\mathsf{BCS}$ is \emph{unforgeable} if there exists a negligible function $\negl$ such that for all $\secpar \in \NN$ and all PPT adversaries $\adv$, the following holds:
\begin{equation*}
     \prob{\expSec^{\adv}_{\Pi_{\mathsf{BCS}}}(\secpar) = 1} \le \negl
\end{equation*}
where $\expSec$ is defined in~\Cref{fig:exp_security_ab}.
\end{definition}

\begin{figure}[tbh]
    \centering
    \begin{pchstack}[boxed]
    \begin{pcvstack}
    \procedure[space=auto,codesize=\small]{$\expSec^{\adv}_{\Pi_{\mathsf{BCS}}}(\secpar)$}{
    \hfill\\[-0.5\baselineskip]
\LL := \emptyset, Q := 0 \\
(\tilde{\ek}, \tilde{\dk}) \gets \Setup(\secparam)\\
(\vk_1^H, m_1, \sigma_1), \ldots, (\vk_q^H, m_q, \sigma_q)  \gets \adv^{\oracle\mathsf{PP}(\cdot), \oracle\mathsf{PS}(\cdot)}(\tilde{\ek})\\
b_0 := \exists i \in [q] \suchthat (\vk_i^H,\cdot) \in \LL \land (\vk_i^H,m_i) \notin \LL \\
\qquad \qquad  \land \vrfy(\vk_i^H,m_i,\sigma_i) = 1\\
b_1 := \forall i \in [q], (\vk_i^H,m_i) \in \LL \land \vrfy(\vk_i^H, m_i, \sigma_i) = 1\\
b_2 := \bigwedge_{i,j \in [q], i \ne j}  (\vk_i^H, m_i, \sigma_i) \ne (\vk_j^H, m_j, \sigma_j) \\
b_3 := (Q \le q-1) \\
\mathbf{return}\ b_0 \lor (b_1 \land b_2 \land b_3)
}
\pcvspace

\procedure[space=auto,codesize=\small]{$\oracle\mathsf{PP}(m)$}{
\hfill\\[-0.5\baselineskip]
(\vk^H, \sk^H) \gets \Pi_\ADP.\kgen(\secparam)\\
\LL := \LL \cup \{ (\vk^H,m) \}\\
\bot \gets \Promise\langle H(\tilde{\dk}, \sk^H, m), \adv(\vk^H) \rangle \\
}


\procedure[space=auto,codesize=\small]{$\oracle\mathsf{PS}(\vk^A, m')$}{
\hfill\\[-0.5\baselineskip]
\sigma^* \gets \Pay\langle \adv, H(\tilde{\dk}, \vk^A, m') \rangle \\
\mathbf{if}\ \sigma^* \ne \bot\ \mathbf{then}~ Q:= Q+ 1
}
    \end{pcvstack}
\end{pchstack}
    \caption{Unforgeability experiment}
    \label{fig:exp_security_ab}
\end{figure}

We define security as the collection of all properties.
\begin{definition}[Security]
A blind conditional signature $\Pi_\mathsf{BCS}$ is secure if it is blind, unlockable, and unforgeable.
\end{definition}
\subsection{On-chain private voting}

\subsubsection{Contribution: Cicada, a framework for private non-interactive on-chain auctions and voting}

In this section, we summarize the contributions and constructions of~\cite{EPRINT:GSZB23}. \todo{...}


\section{Proposed Work}\label{sec:proposed}
\subsection{Threshold cryptocurrency wallets in the hot-cold paradigm}

\subsection{SoK: On-chain key management}\label{sec:sok}

In \cite{CCS:GKMR23} we introduced the first efficient construction of registration-based encryption (RBE). RBE can be seen as a transparent version of identity-based encryption (IBE)~\cite{C:Shamir84} in which the trusted third party (TTP) is replaced with an untrusted, transparent \emph{key curator} (KC). This allows encryption directly to the identity (i.e., some unique identifying string such as a username or phone number) of any party registered in the system. The transparent nature of the KC, who simply accumulates the public keys of registering parties into a succinct and publicly available key registry ``digest'', makes RBE a natural fit to the blockchain. With an efficient construction now available, such a deployment is a realistic possibility.

An important problem in all secure systems is correctly linking identities to public keys. The usual approach in classic networks is some kind of public key infrastructure (PKI), but the trustless and decentralized nature of blockchains offers alternative trust-minimized approaches. I propose a detailed analysis of the tradeoffs between various approaches to key distribution that can be deployed on-chain in the form of a systematization of knowledge (SoK). These include a simple public-key registry (PKR) of identity-public key pairs (the existing Ethereum Name Service (ENS)~\cite{ens} is an example), (threshold) IBE, and RBE. For the case of sending encrypted messages, \Cref{tab:pki-comparison} gives an initial comparison of a simple public-key registry (where a list of identity-public key pairs are stored on-chain), the Boneh-Franklin IBE~\cite{C:BonFra01} with a secret-shared master secret key, and our efficient RBE construction~\cite{CCS:GKMR23}.

\newcommand{\med}{\LEFTcircle}
\begin{table}[htb]
    \centering
    \begin{tabular}{lccc}
        \toprule
            & Public-key registry & Threshold IBE & RBE \\
        \midrule
        Succinct on-chain storage    & \xmark & \cmark & \med   \\
        Non-interactive encryption   & \med   & \cmark & \med   \\
        Non-interactive decryption   & \cmark & \cmark & \med   \\
        Sender-anonymous             & \med   & \cmark & \cmark \\
        Recipient-anonymous          & \cmark & \xmark & \med   \\
        No TTP                       & \cmark & \med   & \cmark \\
        Arbitrary IDs                & \cmark & \cmark & \cmark$^*$ \\
        \bottomrule
    \end{tabular}
    \caption{Properties of an on-chain public key registry, threshold IBE~\cite{C:BonFra01}, and RBE~\cite{CCS:GKMR23}. ``Succint'' storage means constant (\cmark) or sublinear (\med) in the number of parties, non-interactive means only once (\cmark) or infrequently (\med) in the lifetime of the system.
    Since the PKR keeps the full list of keys on-chain, it does not have succinct storage; furthermore, the sender occasionally has to retrieve the public key for a new recipient, so encryption and sender-anonymity are not fully achieved. Meanwhile threshold IBE ciphertexts do not normally hide the recipient~\cite{EC:BLSV18} and rely on a set of TTPs. Finally, recipient-anonymity can be added to the RBE construction (\med) as an extension; the asterisk indicates that arbitrary identity strings for RBE are enabled by a follow-up work~\cite{AC:FioKolPer23}.}\label{tab:pki-comparison}
\end{table}

A more in-depth evaluation would include various proposed alternatives to storing the full PKR on-chain while maintaining transparency~\cite{USENIX:MBBFF15,CCS:CDGM19,FCW:Bonneau16b,SP:TomDev17,EPRINT:MKSGOLL23}. Furthermore, our basic RBE construction did not consider key updates or revocation, which are important in practice, but recent work~\cite{AC:FioKolPer23} has made progress in that direction and should be incorporated into the analysis. I am also working on a smart contract implementation of the RBE construction to get a more accurate comparison of the on-chain costs of the aforementioned approaches.
% \subsection{Formalizing Bolt}

\section{Timeline}

\paragraph{January 2024:} Preliminary exam (Jan. 22); submit Cicada (\Cref{sec:cicada}) and threshold hot-cold wallet (\Cref{sec:ksp}) to CCS (Jan. 28).

\paragraph{March 2024:} Publish a16z blog post about on-chain RBE along with smart contract implementation.

\paragraph{Summer/Fall 2024:} Submit SoK (\Cref{sec:sok}) to IEEE S\&P'25 or FC'25.

\paragraph{December 2024:} Dissertation defense.

{\small
\bibliographystyle{alpha}
\bibliography{
    cryptobib/abbrev3,
    cryptobib/crypto,
    extrarefs
}
}
\end{document}