\chapter{Conclusion}\label{sec:conclusion}

In this dissertation, I gave three ways in which advanced cryptography can improve the security and trust assumptions of blockchain applications in a practical and deployable manner: naysayer proofs, the Cicada framework for fair and non-interactive voting and auctions, and \sysname, a BLS-based \hcwl. Given that these works have had significant industry discussion and/or collaboration, it is my hope that they will also see practical deployment in the near future. 

The ideas discussed in this dissertation also have significant potential to spark additional discussion and future work. The naysayer paradigm can be substituted for classic zero-knowledge proofs in any application which uses them as a building block. A more theoretical analysis of naysayer complexity, lower bounds, and application to particular classes of underlying verifiers or popular proof systems may also be fruitful. Finally, like optimistic rollups, naysayer proofs would benefit from a rigorous analysis of how to set collateral and delay periods in optimistic settings.

Next, I showed that the Cicada framework is a useful blueprint for fair on-chain elections and auctions and demonstrated its practicality for many popular protocols. Extending Cicada to other, non-additive scoring protocols could be a useful future work. Each of the proposed extensions can be explored in greater detail as well.

Finally, the \hcwl protocol \sysname is designed for a popular threshold signature scheme
% cover two popular threshold signature schemes 
used in the blockchain ecosystem today. It may be desirable or necessary to extend these ideas to other threshold signatures (e.g., pairing-free signatures like Schnorr). Furthermore, although \sysname already offers several advanced functionalities, deployments may wish to extend it even more, e.g., by adding distributed key generation, hot share refreshes, and secret resharing.

Cryptography has seen significant deployment in the blockchain ecosystem and driven some of its most important innovations. As the space continues to grow, I expect that cryptographic protocols and primitives will continue to play a large role in furthering security, decentralization, and scalability.